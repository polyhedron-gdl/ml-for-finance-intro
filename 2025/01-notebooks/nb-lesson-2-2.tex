% Default to the notebook output style

    


% Inherit from the specified cell style.

\documentclass{article}


    

    % Basic figure setup, for now with no caption control since it's done
    % automatically by Pandoc (which extracts ![](path) syntax from Markdown).
    \usepackage{graphicx}
    % Maintain compatibility with old templates. Remove in nbconvert 6.0
    \let\Oldincludegraphics\includegraphics
    % Ensure that by default, figures have no caption (until we provide a
    % proper Figure object with a Caption API and a way to capture that
    % in the conversion process - todo).
    \usepackage{caption}
    \DeclareCaptionFormat{nocaption}{}
    \captionsetup{format=nocaption,aboveskip=0pt,belowskip=0pt}

    \usepackage{float}
    \floatplacement{figure}{H} % forces figures to be placed at the correct location
    \usepackage{xcolor} % Allow colors to be defined
    \usepackage{enumerate} % Needed for markdown enumerations to work
    \usepackage{geometry} % Used to adjust the document margins
    \usepackage{amsmath} % Equations
    \usepackage{amssymb} % Equations
    \usepackage{textcomp} % defines textquotesingle
    % Hack from http://tex.stackexchange.com/a/47451/13684:
    \AtBeginDocument{%
        \def\PYZsq{\textquotesingle}% Upright quotes in Pygmentized code
    }
    \usepackage{upquote} % Upright quotes for verbatim code
    \usepackage{eurosym} % defines \euro

    \usepackage{iftex}
    \ifPDFTeX
        \usepackage[T1]{fontenc}
        \IfFileExists{alphabeta.sty}{
              \usepackage{alphabeta}
          }{
              \usepackage[mathletters]{ucs}
              \usepackage[utf8x]{inputenc}
          }
    \else
        \usepackage{fontspec}
        \usepackage{unicode-math}
    \fi

    \usepackage{fancyvrb} % verbatim replacement that allows latex
    \usepackage{grffile} % extends the file name processing of package graphics
                         % to support a larger range
    \makeatletter % fix for old versions of grffile with XeLaTeX
    \@ifpackagelater{grffile}{2019/11/01}
    {
      % Do nothing on new versions
    }
    {
      \def\Gread@@xetex#1{%
        \IfFileExists{"\Gin@base".bb}%
        {\Gread@eps{\Gin@base.bb}}%
        {\Gread@@xetex@aux#1}%
      }
    }
    \makeatother
    \usepackage[Export]{adjustbox} % Used to constrain images to a maximum size
    \adjustboxset{max size={0.9\linewidth}{0.9\paperheight}}

    % The hyperref package gives us a pdf with properly built
    % internal navigation ('pdf bookmarks' for the table of contents,
    % internal cross-reference links, web links for URLs, etc.)
    \usepackage{hyperref}
    % The default LaTeX title has an obnoxious amount of whitespace. By default,
    % titling removes some of it. It also provides customization options.
    \usepackage{titling}
    \usepackage{longtable} % longtable support required by pandoc >1.10
    \usepackage{booktabs}  % table support for pandoc > 1.12.2
    \usepackage{array}     % table support for pandoc >= 2.11.3
    \usepackage{calc}      % table minipage width calculation for pandoc >= 2.11.1
    \usepackage[inline]{enumitem} % IRkernel/repr support (it uses the enumerate* environment)
    \usepackage[normalem]{ulem} % ulem is needed to support strikethroughs (\sout)
                                % normalem makes italics be italics, not underlines
    \usepackage{soul}      % strikethrough (\st) support for pandoc >= 3.0.0
    \usepackage{mathrsfs}
    

    
    
    % Colors for the hyperref package
    \definecolor{urlcolor}{rgb}{0,.145,.698}
    \definecolor{linkcolor}{rgb}{.71,0.21,0.01}
    \definecolor{citecolor}{rgb}{.12,.54,.11}

    % ANSI colors
    \definecolor{ansi-black}{HTML}{3E424D}
    \definecolor{ansi-black-intense}{HTML}{282C36}
    \definecolor{ansi-red}{HTML}{E75C58}
    \definecolor{ansi-red-intense}{HTML}{B22B31}
    \definecolor{ansi-green}{HTML}{00A250}
    \definecolor{ansi-green-intense}{HTML}{007427}
    \definecolor{ansi-yellow}{HTML}{DDB62B}
    \definecolor{ansi-yellow-intense}{HTML}{B27D12}
    \definecolor{ansi-blue}{HTML}{208FFB}
    \definecolor{ansi-blue-intense}{HTML}{0065CA}
    \definecolor{ansi-magenta}{HTML}{D160C4}
    \definecolor{ansi-magenta-intense}{HTML}{A03196}
    \definecolor{ansi-cyan}{HTML}{60C6C8}
    \definecolor{ansi-cyan-intense}{HTML}{258F8F}
    \definecolor{ansi-white}{HTML}{C5C1B4}
    \definecolor{ansi-white-intense}{HTML}{A1A6B2}
    \definecolor{ansi-default-inverse-fg}{HTML}{FFFFFF}
    \definecolor{ansi-default-inverse-bg}{HTML}{000000}

    % common color for the border for error outputs.
    \definecolor{outerrorbackground}{HTML}{FFDFDF}

    % commands and environments needed by pandoc snippets
    % extracted from the output of `pandoc -s`
    \providecommand{\tightlist}{%
      \setlength{\itemsep}{0pt}\setlength{\parskip}{0pt}}
    \DefineVerbatimEnvironment{Highlighting}{Verbatim}{commandchars=\\\{\}}
    % Add ',fontsize=\small' for more characters per line
    \newenvironment{Shaded}{}{}
    \newcommand{\KeywordTok}[1]{\textcolor[rgb]{0.00,0.44,0.13}{\textbf{{#1}}}}
    \newcommand{\DataTypeTok}[1]{\textcolor[rgb]{0.56,0.13,0.00}{{#1}}}
    \newcommand{\DecValTok}[1]{\textcolor[rgb]{0.25,0.63,0.44}{{#1}}}
    \newcommand{\BaseNTok}[1]{\textcolor[rgb]{0.25,0.63,0.44}{{#1}}}
    \newcommand{\FloatTok}[1]{\textcolor[rgb]{0.25,0.63,0.44}{{#1}}}
    \newcommand{\CharTok}[1]{\textcolor[rgb]{0.25,0.44,0.63}{{#1}}}
    \newcommand{\StringTok}[1]{\textcolor[rgb]{0.25,0.44,0.63}{{#1}}}
    \newcommand{\CommentTok}[1]{\textcolor[rgb]{0.38,0.63,0.69}{\textit{{#1}}}}
    \newcommand{\OtherTok}[1]{\textcolor[rgb]{0.00,0.44,0.13}{{#1}}}
    \newcommand{\AlertTok}[1]{\textcolor[rgb]{1.00,0.00,0.00}{\textbf{{#1}}}}
    \newcommand{\FunctionTok}[1]{\textcolor[rgb]{0.02,0.16,0.49}{{#1}}}
    \newcommand{\RegionMarkerTok}[1]{{#1}}
    \newcommand{\ErrorTok}[1]{\textcolor[rgb]{1.00,0.00,0.00}{\textbf{{#1}}}}
    \newcommand{\NormalTok}[1]{{#1}}

    % Additional commands for more recent versions of Pandoc
    \newcommand{\ConstantTok}[1]{\textcolor[rgb]{0.53,0.00,0.00}{{#1}}}
    \newcommand{\SpecialCharTok}[1]{\textcolor[rgb]{0.25,0.44,0.63}{{#1}}}
    \newcommand{\VerbatimStringTok}[1]{\textcolor[rgb]{0.25,0.44,0.63}{{#1}}}
    \newcommand{\SpecialStringTok}[1]{\textcolor[rgb]{0.73,0.40,0.53}{{#1}}}
    \newcommand{\ImportTok}[1]{{#1}}
    \newcommand{\DocumentationTok}[1]{\textcolor[rgb]{0.73,0.13,0.13}{\textit{{#1}}}}
    \newcommand{\AnnotationTok}[1]{\textcolor[rgb]{0.38,0.63,0.69}{\textbf{\textit{{#1}}}}}
    \newcommand{\CommentVarTok}[1]{\textcolor[rgb]{0.38,0.63,0.69}{\textbf{\textit{{#1}}}}}
    \newcommand{\VariableTok}[1]{\textcolor[rgb]{0.10,0.09,0.49}{{#1}}}
    \newcommand{\ControlFlowTok}[1]{\textcolor[rgb]{0.00,0.44,0.13}{\textbf{{#1}}}}
    \newcommand{\OperatorTok}[1]{\textcolor[rgb]{0.40,0.40,0.40}{{#1}}}
    \newcommand{\BuiltInTok}[1]{{#1}}
    \newcommand{\ExtensionTok}[1]{{#1}}
    \newcommand{\PreprocessorTok}[1]{\textcolor[rgb]{0.74,0.48,0.00}{{#1}}}
    \newcommand{\AttributeTok}[1]{\textcolor[rgb]{0.49,0.56,0.16}{{#1}}}
    \newcommand{\InformationTok}[1]{\textcolor[rgb]{0.38,0.63,0.69}{\textbf{\textit{{#1}}}}}
    \newcommand{\WarningTok}[1]{\textcolor[rgb]{0.38,0.63,0.69}{\textbf{\textit{{#1}}}}}


    % Define a nice break command that doesn't care if a line doesn't already
    % exist.
    \def\br{\hspace*{\fill} \\* }
    % Math Jax compatibility definitions
    \def\gt{>}
    \def\lt{<}
    \let\Oldtex\TeX
    \let\Oldlatex\LaTeX
    \renewcommand{\TeX}{\textrm{\Oldtex}}
    \renewcommand{\LaTeX}{\textrm{\Oldlatex}}
    % Document parameters
    % Document title
    \title{nb-lesson-2-2}
    
    
    
    
    
    
    

    % Pygments definitions
    
\makeatletter
\def\PY@reset{\let\PY@it=\relax \let\PY@bf=\relax%
    \let\PY@ul=\relax \let\PY@tc=\relax%
    \let\PY@bc=\relax \let\PY@ff=\relax}
\def\PY@tok#1{\csname PY@tok@#1\endcsname}
\def\PY@toks#1+{\ifx\relax#1\empty\else%
    \PY@tok{#1}\expandafter\PY@toks\fi}
\def\PY@do#1{\PY@bc{\PY@tc{\PY@ul{%
    \PY@it{\PY@bf{\PY@ff{#1}}}}}}}
\def\PY#1#2{\PY@reset\PY@toks#1+\relax+\PY@do{#2}}

\@namedef{PY@tok@w}{\def\PY@tc##1{\textcolor[rgb]{0.73,0.73,0.73}{##1}}}
\@namedef{PY@tok@c}{\let\PY@it=\textit\def\PY@tc##1{\textcolor[rgb]{0.24,0.48,0.48}{##1}}}
\@namedef{PY@tok@cp}{\def\PY@tc##1{\textcolor[rgb]{0.61,0.40,0.00}{##1}}}
\@namedef{PY@tok@k}{\let\PY@bf=\textbf\def\PY@tc##1{\textcolor[rgb]{0.00,0.50,0.00}{##1}}}
\@namedef{PY@tok@kp}{\def\PY@tc##1{\textcolor[rgb]{0.00,0.50,0.00}{##1}}}
\@namedef{PY@tok@kt}{\def\PY@tc##1{\textcolor[rgb]{0.69,0.00,0.25}{##1}}}
\@namedef{PY@tok@o}{\def\PY@tc##1{\textcolor[rgb]{0.40,0.40,0.40}{##1}}}
\@namedef{PY@tok@ow}{\let\PY@bf=\textbf\def\PY@tc##1{\textcolor[rgb]{0.67,0.13,1.00}{##1}}}
\@namedef{PY@tok@nb}{\def\PY@tc##1{\textcolor[rgb]{0.00,0.50,0.00}{##1}}}
\@namedef{PY@tok@nf}{\def\PY@tc##1{\textcolor[rgb]{0.00,0.00,1.00}{##1}}}
\@namedef{PY@tok@nc}{\let\PY@bf=\textbf\def\PY@tc##1{\textcolor[rgb]{0.00,0.00,1.00}{##1}}}
\@namedef{PY@tok@nn}{\let\PY@bf=\textbf\def\PY@tc##1{\textcolor[rgb]{0.00,0.00,1.00}{##1}}}
\@namedef{PY@tok@ne}{\let\PY@bf=\textbf\def\PY@tc##1{\textcolor[rgb]{0.80,0.25,0.22}{##1}}}
\@namedef{PY@tok@nv}{\def\PY@tc##1{\textcolor[rgb]{0.10,0.09,0.49}{##1}}}
\@namedef{PY@tok@no}{\def\PY@tc##1{\textcolor[rgb]{0.53,0.00,0.00}{##1}}}
\@namedef{PY@tok@nl}{\def\PY@tc##1{\textcolor[rgb]{0.46,0.46,0.00}{##1}}}
\@namedef{PY@tok@ni}{\let\PY@bf=\textbf\def\PY@tc##1{\textcolor[rgb]{0.44,0.44,0.44}{##1}}}
\@namedef{PY@tok@na}{\def\PY@tc##1{\textcolor[rgb]{0.41,0.47,0.13}{##1}}}
\@namedef{PY@tok@nt}{\let\PY@bf=\textbf\def\PY@tc##1{\textcolor[rgb]{0.00,0.50,0.00}{##1}}}
\@namedef{PY@tok@nd}{\def\PY@tc##1{\textcolor[rgb]{0.67,0.13,1.00}{##1}}}
\@namedef{PY@tok@s}{\def\PY@tc##1{\textcolor[rgb]{0.73,0.13,0.13}{##1}}}
\@namedef{PY@tok@sd}{\let\PY@it=\textit\def\PY@tc##1{\textcolor[rgb]{0.73,0.13,0.13}{##1}}}
\@namedef{PY@tok@si}{\let\PY@bf=\textbf\def\PY@tc##1{\textcolor[rgb]{0.64,0.35,0.47}{##1}}}
\@namedef{PY@tok@se}{\let\PY@bf=\textbf\def\PY@tc##1{\textcolor[rgb]{0.67,0.36,0.12}{##1}}}
\@namedef{PY@tok@sr}{\def\PY@tc##1{\textcolor[rgb]{0.64,0.35,0.47}{##1}}}
\@namedef{PY@tok@ss}{\def\PY@tc##1{\textcolor[rgb]{0.10,0.09,0.49}{##1}}}
\@namedef{PY@tok@sx}{\def\PY@tc##1{\textcolor[rgb]{0.00,0.50,0.00}{##1}}}
\@namedef{PY@tok@m}{\def\PY@tc##1{\textcolor[rgb]{0.40,0.40,0.40}{##1}}}
\@namedef{PY@tok@gh}{\let\PY@bf=\textbf\def\PY@tc##1{\textcolor[rgb]{0.00,0.00,0.50}{##1}}}
\@namedef{PY@tok@gu}{\let\PY@bf=\textbf\def\PY@tc##1{\textcolor[rgb]{0.50,0.00,0.50}{##1}}}
\@namedef{PY@tok@gd}{\def\PY@tc##1{\textcolor[rgb]{0.63,0.00,0.00}{##1}}}
\@namedef{PY@tok@gi}{\def\PY@tc##1{\textcolor[rgb]{0.00,0.52,0.00}{##1}}}
\@namedef{PY@tok@gr}{\def\PY@tc##1{\textcolor[rgb]{0.89,0.00,0.00}{##1}}}
\@namedef{PY@tok@ge}{\let\PY@it=\textit}
\@namedef{PY@tok@gs}{\let\PY@bf=\textbf}
\@namedef{PY@tok@gp}{\let\PY@bf=\textbf\def\PY@tc##1{\textcolor[rgb]{0.00,0.00,0.50}{##1}}}
\@namedef{PY@tok@go}{\def\PY@tc##1{\textcolor[rgb]{0.44,0.44,0.44}{##1}}}
\@namedef{PY@tok@gt}{\def\PY@tc##1{\textcolor[rgb]{0.00,0.27,0.87}{##1}}}
\@namedef{PY@tok@err}{\def\PY@bc##1{{\setlength{\fboxsep}{\string -\fboxrule}\fcolorbox[rgb]{1.00,0.00,0.00}{1,1,1}{\strut ##1}}}}
\@namedef{PY@tok@kc}{\let\PY@bf=\textbf\def\PY@tc##1{\textcolor[rgb]{0.00,0.50,0.00}{##1}}}
\@namedef{PY@tok@kd}{\let\PY@bf=\textbf\def\PY@tc##1{\textcolor[rgb]{0.00,0.50,0.00}{##1}}}
\@namedef{PY@tok@kn}{\let\PY@bf=\textbf\def\PY@tc##1{\textcolor[rgb]{0.00,0.50,0.00}{##1}}}
\@namedef{PY@tok@kr}{\let\PY@bf=\textbf\def\PY@tc##1{\textcolor[rgb]{0.00,0.50,0.00}{##1}}}
\@namedef{PY@tok@bp}{\def\PY@tc##1{\textcolor[rgb]{0.00,0.50,0.00}{##1}}}
\@namedef{PY@tok@fm}{\def\PY@tc##1{\textcolor[rgb]{0.00,0.00,1.00}{##1}}}
\@namedef{PY@tok@vc}{\def\PY@tc##1{\textcolor[rgb]{0.10,0.09,0.49}{##1}}}
\@namedef{PY@tok@vg}{\def\PY@tc##1{\textcolor[rgb]{0.10,0.09,0.49}{##1}}}
\@namedef{PY@tok@vi}{\def\PY@tc##1{\textcolor[rgb]{0.10,0.09,0.49}{##1}}}
\@namedef{PY@tok@vm}{\def\PY@tc##1{\textcolor[rgb]{0.10,0.09,0.49}{##1}}}
\@namedef{PY@tok@sa}{\def\PY@tc##1{\textcolor[rgb]{0.73,0.13,0.13}{##1}}}
\@namedef{PY@tok@sb}{\def\PY@tc##1{\textcolor[rgb]{0.73,0.13,0.13}{##1}}}
\@namedef{PY@tok@sc}{\def\PY@tc##1{\textcolor[rgb]{0.73,0.13,0.13}{##1}}}
\@namedef{PY@tok@dl}{\def\PY@tc##1{\textcolor[rgb]{0.73,0.13,0.13}{##1}}}
\@namedef{PY@tok@s2}{\def\PY@tc##1{\textcolor[rgb]{0.73,0.13,0.13}{##1}}}
\@namedef{PY@tok@sh}{\def\PY@tc##1{\textcolor[rgb]{0.73,0.13,0.13}{##1}}}
\@namedef{PY@tok@s1}{\def\PY@tc##1{\textcolor[rgb]{0.73,0.13,0.13}{##1}}}
\@namedef{PY@tok@mb}{\def\PY@tc##1{\textcolor[rgb]{0.40,0.40,0.40}{##1}}}
\@namedef{PY@tok@mf}{\def\PY@tc##1{\textcolor[rgb]{0.40,0.40,0.40}{##1}}}
\@namedef{PY@tok@mh}{\def\PY@tc##1{\textcolor[rgb]{0.40,0.40,0.40}{##1}}}
\@namedef{PY@tok@mi}{\def\PY@tc##1{\textcolor[rgb]{0.40,0.40,0.40}{##1}}}
\@namedef{PY@tok@il}{\def\PY@tc##1{\textcolor[rgb]{0.40,0.40,0.40}{##1}}}
\@namedef{PY@tok@mo}{\def\PY@tc##1{\textcolor[rgb]{0.40,0.40,0.40}{##1}}}
\@namedef{PY@tok@ch}{\let\PY@it=\textit\def\PY@tc##1{\textcolor[rgb]{0.24,0.48,0.48}{##1}}}
\@namedef{PY@tok@cm}{\let\PY@it=\textit\def\PY@tc##1{\textcolor[rgb]{0.24,0.48,0.48}{##1}}}
\@namedef{PY@tok@cpf}{\let\PY@it=\textit\def\PY@tc##1{\textcolor[rgb]{0.24,0.48,0.48}{##1}}}
\@namedef{PY@tok@c1}{\let\PY@it=\textit\def\PY@tc##1{\textcolor[rgb]{0.24,0.48,0.48}{##1}}}
\@namedef{PY@tok@cs}{\let\PY@it=\textit\def\PY@tc##1{\textcolor[rgb]{0.24,0.48,0.48}{##1}}}

\def\PYZbs{\char`\\}
\def\PYZus{\char`\_}
\def\PYZob{\char`\{}
\def\PYZcb{\char`\}}
\def\PYZca{\char`\^}
\def\PYZam{\char`\&}
\def\PYZlt{\char`\<}
\def\PYZgt{\char`\>}
\def\PYZsh{\char`\#}
\def\PYZpc{\char`\%}
\def\PYZdl{\char`\$}
\def\PYZhy{\char`\-}
\def\PYZsq{\char`\'}
\def\PYZdq{\char`\"}
\def\PYZti{\char`\~}
% for compatibility with earlier versions
\def\PYZat{@}
\def\PYZlb{[}
\def\PYZrb{]}
\makeatother


    % Exact colors from NB
    \definecolor{incolor}{rgb}{0.0, 0.0, 0.5}
    \definecolor{outcolor}{rgb}{0.545, 0.0, 0.0}



    
    % Prevent overflowing lines due to hard-to-break entities
    \sloppy
    % Setup hyperref package
    \hypersetup{
      breaklinks=true,  % so long urls are correctly broken across lines
      colorlinks=true,
      urlcolor=urlcolor,
      linkcolor=linkcolor,
      citecolor=citecolor,
      }
    % Slightly bigger margins than the latex defaults
    
    \geometry{verbose,tmargin=1in,bmargin=1in,lmargin=1in,rmargin=1in}
    
    

\begin{document}
    
    \maketitle
    
    

    
    Run in Google Colab

    \section{Data Preprocessing}\label{data-preprocessing}

    \subsection{Definitions}\label{definitions}

Raw data rarely comes in the form and shape that is necessary for the
optimal performance of a learning algorithm. On the other hand, the
success of a machine learning algorithm highly depends on the quality of
the data fed into the model. Real-world data is often dirty containing
outliers, missing values, wrong data types, irrelevant features, or
non-standardized data. The presence of any of these will prevent the
machine learning model to properly learn. For this reason, transforming
raw data into a useful format is an essential stage in the machine
learning process. Therefore, it is absolutely critical to ensure that we
examine and preprocess a dataset before we feed it to a learning
algorithm. In this section, we will discuss the essential data
preprocessing techniques that will help us to build good machine
learning models.

The topics that we will cover in this lesson are as follows:

\begin{itemize}
\tightlist
\item
  Removing and imputing missing values from the dataset
\item
  Getting categorical data into shape for machine learning algorithms
\item
  Selecting relevant features for the model construction
\item
  Feature Normalization
\end{itemize}

    \subsection{Philosophy of Use of
Scikit-Learn}\label{philosophy-of-use-of-scikit-learn}

Scikit-learn is one of the most widely used Python libraries for machine
learning. Its design philosophy is centered on \textbf{simplicity,
modularity, and consistency}, making it accessible to both beginners and
advanced users.

The \textbf{core principles} that guide the usage of scikit-learn are:
1. \textbf{Unified API}: Every model (regression, classification,
clustering, dimensionality reduction, etc.) follows the same pattern. 2.
\textbf{Minimal Configuration}: Most models work well with default
parameters and can be fine-tuned later. 3. \textbf{Consistency}: Whether
you are dealing with a linear regression, decision tree, or neural
network, the interaction with models remains the same. 4.
\textbf{Pipeline-Oriented}: Scikit-learn encourages a step-by-step
workflow involving data preprocessing, model training, and prediction.

\subsubsection{\texorpdfstring{The Core Methods: \texttt{fit()},
\texttt{transform()},
\texttt{predict()}}{The Core Methods: fit(), transform(), predict()}}\label{the-core-methods-fit-transform-predict}

Scikit-learn is built around a \textbf{three-step workflow}:
\textbf{fitting}, \textbf{transforming}, and \textbf{predicting}. Almost
every estimator (a model or transformer) in scikit-learn follows these
methods.

\textbf{1. \texttt{fit()} -- Learning from Data}

\begin{itemize}
\tightlist
\item
  This method is used to \textbf{train} a model or a transformer on the
  given dataset.
\item
  It extracts relevant patterns, parameters, or statistics from the
  data.
\item
  Used in \textbf{both preprocessing transformers (e.g., scalers, PCA)}
  and \textbf{models (e.g., linear regression, decision trees)}.
\end{itemize}

\textbf{Usage:}

\begin{Shaded}
\begin{Highlighting}[]
\NormalTok{model.fit(X\_train, y\_train)}
\end{Highlighting}
\end{Shaded}

or, for transformers:

\begin{Shaded}
\begin{Highlighting}[]
\NormalTok{scaler.fit(X\_train)}
\end{Highlighting}
\end{Shaded}

\textbf{Example: Linear Regression}

\begin{Shaded}
\begin{Highlighting}[]
\ImportTok{from}\NormalTok{ sklearn.linear\_model }\ImportTok{import}\NormalTok{ LinearRegression}

\NormalTok{X\_train }\OperatorTok{=}\NormalTok{ [[}\DecValTok{1}\NormalTok{], [}\DecValTok{2}\NormalTok{], [}\DecValTok{3}\NormalTok{], [}\DecValTok{4}\NormalTok{]]}
\NormalTok{y\_train }\OperatorTok{=}\NormalTok{ [}\DecValTok{2}\NormalTok{, }\DecValTok{4}\NormalTok{, }\DecValTok{6}\NormalTok{, }\DecValTok{8}\NormalTok{]}

\NormalTok{model }\OperatorTok{=}\NormalTok{ LinearRegression()}
\NormalTok{model.fit(X\_train, y\_train)  }\CommentTok{\# Learns the relationship (y = 2x)}
\end{Highlighting}
\end{Shaded}

\textbf{Example: Standard Scaler}

\begin{Shaded}
\begin{Highlighting}[]
\CommentTok{\# The scaler methods in scikit{-}learn are preprocessing techniques used to normalize or }
\CommentTok{\# standardize numerical data before feeding it into a machine learning model.}

\ImportTok{from}\NormalTok{ sklearn.preprocessing }\ImportTok{import}\NormalTok{ StandardScaler}

\NormalTok{X\_train }\OperatorTok{=}\NormalTok{ [[}\DecValTok{10}\NormalTok{], [}\DecValTok{20}\NormalTok{], [}\DecValTok{30}\NormalTok{], [}\DecValTok{40}\NormalTok{]]}

\NormalTok{scaler }\OperatorTok{=}\NormalTok{ StandardScaler()}
\NormalTok{scaler.fit(X\_train)  }\CommentTok{\# Computes mean and standard deviation}
\end{Highlighting}
\end{Shaded}

\textbf{2. \texttt{transform()} -- Applying a Transformation}

\begin{itemize}
\tightlist
\item
  Used \textbf{only by transformers} (not predictive models).
\item
  It applies a learned transformation to new data.
\item
  Example use cases:

  \begin{itemize}
  \tightlist
  \item
    \textbf{Feature scaling (e.g., StandardScaler, MinMaxScaler)}
  \item
    \textbf{Dimensionality reduction (e.g., PCA)}
  \item
    \textbf{Encoding categorical variables (e.g., OneHotEncoder)}
  \end{itemize}
\end{itemize}

\textbf{Usage:}

\begin{Shaded}
\begin{Highlighting}[]
\NormalTok{X\_transformed }\OperatorTok{=}\NormalTok{ transformer.transform(X\_new)}
\end{Highlighting}
\end{Shaded}

\textbf{Example: Standard Scaler}

\begin{Shaded}
\begin{Highlighting}[]
\NormalTok{X\_test }\OperatorTok{=}\NormalTok{ [[}\DecValTok{25}\NormalTok{], [}\DecValTok{35}\NormalTok{]]}

\NormalTok{X\_scaled }\OperatorTok{=}\NormalTok{ scaler.transform(X\_test)  }\CommentTok{\# Applies scaling learned from fit()}
\end{Highlighting}
\end{Shaded}

\begin{quote}
\textbf{Important:} \texttt{fit\_transform(X)} is a shortcut for
\texttt{fit(X)} followed by \texttt{transform(X)}.
\end{quote}

\begin{Shaded}
\begin{Highlighting}[]
\NormalTok{X\_scaled }\OperatorTok{=}\NormalTok{ scaler.fit\_transform(X\_train)  }\CommentTok{\# Often used in pipelines}
\end{Highlighting}
\end{Shaded}

\textbf{3. \texttt{predict()} -- Making Predictions}

\begin{itemize}
\tightlist
\item
  Used \textbf{only by predictive models} (not transformers).
\item
  Takes new input data (\texttt{X\_test}) and outputs predictions
  (\texttt{y\_pred}).
\item
  Works with both \textbf{classification (e.g., DecisionTreeClassifier,
  SVM)} and \textbf{regression (e.g., LinearRegression,
  RandomForestRegressor)}.
\end{itemize}

\textbf{Usage:}

\begin{Shaded}
\begin{Highlighting}[]
\NormalTok{y\_pred }\OperatorTok{=}\NormalTok{ model.predict(X\_test)}
\end{Highlighting}
\end{Shaded}

\textbf{Example: Predicting with Linear Regression}

\begin{Shaded}
\begin{Highlighting}[]
\NormalTok{X\_test }\OperatorTok{=}\NormalTok{ [[}\DecValTok{5}\NormalTok{], [}\DecValTok{6}\NormalTok{]]}

\NormalTok{y\_pred }\OperatorTok{=}\NormalTok{ model.predict(X\_test)  }\CommentTok{\# Output: [10, 12] (y = 2x)}
\end{Highlighting}
\end{Shaded}

\subsubsection{How These Methods Fit Together in a Typical
Pipeline}\label{how-these-methods-fit-together-in-a-typical-pipeline}

A typical \textbf{machine learning workflow} in scikit-learn follows
these steps:

\begin{enumerate}
\def\labelenumi{\arabic{enumi}.}
\tightlist
\item
  \textbf{Preprocess the data} (fit and transform):

  \begin{itemize}
  \tightlist
  \item
    Handle missing values, scale features, encode categorical variables.
  \item
    Example: \texttt{StandardScaler().fit\_transform(X)}
  \end{itemize}
\end{enumerate}

\begin{enumerate}
\def\labelenumi{\arabic{enumi}.}
\setcounter{enumi}{1}
\tightlist
\item
  \textbf{Train the model} (fit):

  \begin{itemize}
  \tightlist
  \item
    Example: \texttt{model.fit(X\_train,\ y\_train)}
  \end{itemize}
\end{enumerate}

\begin{enumerate}
\def\labelenumi{\arabic{enumi}.}
\setcounter{enumi}{2}
\tightlist
\item
  \textbf{Make predictions} (predict):

  \begin{itemize}
  \tightlist
  \item
    Example: \texttt{y\_pred\ =\ model.predict(X\_test)}
  \end{itemize}
\end{enumerate}

\textbf{Example: Full Pipeline}

\begin{Shaded}
\begin{Highlighting}[]
\ImportTok{from}\NormalTok{ sklearn.pipeline }\ImportTok{import}\NormalTok{ Pipeline}
\ImportTok{from}\NormalTok{ sklearn.preprocessing }\ImportTok{import}\NormalTok{ StandardScaler}
\ImportTok{from}\NormalTok{ sklearn.linear\_model }\ImportTok{import}\NormalTok{ LinearRegression}

\CommentTok{\# Define pipeline: Scaling + Linear Regression}
\NormalTok{pipeline }\OperatorTok{=}\NormalTok{ Pipeline([}
\NormalTok{    (}\StringTok{\textquotesingle{}scaler\textquotesingle{}}\NormalTok{, StandardScaler()),  }\CommentTok{\# First, scale features}
\NormalTok{    (}\StringTok{\textquotesingle{}regressor\textquotesingle{}}\NormalTok{, LinearRegression())  }\CommentTok{\# Then, fit regression model}
\NormalTok{])}

\CommentTok{\# Fit pipeline}
\NormalTok{pipeline.fit(X\_train, y\_train)}

\CommentTok{\# Predict}
\NormalTok{y\_pred }\OperatorTok{=}\NormalTok{ pipeline.predict(X\_test)}
\end{Highlighting}
\end{Shaded}

\textbf{Key Takeaways} - \texttt{fit()}: \textbf{Learns} from the data
(used by both transformers and models). - \texttt{transform()}:
\textbf{Applies} learned transformations (only for transformers). -
\texttt{predict()}: \textbf{Generates predictions} from trained models
(only for predictive models). - \textbf{Scikit-learn enforces a uniform
API}, making it easy to switch between models. - \textbf{Pipelines}
streamline the workflow by combining preprocessing and modeling in a
single object.

    \subsection{Data Cleaning}\label{data-cleaning}

    \subsubsection{Dealing with missing
data}\label{dealing-with-missing-data}

The real-world data often has a lot of missing values. The cause of
missing values can be data corruption or failure to record data. The
handling of missing data is very important during the preprocessing of
the dataset as many machine learning algorithms do not support missing
values.

Let's create a simple example data frame from a comma-separated values
(CSV) file to get a better grasp of the problem:

    \begin{Verbatim}[commandchars=\\\{\}]
{\color{incolor}In [{\color{incolor}12}]:} \PY{k+kn}{import} \PY{n+nn}{pandas} \PY{k}{as} \PY{n+nn}{pd}
         \PY{c+c1}{\PYZsh{}}
         \PY{c+c1}{\PYZsh{} The StringIO module is an in\PYZhy{}memory file\PYZhy{}like object. This object can be used as input or output }
         \PY{c+c1}{\PYZsh{} to the most function that would expect a standard file object. When the StringIO object is created }
         \PY{c+c1}{\PYZsh{} it is initialized by passing a string to the constructor. If no string is passed the StringIO will }
         \PY{c+c1}{\PYZsh{} start empty. In both cases, the initial cursor on the file starts at zero. NOTE: This module does }
         \PY{c+c1}{\PYZsh{} not exist in the latest version of Python so to work with this module we have to import it from }
         \PY{c+c1}{\PYZsh{} the io module.}
         \PY{c+c1}{\PYZsh{}}
         \PY{k+kn}{from} \PY{n+nn}{io} \PY{k+kn}{import} \PY{n}{StringIO}
         
         \PY{n}{csv\PYZus{}data} \PY{o}{=} \PYZbs{}
         \PY{+w}{    }\PY{l+s+sd}{\PYZsq{}\PYZsq{}\PYZsq{}}
         \PY{l+s+sd}{    A,B,C,D}
         \PY{l+s+sd}{    1.0,2.0,3.0,4.0}
         \PY{l+s+sd}{    5.0,6.0,,8.0}
         \PY{l+s+sd}{    10.0,11.0,12.0,}
         \PY{l+s+sd}{    10.0,11.0,12.0,13.0}
         \PY{l+s+sd}{    \PYZsq{}\PYZsq{}\PYZsq{}}
         \PY{n}{df} \PY{o}{=} \PY{n}{pd}\PY{o}{.}\PY{n}{read\PYZus{}csv}\PY{p}{(}\PY{n}{StringIO}\PY{p}{(}\PY{n}{csv\PYZus{}data}\PY{p}{)}\PY{p}{)}
         \PY{n}{df}
\end{Verbatim}
\begin{Verbatim}[commandchars=\\\{\}]
{\color{outcolor}Out[{\color{outcolor}12}]:}        A     B     C     D
         0    1.0   2.0   3.0   4.0
         1    5.0   6.0   NaN   8.0
         2   10.0  11.0  12.0   NaN
         3   10.0  11.0  12.0  13.0
\end{Verbatim}
    \subsubsection{Delete Rows with Missing
Values}\label{delete-rows-with-missing-values}

One of the easiest ways to deal with missing data is simply to remove
the corresponding features (columns) or training examples (rows) from
the dataset entirely. Missing values can be handled by deleting the rows
or columns having null values. If columns have more than half of the
rows as null then the entire column can be dropped. The rows which are
having one or more columns values as null can also be dropped.

Remember that, in pandas, rows with missing values can easily be dropped
via the \textbf{dropna} method:

    \begin{Verbatim}[commandchars=\\\{\}]
{\color{incolor}In [{\color{incolor}13}]:} \PY{n}{df1} \PY{o}{=} \PY{n}{df}\PY{o}{.}\PY{n}{dropna}\PY{p}{(}\PY{n}{axis}\PY{o}{=}\PY{l+m+mi}{0}\PY{p}{)}
         \PY{n}{df1}
\end{Verbatim}
\begin{Verbatim}[commandchars=\\\{\}]
{\color{outcolor}Out[{\color{outcolor}13}]:}        A     B     C     D
         0    1.0   2.0   3.0   4.0
         3   10.0  11.0  12.0  13.0
\end{Verbatim}
    \begin{Verbatim}[commandchars=\\\{\}]
{\color{incolor}In [{\color{incolor}14}]:} \PY{n}{df2} \PY{o}{=} \PY{n}{df}\PY{o}{.}\PY{n}{dropna}\PY{p}{(}\PY{n}{axis}\PY{o}{=}\PY{l+m+mi}{1}\PY{p}{)}
         \PY{n}{df2}
\end{Verbatim}
\begin{Verbatim}[commandchars=\\\{\}]
{\color{outcolor}Out[{\color{outcolor}14}]:}        A     B
         0    1.0   2.0
         1    5.0   6.0
         2   10.0  11.0
         3   10.0  11.0
\end{Verbatim}
    Although the removal of missing data seems to be a convenient approach,
it also comes with certain disadvantages; for example, we may end up
removing too many samples, which will make a reliable analysis
impossible. Or, if we remove too many feature columns, we will run the
risk of losing valuable information that our classifier needs to
discriminate between classes. In the next section, we will look at one
of the most commonly used alternatives for dealing with missing values:
interpolation techniques.

    \textbf{Pros}: - A model trained with the removal of all missing values
creates a robust model.

\textbf{Cons}: - Loss of a lot of information. - Works poorly if the
percentage of missing values is excessive in comparison to the complete
dataset.

    \subsubsection{Imputing missing values}\label{imputing-missing-values}

One of the most common interpolation techniques is called
\textbf{imputation}, where we simply replace the missing value with the
mean value of the entire feature column.

    \begin{quote}
\textbf{scikit-learn - SimpleImputer}

A convenient way to achieve this is by using the \textbf{SimpleImputer}
class from scikit-learn. Scikit-learn, infact, has built-in methods to
perform these preprocessing steps. For example, the
\texttt{SimpleImputer()} fills in missing values using a method of your
choice (see the code \textgreater below). The Scikit-learn documentation
lists the full options for data preprocessing
\href{https://scikit-learn.org/stable/modules/preprocessing.html}{here}.
\end{quote}

    \begin{Verbatim}[commandchars=\\\{\}]
{\color{incolor}In [{\color{incolor}15}]:} \PY{k+kn}{from} \PY{n+nn}{sklearn}\PY{n+nn}{.}\PY{n+nn}{impute} \PY{k+kn}{import} \PY{n}{SimpleImputer}
         \PY{k+kn}{import} \PY{n+nn}{numpy} \PY{k}{as} \PY{n+nn}{np}
         \PY{c+c1}{\PYZsh{}}
         \PY{c+c1}{\PYZsh{} define the imputing method}
         \PY{c+c1}{\PYZsh{}}
         \PY{n}{imr} \PY{o}{=} \PY{n}{SimpleImputer}\PY{p}{(}\PY{n}{missing\PYZus{}values}\PY{o}{=}\PY{n}{np}\PY{o}{.}\PY{n}{nan}\PY{p}{,} \PY{n}{strategy}\PY{o}{=}\PY{l+s+s1}{\PYZsq{}}\PY{l+s+s1}{mean}\PY{l+s+s1}{\PYZsq{}}\PY{p}{)}
         
         \PY{n}{imr} \PY{o}{=} \PY{n}{imr}\PY{o}{.}\PY{n}{fit}\PY{p}{(}\PY{n}{df}\PY{o}{.}\PY{n}{values}\PY{p}{)}
         \PY{n}{imputed\PYZus{}data} \PY{o}{=} \PY{n}{imr}\PY{o}{.}\PY{n}{transform}\PY{p}{(}\PY{n}{df}\PY{o}{.}\PY{n}{values}\PY{p}{)}
         
         \PY{n}{imputed\PYZus{}data}
\end{Verbatim}
\begin{Verbatim}[commandchars=\\\{\}]
{\color{outcolor}Out[{\color{outcolor}15}]:} array([[ 1.        ,  2.        ,  3.        ,  4.        ],
                [ 5.        ,  6.        ,  9.        ,  8.        ],
                [10.        , 11.        , 12.        ,  8.33333333],
                [10.        , 11.        , 12.        , 13.        ]])
\end{Verbatim}
    Alternatively, an even more convenient way to impute missing values is
by using pandas' \textbf{fillna} method and providing an imputation
method as an argument. For example, using pandas, we could achieve the
same mean imputation directly in the DataFrame object via the following
command:

    \begin{Verbatim}[commandchars=\\\{\}]
{\color{incolor}In [{\color{incolor}29}]:} \PY{n}{df}\PY{o}{.}\PY{n}{fillna}\PY{p}{(}\PY{n}{df}\PY{o}{.}\PY{n}{mean}\PY{p}{(}\PY{p}{)}\PY{p}{)}
\end{Verbatim}
\begin{Verbatim}[commandchars=\\\{\}]
{\color{outcolor}Out[{\color{outcolor}29}]:}       A     B     C     D     E     F     G
         0   1.0   2.0   3.0   4.0   5.0   6.0   7.0
         1   5.0   2.0   7.0   8.0   5.0   6.0  11.0
         2   9.0   6.0  11.0  12.0   9.0  10.0  15.0
         3  13.0   6.0  15.0  16.0   9.0  10.0  19.0
         4  17.0  10.0  19.0  20.0  13.0  14.0  23.0
         5  21.0  10.0  23.0  24.0  13.0  14.0  27.0
         6  25.0  14.0  27.0  28.0  17.0  18.0  31.0
         7  29.0  14.0  31.0  32.0  17.0  18.0  35.0
         8  33.0  18.0  35.0  36.0  21.0  22.0  39.0
         9  37.0  18.0  39.0  40.0  21.0  22.0  43.0
\end{Verbatim}
    \textbf{Pros}: - Prevent data loss which results in deletion of rows or
columns - Works well with a small dataset and is easy to implement.

\textbf{Cons}: - Works only with numerical continuous variables. - Can
cause data leakage - Do not factor the covariance between features.

    \subsubsection{Identify and Delete Zero-Variance
Predictors}\label{identify-and-delete-zero-variance-predictors}

Zero-variance predictors refer to input features that contain a single
value across the entire spectrum of observations. Accordingly, they do
not add any value to the prediction algorithm since the target variable
is not affected by the input value, making them redundant. Some ML
algorithms might also run into unexpected errors or output wrong
results. Pandas provides a function to count and list the number of
unique values in each column of a Pandas dataframe:

    \begin{Verbatim}[commandchars=\\\{\}]
{\color{incolor}In [{\color{incolor}30}]:} \PY{n}{csv\PYZus{}data} \PY{o}{=} \PYZbs{}
         \PY{n}{data\PYZus{}string} \PY{o}{=} \PY{l+s+s1}{\PYZsq{}\PYZsq{}\PYZsq{}}\PY{l+s+s1}{A,B,C,D,E,F,G,H}
         \PY{l+s+s1}{1.0,2.0,3.0,4.0,5.0,6.0,7.0,42.0}
         \PY{l+s+s1}{5.0,2.0,7.0,8.0,5.0,6.0,11.0,42.0}
         \PY{l+s+s1}{9.0,6.0,11.0,12.0,9.0,10.0,15.0,42.0}
         \PY{l+s+s1}{13.0,6.0,15.0,16.0,9.0,10.0,19.0,42.0}
         \PY{l+s+s1}{17.0,10.0,19.0,20.0,13.0,14.0,23.0,42.0}
         \PY{l+s+s1}{21.0,10.0,23.0,24.0,13.0,14.0,27.0,42.0}
         \PY{l+s+s1}{25.0,14.0,27.0,28.0,17.0,18.0,31.0,42.0}
         \PY{l+s+s1}{29.0,14.0,31.0,32.0,17.0,18.0,35.0,42.0}
         \PY{l+s+s1}{33.0,18.0,35.0,36.0,21.0,22.0,39.0,42.0}
         \PY{l+s+s1}{37.0,18.0,39.0,40.0,21.0,22.0,43.0,42.0}\PY{l+s+s1}{\PYZsq{}\PYZsq{}\PYZsq{}}
         
         \PY{n}{df} \PY{o}{=} \PY{n}{pd}\PY{o}{.}\PY{n}{read\PYZus{}csv}\PY{p}{(}\PY{n}{StringIO}\PY{p}{(}\PY{n}{csv\PYZus{}data}\PY{p}{)}\PY{p}{)}
         \PY{c+c1}{\PYZsh{} Get number of rows and columns}
         \PY{n}{num\PYZus{}rows}\PY{p}{,} \PY{n}{num\PYZus{}columns} \PY{o}{=} \PY{n}{df}\PY{o}{.}\PY{n}{shape}
         \PY{n+nb}{print}\PY{p}{(}\PY{n}{num\PYZus{}rows}\PY{p}{,} \PY{n}{num\PYZus{}columns}\PY{p}{)}
         \PY{n}{df}
\end{Verbatim}

    \begin{Verbatim}[commandchars=\\\{\}]
10 8

    \end{Verbatim}
\begin{Verbatim}[commandchars=\\\{\}]
{\color{outcolor}Out[{\color{outcolor}30}]:}       A     B     C     D     E     F     G     H
         0   1.0   2.0   3.0   4.0   5.0   6.0   7.0  42.0
         1   5.0   2.0   7.0   8.0   5.0   6.0  11.0  42.0
         2   9.0   6.0  11.0  12.0   9.0  10.0  15.0  42.0
         3  13.0   6.0  15.0  16.0   9.0  10.0  19.0  42.0
         4  17.0  10.0  19.0  20.0  13.0  14.0  23.0  42.0
         5  21.0  10.0  23.0  24.0  13.0  14.0  27.0  42.0
         6  25.0  14.0  27.0  28.0  17.0  18.0  31.0  42.0
         7  29.0  14.0  31.0  32.0  17.0  18.0  35.0  42.0
         8  33.0  18.0  35.0  36.0  21.0  22.0  39.0  42.0
         9  37.0  18.0  39.0  40.0  21.0  22.0  43.0  42.0
\end{Verbatim}
    \begin{Verbatim}[commandchars=\\\{\}]
{\color{incolor}In [{\color{incolor}31}]:} \PY{n}{df}\PY{o}{.}\PY{n}{nunique}\PY{p}{(}\PY{p}{)}
\end{Verbatim}
\begin{Verbatim}[commandchars=\\\{\}]
{\color{outcolor}Out[{\color{outcolor}31}]:} A    10
         B     5
         C    10
         D    10
         E     5
         F     5
         G    10
         H     1
         dtype: int64
\end{Verbatim}
    The code below will drop all columns that have a single value and update
the df dataframe.

    \begin{Verbatim}[commandchars=\\\{\}]
{\color{incolor}In [{\color{incolor}32}]:} \PY{n}{df2} \PY{o}{=} \PY{n}{df}\PY{o}{.}\PY{n}{drop}\PY{p}{(}\PY{n}{columns} \PY{o}{=} \PY{n}{df}\PY{o}{.}\PY{n}{columns}\PY{p}{[}\PY{n}{df}\PY{o}{.}\PY{n}{nunique}\PY{p}{(}\PY{p}{)} \PY{o}{==} \PY{l+m+mi}{1}\PY{p}{]}\PY{p}{,}\PY{n}{inplace} \PY{o}{=} \PY{k+kc}{False}\PY{p}{)}
         \PY{n+nb}{print}\PY{p}{(}\PY{n}{df2}\PY{p}{)}
         \PY{n+nb}{print}\PY{p}{(}\PY{n}{df}\PY{p}{)}
\end{Verbatim}

    \begin{Verbatim}[commandchars=\\\{\}]
      A     B     C     D     E     F     G
0   1.0   2.0   3.0   4.0   5.0   6.0   7.0
1   5.0   2.0   7.0   8.0   5.0   6.0  11.0
2   9.0   6.0  11.0  12.0   9.0  10.0  15.0
3  13.0   6.0  15.0  16.0   9.0  10.0  19.0
4  17.0  10.0  19.0  20.0  13.0  14.0  23.0
5  21.0  10.0  23.0  24.0  13.0  14.0  27.0
6  25.0  14.0  27.0  28.0  17.0  18.0  31.0
7  29.0  14.0  31.0  32.0  17.0  18.0  35.0
8  33.0  18.0  35.0  36.0  21.0  22.0  39.0
9  37.0  18.0  39.0  40.0  21.0  22.0  43.0
      A     B     C     D     E     F     G     H
0   1.0   2.0   3.0   4.0   5.0   6.0   7.0  42.0
1   5.0   2.0   7.0   8.0   5.0   6.0  11.0  42.0
2   9.0   6.0  11.0  12.0   9.0  10.0  15.0  42.0
3  13.0   6.0  15.0  16.0   9.0  10.0  19.0  42.0
4  17.0  10.0  19.0  20.0  13.0  14.0  23.0  42.0
5  21.0  10.0  23.0  24.0  13.0  14.0  27.0  42.0
6  25.0  14.0  27.0  28.0  17.0  18.0  31.0  42.0
7  29.0  14.0  31.0  32.0  17.0  18.0  35.0  42.0
8  33.0  18.0  35.0  36.0  21.0  22.0  39.0  42.0
9  37.0  18.0  39.0  40.0  21.0  22.0  43.0  42.0

    \end{Verbatim}

    \begin{Verbatim}[commandchars=\\\{\}]
{\color{incolor}In [{\color{incolor}33}]:} \PY{n}{df}\PY{o}{.}\PY{n}{drop}\PY{p}{(}\PY{n}{columns} \PY{o}{=} \PY{n}{df}\PY{o}{.}\PY{n}{columns}\PY{p}{[}\PY{n}{df}\PY{o}{.}\PY{n}{nunique}\PY{p}{(}\PY{p}{)} \PY{o}{==} \PY{l+m+mi}{1}\PY{p}{]}\PY{p}{,} \PY{n}{inplace} \PY{o}{=} \PY{k+kc}{True}\PY{p}{)}
         \PY{n}{df}
\end{Verbatim}
\begin{Verbatim}[commandchars=\\\{\}]
{\color{outcolor}Out[{\color{outcolor}33}]:}       A     B     C     D     E     F     G
         0   1.0   2.0   3.0   4.0   5.0   6.0   7.0
         1   5.0   2.0   7.0   8.0   5.0   6.0  11.0
         2   9.0   6.0  11.0  12.0   9.0  10.0  15.0
         3  13.0   6.0  15.0  16.0   9.0  10.0  19.0
         4  17.0  10.0  19.0  20.0  13.0  14.0  23.0
         5  21.0  10.0  23.0  24.0  13.0  14.0  27.0
         6  25.0  14.0  27.0  28.0  17.0  18.0  31.0
         7  29.0  14.0  31.0  32.0  17.0  18.0  35.0
         8  33.0  18.0  35.0  36.0  21.0  22.0  39.0
         9  37.0  18.0  39.0  40.0  21.0  22.0  43.0
\end{Verbatim}
    \begin{quote}
\textbf{pandas remind}: Here's a concise reminder of the key
\textbf{pandas} syntax properties used in the given instruction:

\begin{enumerate}
\def\labelenumi{\arabic{enumi}.}
\tightlist
\item
  \textbf{\texttt{df.nunique()}}\strut \\
\end{enumerate}

\begin{itemize}
\tightlist
\item
  Returns the number of unique values for each column in the DataFrame.
\end{itemize}

\begin{enumerate}
\def\labelenumi{\arabic{enumi}.}
\setcounter{enumi}{1}
\tightlist
\item
  \textbf{\texttt{df.columns{[}...{]}}}\strut \\
\end{enumerate}

\begin{itemize}
\tightlist
\item
  Retrieves the column labels of the DataFrame.
\item
  \texttt{df.columns{[}df.nunique()\ ==\ 1{]}} selects columns where all
  values are the same (i.e., with only one unique value).
\end{itemize}

\begin{enumerate}
\def\labelenumi{\arabic{enumi}.}
\setcounter{enumi}{2}
\tightlist
\item
  \textbf{\texttt{df.drop(columns=...)}}\strut \\
\end{enumerate}

\begin{itemize}
\tightlist
\item
  Drops the specified columns from the DataFrame.
\item
  \texttt{inplace=False} ensures that the original DataFrame remains
  unchanged, returning a new modified DataFrame (\texttt{df2} in this
  \textgreater case).
\end{itemize}
\end{quote}

    \subsection{Categorical Data}\label{categorical-data}

    Categorical data is a form of data that takes on values within a finite
set of discrete classes. It is difficult to count or measure categorical
data using numbers and therefore they are divided into categories:
\textbf{ordinal} and \textbf{nominal} features.

\textbf{Ordinal} features can be understood as categorical values that
\emph{can be sorted or ordered}. For example, t-shirt size would be an
ordinal feature, because we can define an order: XL \textgreater{} L
\textgreater{} M.

In contrast, \textbf{nominal} features don't imply any order and, to
continue with the previous example, we could think of t-shirt color as a
nominal feature since it typically doesn't make sense to say that, for
example, red is larger than blue.

    \subsubsection{Encoding}\label{encoding}

    Before we explore different techniques for handling such categorical
data, let's create a new DataFrame to illustrate the problem:

    \begin{Verbatim}[commandchars=\\\{\}]
{\color{incolor}In [{\color{incolor}53}]:} \PY{c+c1}{\PYZsh{} Define possible S\PYZam{}P ratings}
         \PY{n}{ratings} \PY{o}{=} \PY{p}{[}\PY{l+s+s2}{\PYZdq{}}\PY{l+s+s2}{AAA}\PY{l+s+s2}{\PYZdq{}}\PY{p}{,} \PY{l+s+s2}{\PYZdq{}}\PY{l+s+s2}{AA}\PY{l+s+s2}{\PYZdq{}}\PY{p}{,} \PY{l+s+s2}{\PYZdq{}}\PY{l+s+s2}{A}\PY{l+s+s2}{\PYZdq{}}\PY{p}{,} \PY{l+s+s2}{\PYZdq{}}\PY{l+s+s2}{BBB}\PY{l+s+s2}{\PYZdq{}}\PY{p}{,} \PY{l+s+s2}{\PYZdq{}}\PY{l+s+s2}{BB}\PY{l+s+s2}{\PYZdq{}}\PY{p}{,} \PY{l+s+s2}{\PYZdq{}}\PY{l+s+s2}{B}\PY{l+s+s2}{\PYZdq{}}\PY{p}{,} \PY{l+s+s2}{\PYZdq{}}\PY{l+s+s2}{CCC}\PY{l+s+s2}{\PYZdq{}}\PY{p}{,} \PY{l+s+s2}{\PYZdq{}}\PY{l+s+s2}{CC}\PY{l+s+s2}{\PYZdq{}}\PY{p}{,} \PY{l+s+s2}{\PYZdq{}}\PY{l+s+s2}{C}\PY{l+s+s2}{\PYZdq{}}\PY{p}{,} \PY{l+s+s2}{\PYZdq{}}\PY{l+s+s2}{D}\PY{l+s+s2}{\PYZdq{}}\PY{p}{]}
         
         \PY{n}{num\PYZus{}samples} \PY{o}{=} \PY{l+m+mi}{10}
         
         \PY{c+c1}{\PYZsh{} Generate an updated synthetic dataset}
         \PY{n}{df} \PY{o}{=} \PY{n}{pd}\PY{o}{.}\PY{n}{DataFrame}\PY{p}{(}\PY{p}{\PYZob{}}
             \PY{l+s+s2}{\PYZdq{}}\PY{l+s+s2}{rating}\PY{l+s+s2}{\PYZdq{}}\PY{p}{:} \PY{n}{np}\PY{o}{.}\PY{n}{random}\PY{o}{.}\PY{n}{choice}\PY{p}{(}\PY{n}{ratings}\PY{p}{,} \PY{n}{num\PYZus{}samples}\PY{p}{)}\PY{p}{,}                 \PY{c+c1}{\PYZsh{} Random S\PYZam{}P rating assignment}
             \PY{l+s+s2}{\PYZdq{}}\PY{l+s+s2}{income}\PY{l+s+s2}{\PYZdq{}}\PY{p}{:} \PY{n}{np}\PY{o}{.}\PY{n}{random}\PY{o}{.}\PY{n}{randint}\PY{p}{(}\PY{l+m+mi}{20000}\PY{p}{,} \PY{l+m+mi}{200000}\PY{p}{,} \PY{n}{num\PYZus{}samples}\PY{p}{)}\PY{p}{,}          \PY{c+c1}{\PYZsh{} Income in dollars}
             \PY{l+s+s2}{\PYZdq{}}\PY{l+s+s2}{age}\PY{l+s+s2}{\PYZdq{}}\PY{p}{:} \PY{n}{np}\PY{o}{.}\PY{n}{random}\PY{o}{.}\PY{n}{randint}\PY{p}{(}\PY{l+m+mi}{18}\PY{p}{,} \PY{l+m+mi}{75}\PY{p}{,} \PY{n}{num\PYZus{}samples}\PY{p}{)}\PY{p}{,}                    \PY{c+c1}{\PYZsh{} Age of the individual}
             \PY{l+s+s2}{\PYZdq{}}\PY{l+s+s2}{employment\PYZus{}status}\PY{l+s+s2}{\PYZdq{}}\PY{p}{:} \PY{n}{np}\PY{o}{.}\PY{n}{random}\PY{o}{.}\PY{n}{choice}\PY{p}{(}\PY{p}{[}\PY{l+s+s2}{\PYZdq{}}\PY{l+s+s2}{Employed}\PY{l+s+s2}{\PYZdq{}}\PY{p}{,} \PY{l+s+s2}{\PYZdq{}}\PY{l+s+s2}{Unemployed}\PY{l+s+s2}{\PYZdq{}}\PY{p}{,} \PY{l+s+s2}{\PYZdq{}}\PY{l+s+s2}{Self\PYZhy{}Employed}\PY{l+s+s2}{\PYZdq{}}\PY{p}{]}\PY{p}{,} \PY{n}{num\PYZus{}samples}\PY{p}{)}\PY{p}{,}
             \PY{l+s+s2}{\PYZdq{}}\PY{l+s+s2}{loan\PYZus{}amount}\PY{l+s+s2}{\PYZdq{}}\PY{p}{:} \PY{n}{np}\PY{o}{.}\PY{n}{random}\PY{o}{.}\PY{n}{randint}\PY{p}{(}\PY{l+m+mi}{5000}\PY{p}{,} \PY{l+m+mi}{500000}\PY{p}{,} \PY{n}{num\PYZus{}samples}\PY{p}{)}\PY{p}{,}      \PY{c+c1}{\PYZsh{} Loan amount in dollars}
             \PY{l+s+s2}{\PYZdq{}}\PY{l+s+s2}{default\PYZus{}history}\PY{l+s+s2}{\PYZdq{}}\PY{p}{:} \PY{n}{np}\PY{o}{.}\PY{n}{random}\PY{o}{.}\PY{n}{choice}\PY{p}{(}\PY{p}{[}\PY{l+s+s2}{\PYZdq{}}\PY{l+s+s2}{Yes}\PY{l+s+s2}{\PYZdq{}}\PY{p}{,} \PY{l+s+s2}{\PYZdq{}}\PY{l+s+s2}{No}\PY{l+s+s2}{\PYZdq{}}\PY{p}{]}\PY{p}{,} \PY{n}{num\PYZus{}samples}\PY{p}{,} \PY{n}{p}\PY{o}{=}\PY{p}{[}\PY{l+m+mf}{0.2}\PY{p}{,} \PY{l+m+mf}{0.8}\PY{p}{]}\PY{p}{)}  \PY{c+c1}{\PYZsh{} 20\PYZpc{} default history}
         \PY{p}{\PYZcb{}}\PY{p}{)}
         
         \PY{n}{df}
\end{Verbatim}
\begin{Verbatim}[commandchars=\\\{\}]
{\color{outcolor}Out[{\color{outcolor}53}]:}   rating  income  age employment\_status  loan\_amount default\_history
         0    AAA   41922   49          Employed       179441             Yes
         1    CCC  120260   61          Employed        56907             Yes
         2      A  125535   27        Unemployed       161385             Yes
         3      D   82743   24          Employed       320422              No
         4    AAA   90074   58        Unemployed       449747              No
         5      B  126547   39          Employed       290341              No
         6    CCC   58674   60     Self-Employed       210891              No
         7    BBB  178618   47          Employed       125324              No
         8    BBB   83793   25          Employed       200304             Yes
         9     CC   29897   54          Employed       299131              No
\end{Verbatim}
    \begin{quote}
\textbf{REMIND - FEATURES AND LABELS} *\textbf{ Remember that in machine
learning, you have }features\textbf{ and }labels\textbf{. \emph{The
features are the \textbf{descriptive} attributes}, and \emph{the label
is what you're attempting to predict or forecast}. In this simple
example, }rating\textbf{, }income\textbf{, }age\textbf{,
}employment\_status\textbf{ and }loan\_amount\textbf{ are
}features\textbf{ while }default\_history\textbf{ is the field that
contains the }label** of the corresponding record.
\end{quote}

    To make sure that the learning algorithm interprets the ordinal features
correctly, we need to convert the categorical string values into
integers.

    \begin{Verbatim}[commandchars=\\\{\}]
{\color{incolor}In [{\color{incolor}54}]:} \PY{k+kn}{from} \PY{n+nn}{sklearn}\PY{n+nn}{.}\PY{n+nn}{preprocessing} \PY{k+kn}{import} \PY{n}{LabelEncoder}
         
         \PY{n}{le} \PY{o}{=} \PY{n}{LabelEncoder}\PY{p}{(}\PY{p}{)}
         \PY{n}{df}\PY{p}{[}\PY{l+s+s2}{\PYZdq{}}\PY{l+s+s2}{rating\PYZus{}encoded}\PY{l+s+s2}{\PYZdq{}}\PY{p}{]} \PY{o}{=} \PY{n}{le}\PY{o}{.}\PY{n}{fit\PYZus{}transform}\PY{p}{(}\PY{n}{df}\PY{p}{[}\PY{l+s+s2}{\PYZdq{}}\PY{l+s+s2}{rating}\PY{l+s+s2}{\PYZdq{}}\PY{p}{]}\PY{p}{)}
         \PY{n}{df}
\end{Verbatim}
\begin{Verbatim}[commandchars=\\\{\}]
{\color{outcolor}Out[{\color{outcolor}54}]:}   rating  income  age employment\_status  loan\_amount default\_history  \textbackslash{}
         0    AAA   41922   49          Employed       179441             Yes   
         1    CCC  120260   61          Employed        56907             Yes   
         2      A  125535   27        Unemployed       161385             Yes   
         3      D   82743   24          Employed       320422              No   
         4    AAA   90074   58        Unemployed       449747              No   
         5      B  126547   39          Employed       290341              No   
         6    CCC   58674   60     Self-Employed       210891              No   
         7    BBB  178618   47          Employed       125324              No   
         8    BBB   83793   25          Employed       200304             Yes   
         9     CC   29897   54          Employed       299131              No   
         
            rating\_encoded  
         0               1  
         1               5  
         2               0  
         3               6  
         4               1  
         5               2  
         6               5  
         7               3  
         8               3  
         9               4  
\end{Verbatim}
    \begin{Verbatim}[commandchars=\\\{\}]
{\color{incolor}In [{\color{incolor}55}]:} \PY{n}{rating\PYZus{}map} \PY{o}{=} \PY{p}{\PYZob{}}
             \PY{l+s+s2}{\PYZdq{}}\PY{l+s+s2}{AAA}\PY{l+s+s2}{\PYZdq{}}\PY{p}{:} \PY{l+m+mi}{0}\PY{p}{,} \PY{l+s+s2}{\PYZdq{}}\PY{l+s+s2}{AA}\PY{l+s+s2}{\PYZdq{}}\PY{p}{:} \PY{l+m+mi}{1}\PY{p}{,} \PY{l+s+s2}{\PYZdq{}}\PY{l+s+s2}{A}\PY{l+s+s2}{\PYZdq{}}\PY{p}{:} \PY{l+m+mi}{2}\PY{p}{,} \PY{l+s+s2}{\PYZdq{}}\PY{l+s+s2}{BBB}\PY{l+s+s2}{\PYZdq{}}\PY{p}{:} \PY{l+m+mi}{3}\PY{p}{,} \PY{l+s+s2}{\PYZdq{}}\PY{l+s+s2}{BB}\PY{l+s+s2}{\PYZdq{}}\PY{p}{:} \PY{l+m+mi}{4}\PY{p}{,} 
             \PY{l+s+s2}{\PYZdq{}}\PY{l+s+s2}{B}\PY{l+s+s2}{\PYZdq{}}\PY{p}{:} \PY{l+m+mi}{5}\PY{p}{,} \PY{l+s+s2}{\PYZdq{}}\PY{l+s+s2}{CCC}\PY{l+s+s2}{\PYZdq{}}\PY{p}{:} \PY{l+m+mi}{6}\PY{p}{,} \PY{l+s+s2}{\PYZdq{}}\PY{l+s+s2}{CC}\PY{l+s+s2}{\PYZdq{}}\PY{p}{:} \PY{l+m+mi}{7}\PY{p}{,} \PY{l+s+s2}{\PYZdq{}}\PY{l+s+s2}{C}\PY{l+s+s2}{\PYZdq{}}\PY{p}{:} \PY{l+m+mi}{8}\PY{p}{,} \PY{l+s+s2}{\PYZdq{}}\PY{l+s+s2}{D}\PY{l+s+s2}{\PYZdq{}}\PY{p}{:} \PY{l+m+mi}{9}
         \PY{p}{\PYZcb{}}
         \PY{n}{df}\PY{p}{[}\PY{l+s+s2}{\PYZdq{}}\PY{l+s+s2}{rating\PYZus{}numeric}\PY{l+s+s2}{\PYZdq{}}\PY{p}{]} \PY{o}{=} \PY{n}{df}\PY{p}{[}\PY{l+s+s2}{\PYZdq{}}\PY{l+s+s2}{rating}\PY{l+s+s2}{\PYZdq{}}\PY{p}{]}\PY{o}{.}\PY{n}{map}\PY{p}{(}\PY{n}{rating\PYZus{}map}\PY{p}{)}
         \PY{n}{df}
\end{Verbatim}
\begin{Verbatim}[commandchars=\\\{\}]
{\color{outcolor}Out[{\color{outcolor}55}]:}   rating  income  age employment\_status  loan\_amount default\_history  \textbackslash{}
         0    AAA   41922   49          Employed       179441             Yes   
         1    CCC  120260   61          Employed        56907             Yes   
         2      A  125535   27        Unemployed       161385             Yes   
         3      D   82743   24          Employed       320422              No   
         4    AAA   90074   58        Unemployed       449747              No   
         5      B  126547   39          Employed       290341              No   
         6    CCC   58674   60     Self-Employed       210891              No   
         7    BBB  178618   47          Employed       125324              No   
         8    BBB   83793   25          Employed       200304             Yes   
         9     CC   29897   54          Employed       299131              No   
         
            rating\_encoded  rating\_numeric  
         0               1               0  
         1               5               6  
         2               0               2  
         3               6               9  
         4               1               0  
         5               2               5  
         6               5               6  
         7               3               3  
         8               3               3  
         9               4               7  
\end{Verbatim}
    \begin{quote}
\textbf{Preprocessing : sklearn.preprocessing}

Among some commonly used preprocessing tasks come
\texttt{OneHotEncoder}, \texttt{StandardScaler}, \texttt{MinMaxScaler},
etc. These are respectively for encoding of the categorical features
into a one-hot numeric array, standardization of the features and
scaling each feature to a given range. Many other preprocessing methods
are built-in this module. We can import this module as follows:
\end{quote}

    \begin{Verbatim}[commandchars=\\\{\}]
{\color{incolor}In [{\color{incolor}59}]:} \PY{n}{df2} \PY{o}{=} \PY{n}{df}\PY{o}{.}\PY{n}{drop}\PY{p}{(}\PY{n}{columns}\PY{o}{=}\PY{p}{[}\PY{l+s+s2}{\PYZdq{}}\PY{l+s+s2}{rating}\PY{l+s+s2}{\PYZdq{}}\PY{p}{,} \PY{l+s+s2}{\PYZdq{}}\PY{l+s+s2}{rating\PYZus{}encoded}\PY{l+s+s2}{\PYZdq{}}\PY{p}{]}\PY{p}{,} \PY{n}{inplace}\PY{o}{=}\PY{k+kc}{False}\PY{p}{)}
         \PY{n}{df2}\PY{o}{.}\PY{n}{rename}\PY{p}{(}\PY{n}{columns}\PY{o}{=}\PY{p}{\PYZob{}}\PY{l+s+s2}{\PYZdq{}}\PY{l+s+s2}{rating\PYZus{}numeric}\PY{l+s+s2}{\PYZdq{}}\PY{p}{:}\PY{l+s+s2}{\PYZdq{}}\PY{l+s+s2}{rating}\PY{l+s+s2}{\PYZdq{}}\PY{p}{\PYZcb{}}\PY{p}{,} \PY{n}{inplace}\PY{o}{=}\PY{k+kc}{True}\PY{p}{)}
         
         \PY{n}{cols} \PY{o}{=} \PY{p}{[}\PY{l+s+s2}{\PYZdq{}}\PY{l+s+s2}{rating}\PY{l+s+s2}{\PYZdq{}}\PY{p}{]} \PY{o}{+} \PY{p}{[}\PY{n}{col} \PY{k}{for} \PY{n}{col} \PY{o+ow}{in} \PY{n}{df2}\PY{o}{.}\PY{n}{columns} \PY{k}{if} \PY{n}{col} \PY{o}{!=} \PY{l+s+s2}{\PYZdq{}}\PY{l+s+s2}{rating}\PY{l+s+s2}{\PYZdq{}}\PY{p}{]}
         \PY{n}{df2} \PY{o}{=} \PY{n}{df2}\PY{p}{[}\PY{n}{cols}\PY{p}{]}
         \PY{n}{df2}
\end{Verbatim}
\begin{Verbatim}[commandchars=\\\{\}]
{\color{outcolor}Out[{\color{outcolor}59}]:}    rating  income  age employment\_status  loan\_amount default\_history
         0       0   41922   49          Employed       179441             Yes
         1       6  120260   61          Employed        56907             Yes
         2       2  125535   27        Unemployed       161385             Yes
         3       9   82743   24          Employed       320422              No
         4       0   90074   58        Unemployed       449747              No
         5       5  126547   39          Employed       290341              No
         6       6   58674   60     Self-Employed       210891              No
         7       3  178618   47          Employed       125324              No
         8       3   83793   25          Employed       200304             Yes
         9       7   29897   54          Employed       299131              No
\end{Verbatim}
    \subsubsection{Encoding Class Labels}\label{encoding-class-labels}

    Many machine learning libraries require that class labels are encoded as
integer values. Although most estimators for classification in
scikit-learn convert class labels to integers internally, it is
considered good practice to provide class labels as integer arrays to
avoid technical glitches. To encode the class labels, we can use an
approach similar to the mapping of ordinal features discussed
previously. We need to remember that class labels are not ordinal, and
it doesn't matter which integer number we assign to a particular string
label. Thus, we can simply enumerate the class labels, starting at 0:

    \begin{quote}
\textbf{REMIND - enumerate() method in Python} ***

\texttt{Enumerate()} method adds a counter to an iterable and returns it
in a form of enumerating object. This enumerated object can then be used
directly for loops or converted into a list of tuples using the
\texttt{list()} method.
\end{quote}

    \begin{Verbatim}[commandchars=\\\{\}]
{\color{incolor}In [{\color{incolor}65}]:} \PY{k+kn}{import} \PY{n+nn}{numpy} \PY{k}{as} \PY{n+nn}{np}
         
         
         \PY{n}{class\PYZus{}mapping} \PY{o}{=} \PY{p}{\PYZob{}}\PY{n}{label}\PY{p}{:} \PY{n}{idx} \PY{k}{for} \PY{n}{idx}\PY{p}{,} \PY{n}{label} \PY{o+ow}{in} \PY{n+nb}{enumerate}\PY{p}{(}\PY{n}{np}\PY{o}{.}\PY{n}{unique}\PY{p}{(}\PY{n}{df2}\PY{p}{[}\PY{l+s+s1}{\PYZsq{}}\PY{l+s+s1}{default\PYZus{}history}\PY{l+s+s1}{\PYZsq{}}\PY{p}{]}\PY{p}{)}\PY{p}{)}\PY{p}{\PYZcb{}}
         \PY{n}{class\PYZus{}mapping}
\end{Verbatim}
\begin{Verbatim}[commandchars=\\\{\}]
{\color{outcolor}Out[{\color{outcolor}65}]:} \{'No': 0, 'Yes': 1\}
\end{Verbatim}
    Next, we can use the mapping dictionary to transform the class labels
into integers:

    \begin{Verbatim}[commandchars=\\\{\}]
{\color{incolor}In [{\color{incolor}66}]:} \PY{n}{df2}\PY{p}{[}\PY{l+s+s1}{\PYZsq{}}\PY{l+s+s1}{default\PYZus{}history}\PY{l+s+s1}{\PYZsq{}}\PY{p}{]} \PY{o}{=} \PY{n}{df}\PY{p}{[}\PY{l+s+s1}{\PYZsq{}}\PY{l+s+s1}{default\PYZus{}history}\PY{l+s+s1}{\PYZsq{}}\PY{p}{]}\PY{o}{.}\PY{n}{map}\PY{p}{(}\PY{n}{class\PYZus{}mapping}\PY{p}{)}
         \PY{n}{df2}
\end{Verbatim}
\begin{Verbatim}[commandchars=\\\{\}]
{\color{outcolor}Out[{\color{outcolor}66}]:}    rating  income  age employment\_status  loan\_amount  default\_history
         0       0   41922   49          Employed       179441                1
         1       6  120260   61          Employed        56907                1
         2       2  125535   27        Unemployed       161385                1
         3       9   82743   24          Employed       320422                0
         4       0   90074   58        Unemployed       449747                0
         5       5  126547   39          Employed       290341                0
         6       6   58674   60     Self-Employed       210891                0
         7       3  178618   47          Employed       125324                0
         8       3   83793   25          Employed       200304                1
         9       7   29897   54          Employed       299131                0
\end{Verbatim}
    We can reverse the key-value pairs in the mapping dictionary as follows
to map the converted class labels back to the original string
representation:

    \begin{Verbatim}[commandchars=\\\{\}]
{\color{incolor}In [{\color{incolor}67}]:} \PY{n}{inv\PYZus{}class\PYZus{}mapping} \PY{o}{=} \PY{p}{\PYZob{}}\PY{n}{v}\PY{p}{:} \PY{n}{k} \PY{k}{for} \PY{n}{k}\PY{p}{,} \PY{n}{v} \PY{o+ow}{in} \PY{n}{class\PYZus{}mapping}\PY{o}{.}\PY{n}{items}\PY{p}{(}\PY{p}{)}\PY{p}{\PYZcb{}}
         \PY{n}{df2}\PY{p}{[}\PY{l+s+s1}{\PYZsq{}}\PY{l+s+s1}{default\PYZus{}history}\PY{l+s+s1}{\PYZsq{}}\PY{p}{]} \PY{o}{=} \PY{n}{df2}\PY{p}{[}\PY{l+s+s1}{\PYZsq{}}\PY{l+s+s1}{default\PYZus{}history}\PY{l+s+s1}{\PYZsq{}}\PY{p}{]}\PY{o}{.}\PY{n}{map}\PY{p}{(}\PY{n}{inv\PYZus{}class\PYZus{}mapping}\PY{p}{)}
         \PY{n}{df2}
\end{Verbatim}
\begin{Verbatim}[commandchars=\\\{\}]
{\color{outcolor}Out[{\color{outcolor}67}]:}    rating  income  age employment\_status  loan\_amount default\_history
         0       0   41922   49          Employed       179441             Yes
         1       6  120260   61          Employed        56907             Yes
         2       2  125535   27        Unemployed       161385             Yes
         3       9   82743   24          Employed       320422              No
         4       0   90074   58        Unemployed       449747              No
         5       5  126547   39          Employed       290341              No
         6       6   58674   60     Self-Employed       210891              No
         7       3  178618   47          Employed       125324              No
         8       3   83793   25          Employed       200304             Yes
         9       7   29897   54          Employed       299131              No
\end{Verbatim}
    Alternatively, there is a convenient LabelEncoder class directly
implemented in scikit-learn to achieve this:

    \begin{Verbatim}[commandchars=\\\{\}]
{\color{incolor}In [{\color{incolor}68}]:} \PY{k+kn}{from} \PY{n+nn}{sklearn}\PY{n+nn}{.}\PY{n+nn}{preprocessing} \PY{k+kn}{import} \PY{n}{LabelEncoder}
         
         \PY{n}{class\PYZus{}le} \PY{o}{=} \PY{n}{LabelEncoder}\PY{p}{(}\PY{p}{)}
         \PY{n}{y} \PY{o}{=} \PY{n}{class\PYZus{}le}\PY{o}{.}\PY{n}{fit\PYZus{}transform}\PY{p}{(}\PY{n}{df2}\PY{p}{[}\PY{l+s+s1}{\PYZsq{}}\PY{l+s+s1}{default\PYZus{}history}\PY{l+s+s1}{\PYZsq{}}\PY{p}{]}\PY{o}{.}\PY{n}{values}\PY{p}{)}
         \PY{n}{y}
\end{Verbatim}
\begin{Verbatim}[commandchars=\\\{\}]
{\color{outcolor}Out[{\color{outcolor}68}]:} array([1, 1, 1, 0, 0, 0, 0, 0, 1, 0])
\end{Verbatim}
    \subsubsection{One-hot Encoding}\label{one-hot-encoding}

    When there is no a natural order we have to resort to a different
approach that is to use the technique called \textbf{one-hot encoding}.
The idea behind this approach is to create a new dummy feature for each
unique value in the nominal feature column. Here, we would convert the
\texttt{employment\_status} feature into three new features:
\emph{employed}, \emph{self\_employed}, and \emph{unemployed}. Binary
values can then be used to indicate the particular employment status of
an example; for example, an employed customer can be encoded as
\emph{employed=1, self\_employed=0, unemployed=0}. To perform this
transformation, we can use the \texttt{OneHotEncoder} that is
implemented in \texttt{scikit-learn}'s preprocessing module:

    \begin{Verbatim}[commandchars=\\\{\}]
{\color{incolor}In [{\color{incolor}69}]:} \PY{k+kn}{from} \PY{n+nn}{sklearn}\PY{n+nn}{.}\PY{n+nn}{preprocessing} \PY{k+kn}{import} \PY{n}{OneHotEncoder}
         
         \PY{n}{X} \PY{o}{=} \PY{n}{df2}\PY{p}{[}\PY{p}{[}\PY{l+s+s1}{\PYZsq{}}\PY{l+s+s1}{employment\PYZus{}status}\PY{l+s+s1}{\PYZsq{}}\PY{p}{]}\PY{p}{]}\PY{o}{.}\PY{n}{values}
         \PY{n}{color\PYZus{}ohe} \PY{o}{=} \PY{n}{OneHotEncoder}\PY{p}{(}\PY{p}{)}
         \PY{n}{color\PYZus{}ohe}\PY{o}{.}\PY{n}{fit\PYZus{}transform}\PY{p}{(}\PY{n}{X}\PY{p}{[}\PY{p}{:}\PY{p}{,} \PY{l+m+mi}{0}\PY{p}{]}\PY{o}{.}\PY{n}{reshape}\PY{p}{(}\PY{o}{\PYZhy{}}\PY{l+m+mi}{1}\PY{p}{,} \PY{l+m+mi}{1}\PY{p}{)}\PY{p}{)}\PY{o}{.}\PY{n}{toarray}\PY{p}{(}\PY{p}{)}
\end{Verbatim}
\begin{Verbatim}[commandchars=\\\{\}]
{\color{outcolor}Out[{\color{outcolor}69}]:} array([[1., 0., 0.],
                [1., 0., 0.],
                [0., 0., 1.],
                [1., 0., 0.],
                [0., 0., 1.],
                [1., 0., 0.],
                [0., 1., 0.],
                [1., 0., 0.],
                [1., 0., 0.],
                [1., 0., 0.]])
\end{Verbatim}
    Another way to create those dummy features via one-hot encoding is to
use the get\_dummies method implemented in pandas. Applied to a
DataFrame, the get\_dummies method will only convert string columns and
leave all other columns unchanged:

    \begin{Verbatim}[commandchars=\\\{\}]
{\color{incolor}In [{\color{incolor}70}]:} \PY{n}{pd}\PY{o}{.}\PY{n}{get\PYZus{}dummies}\PY{p}{(}\PY{n}{df2}\PY{p}{[}\PY{p}{[}\PY{l+s+s1}{\PYZsq{}}\PY{l+s+s1}{employment\PYZus{}status}\PY{l+s+s1}{\PYZsq{}}\PY{p}{]}\PY{p}{]}\PY{p}{)}
\end{Verbatim}
\begin{Verbatim}[commandchars=\\\{\}]
{\color{outcolor}Out[{\color{outcolor}70}]:}    employment\_status\_Employed  employment\_status\_Self-Employed  \textbackslash{}
         0                        True                            False   
         1                        True                            False   
         2                       False                            False   
         3                        True                            False   
         4                       False                            False   
         5                        True                            False   
         6                       False                             True   
         7                        True                            False   
         8                        True                            False   
         9                        True                            False   
         
            employment\_status\_Unemployed  
         0                         False  
         1                         False  
         2                          True  
         3                         False  
         4                          True  
         5                         False  
         6                         False  
         7                         False  
         8                         False  
         9                         False  
\end{Verbatim}
    \subsection{Exercises}\label{exercises}

    \subsubsection{Chocolate Bar Ratings}\label{chocolate-bar-ratings}

\textbf{Context}

Chocolate is one of the most popular candies in the world. Each year,
residents of the United States collectively eat more than 2.8 billions
pounds. However, not all chocolate bars are created equal! This dataset
contains expert ratings of over 1,700 individual chocolate bars, along
with information on their regional origin, percentage of cocoa, the
variety of chocolate bean used and where the beans were grown.

\textbf{Flavors of Cacao Rating System}:

5= Elite (Transcending beyond the ordinary limits) 4= Premium (Superior
flavor development, character and style) 3= Satisfactory(3.0) to
praiseworthy(3.75) (well made with special qualities) 2= Disappointing
(Passable but contains at least one significant flaw) 1= Unpleasant
(mostly unpalatable)

\textbf{Link}

https://www.kaggle.com/rtatman/chocolate-bar-ratings

    \textbf{Problem:}

Download the \texttt{csv} file from the kaggle web page above and
perform a simple visualization

\textbf{Answer:}

    First of all you need to import pandas library, then define a variable
path (the folder in which you saved the csv file) and finally load the
file using the method read\_csv of pandas. Use the method head() to have
a look to the first lines:

\begin{Shaded}
\begin{Highlighting}[]
\ImportTok{import}\NormalTok{ pandas }\ImportTok{as}\NormalTok{ pd}
    
\NormalTok{path }\OperatorTok{=} \StringTok{\textquotesingle{}./data\textquotesingle{}}
\NormalTok{df }\OperatorTok{=}\NormalTok{ pd.read\_csv(path }\OperatorTok{+} \StringTok{"/flavors\_of\_cacao.csv"}\NormalTok{)}
\NormalTok{df.head()}
\end{Highlighting}
\end{Shaded}

    \begin{Verbatim}[commandchars=\\\{\}]
{\color{incolor}In [{\color{incolor}20}]:} \PY{c+c1}{\PYZsh{} put here your code}
\end{Verbatim}

    \textbf{Problem:}

Change columns names into:

\begin{itemize}
\tightlist
\item
  ``Company''
\item
  ``Spec\_Bean\_Origin\_or\_Bar\_Name''
\item
  ``Review\_Date''
\item
  ``Cocoa\_Percent''
\item
  ``Company\_Location''
\item
  ``Bean\_Type''``Broad\_Bean\_Origin''
\end{itemize}

\textbf{Answer:}

    A possible solution is to use a dictionary. Please note that sometimes
in pandas you can find strange characters, in particular the `\xa0'
character that you have to remove as in this example. This seems to be a
common problem in pandas dataframes, see for example this link
https://stackoverflow.com/questions/55442727/remove-unicode-xa0-from-pandas-column

\begin{Shaded}
\begin{Highlighting}[]
\NormalTok{df }\OperatorTok{=}\NormalTok{ df.rename(columns}\OperatorTok{=}\NormalTok{\{}\StringTok{"Company}\CharTok{\textbackslash{}xa0\textbackslash{}n}\StringTok{(Maker{-}if known)"}\NormalTok{: }\StringTok{"Company"}\NormalTok{,}
                        \StringTok{"Specific Bean Origin}\CharTok{\textbackslash{}n}\StringTok{or Bar Name"}\NormalTok{: }\StringTok{"Spec\_Bean\_Origin\_or\_Bar\_Name"}\NormalTok{,}
                        \StringTok{"Review}\CharTok{\textbackslash{}n}\StringTok{Date"}\NormalTok{: }\StringTok{"Review\_Date"}\NormalTok{,}
                        \StringTok{"Cocoa}\CharTok{\textbackslash{}n}\StringTok{Percent"}\NormalTok{: }\StringTok{"Cocoa\_Percent"}\NormalTok{,}
                        \StringTok{"Company}\CharTok{\textbackslash{}n}\StringTok{Location"}\NormalTok{: }\StringTok{"Company\_Location"}\NormalTok{,}
                        \StringTok{"Bean}\CharTok{\textbackslash{}n}\StringTok{Type"}\NormalTok{: }\StringTok{"Bean\_Type"}\NormalTok{,}
                        \StringTok{"Broad Bean}\CharTok{\textbackslash{}n}\StringTok{Origin"}\NormalTok{: }\StringTok{"Broad\_Bean\_Origin"}
\NormalTok{                       \})}
\end{Highlighting}
\end{Shaded}

    \begin{Verbatim}[commandchars=\\\{\}]
{\color{incolor}In [{\color{incolor}21}]:} \PY{c+c1}{\PYZsh{} put here your code}
\end{Verbatim}

    \textbf{Problem:}

Use the pandas data frame function info() is used in order to quickly
check which data types are available and if data is missing. Do you note
something strange?

\textbf{Answer:}

    When looking at the missing values, only the features
Broad\_Bean\_Origin and Bean\_Type are containing one missing value out
of 1795 total samples. However, when looking at the data frame head, the
first five rows of feature Bean\_Type are empty and should be therefore
count as missing value.

Since we don't know exactly what is the content of the first entry
Bean\_Type, we can fetched it in order to check its value and to use
this for replacing these values with NaN.

\begin{Shaded}
\begin{Highlighting}[]
    
\NormalTok{    missing\_val\_indication\_bean\_type }\OperatorTok{=}\NormalTok{ df.Bean\_Type.values[}\DecValTok{0}\NormalTok{]}

    \KeywordTok{def}\NormalTok{ replace\_with\_nan(missing\_val\_indication, current\_val):}
    \ControlFlowTok{if}\NormalTok{ current\_val }\OperatorTok{==}\NormalTok{ missing\_val\_indication:}
        \ControlFlowTok{return}\NormalTok{ np.nan}
    \ControlFlowTok{else}\NormalTok{:}
        \ControlFlowTok{return}\NormalTok{ current\_val}

    \CommentTok{\# replace missing value of Bean\_Type with np.nan}
\NormalTok{    df[}\StringTok{"Bean\_Type"}\NormalTok{] }\OperatorTok{=}\NormalTok{ df[}\StringTok{"Bean\_Type"}\NormalTok{].}\BuiltInTok{apply}\NormalTok{(}\KeywordTok{lambda}\NormalTok{ x: }
\NormalTok{                                        replace\_with\_nan(missing\_val\_indication\_bean\_type, x))}
\end{Highlighting}
\end{Shaded}

    \textbf{Problem:}

Find all categorical features.

\textbf{Answer:}

    \begin{Shaded}
\begin{Highlighting}[]
    \CommentTok{\# get list of categorical features}
\NormalTok{    list\_categorical\_cols }\OperatorTok{=} \BuiltInTok{list}\NormalTok{(df.columns[df.dtypes }\OperatorTok{==}\NormalTok{ np.}\BuiltInTok{object}\NormalTok{])}
\end{Highlighting}
\end{Shaded}

    \begin{Verbatim}[commandchars=\\\{\}]
{\color{incolor}In [{\color{incolor}22}]:} \PY{c+c1}{\PYZsh{} put your code here}
\end{Verbatim}

    \textbf{Problem:}

Find all numerical features

\textbf{Answer:}

    \begin{Shaded}
\begin{Highlighting}[]
    \CommentTok{\# get list of numerical features}
\NormalTok{    list\_numerical\_cols }\OperatorTok{=} \BuiltInTok{list}\NormalTok{(df.columns[df.dtypes }\OperatorTok{!=}\NormalTok{ np.}\BuiltInTok{object}\NormalTok{])}
\end{Highlighting}
\end{Shaded}

    \begin{Verbatim}[commandchars=\\\{\}]
{\color{incolor}In [{\color{incolor}23}]:} \PY{c+c1}{\PYZsh{} put your code here}
\end{Verbatim}

    \subsubsection{Cleaning Data with
Pandas}\label{cleaning-data-with-pandas}

    \textbf{Problem:}

Try to produce the following dataframe

numbers

nums

colors

other\_column

0

\#23

23

green

0

1

\#24

24

red

1

2

\#18

18

yellow

0

3

\#14

14

orange

2

4

\#12

NaN

purple

1

5

\#10

XYZ

blue

0

6

\#35

35

pink

2

\textbf{Answer:}

    \begin{Shaded}
\begin{Highlighting}[]
\NormalTok{df }\OperatorTok{=}\NormalTok{ pd.DataFrame(\{}\StringTok{"numbers"}\NormalTok{: [}\StringTok{"\#23"}\NormalTok{, }\StringTok{"\#24"}\NormalTok{, }\StringTok{"\#18"}\NormalTok{, }\StringTok{"\#14"}\NormalTok{, }\StringTok{"\#12"}\NormalTok{, }\StringTok{"\#10"}\NormalTok{, }\StringTok{"\#35"}\NormalTok{],}
                   \StringTok{"nums"}\NormalTok{: [}\StringTok{"23"}\NormalTok{, }\StringTok{"24"}\NormalTok{, }\StringTok{"18"}\NormalTok{, }\StringTok{"14"}\NormalTok{, np.nan, }\StringTok{"XYZ"}\NormalTok{, }\StringTok{"35"}\NormalTok{],}
                   \StringTok{"colors"}\NormalTok{: [}\StringTok{"green"}\NormalTok{, }\StringTok{"red"}\NormalTok{, }\StringTok{"yellow"}\NormalTok{, }\StringTok{"orange"}\NormalTok{, }\StringTok{"purple"}\NormalTok{, }\StringTok{"blue"}\NormalTok{, }\StringTok{"pink"}\NormalTok{],}
                   \StringTok{"other\_column"}\NormalTok{: [}\DecValTok{0}\NormalTok{, }\DecValTok{1}\NormalTok{, }\DecValTok{0}\NormalTok{, }\DecValTok{2}\NormalTok{, }\DecValTok{1}\NormalTok{, }\DecValTok{0}\NormalTok{, }\DecValTok{2}\NormalTok{]\})}
\NormalTok{df}
\end{Highlighting}
\end{Shaded}

    \begin{Verbatim}[commandchars=\\\{\}]
{\color{incolor}In [{\color{incolor}24}]:} \PY{c+c1}{\PYZsh{} put your code here}
\end{Verbatim}

    \textbf{Problem:}

What would happen if we wanted to try and compute the mean of numbers?

\textbf{Answer:}

    \begin{Shaded}
\begin{Highlighting}[]
\NormalTok{df[}\StringTok{"numbers"}\NormalTok{].mean()}
\end{Highlighting}
\end{Shaded}

    \begin{Verbatim}[commandchars=\\\{\}]
{\color{incolor}In [{\color{incolor}25}]:} \PY{c+c1}{\PYZsh{} put your code here}
\end{Verbatim}

    \textbf{Problem:}

Is there anything wrong with the previous question? Why? How can you
solve the error?

\textbf{Answer:}

    You have first of all convert all the string like `\#32' into numbers.

    \subsection{Feature Normalization}\label{feature-normalization}

    Many machine learning algorithms require that the selected features are
on the same scale for optimal performance, this process is called
``Feature Normalization'' and is the subject of this paragraph.

Data Normalization is a common practice in machine learning which
consists of transforming numeric columns to a common scale. In machine
learning, some feature values differ from others multiple times. The
features with higher values will dominate the leaning process. However,
it does not mean those variables are more important to predict the
outcome of the model. Data normalization transforms multiscaled data to
the same scale. After normalization, all variables have a similar
influence on the model, improving the stability and performance of the
learning algorithm.

There are multiple normalization techniques in statistics. In this
notebook, we will cover the most important ones:

\begin{itemize}
\tightlist
\item
  The maximum absolute scaling
\item
  The min-max feature scaling
\item
  The z-score method
\end{itemize}

    \subsubsection{The maximum absolute
scaling}\label{the-maximum-absolute-scaling}

The maximum absolute scaling rescales each feature between -1 and 1 by
dividing every observation by its maximum absolute value.

\[
x_{new} = \frac{x_{old}}{\max \vert x_{old} \vert}
\]

    \subsubsection{The min-max feature
scaling}\label{the-min-max-feature-scaling}

The min-max approach (often called normalization) rescales the feature
to a fixed range of {[}0,1{]} by subtracting the minimum value of the
feature and then dividing by the range:

\[
x_{new} = \frac{x_{old}-x_{min}}{x_{max}-x_{min}}
\]

The min-max scaling procedure is implemented in scikit-learn and can be
used as follows:

    \begin{Verbatim}[commandchars=\\\{\}]
{\color{incolor}In [{\color{incolor}71}]:} \PY{c+c1}{\PYZsh{}}
         \PY{c+c1}{\PYZsh{} Here we have to load the file \PYZsq{}salary\PYZus{}vs\PYZus{}age\PYZus{}1.csv\PYZsq{}}
         \PY{c+c1}{\PYZsh{}}
         \PY{k}{if} \PY{l+s+s1}{\PYZsq{}}\PY{l+s+s1}{google.colab}\PY{l+s+s1}{\PYZsq{}} \PY{o+ow}{in} \PY{n+nb}{str}\PY{p}{(}\PY{n}{get\PYZus{}ipython}\PY{p}{(}\PY{p}{)}\PY{p}{)}\PY{p}{:}
             \PY{k+kn}{from} \PY{n+nn}{google}\PY{n+nn}{.}\PY{n+nn}{colab} \PY{k+kn}{import} \PY{n}{files}
             \PY{n}{uploaded} \PY{o}{=} \PY{n}{files}\PY{o}{.}\PY{n}{upload}\PY{p}{(}\PY{p}{)}
             \PY{n}{path} \PY{o}{=} \PY{l+s+s1}{\PYZsq{}}\PY{l+s+s1}{\PYZsq{}}
         \PY{k}{else}\PY{p}{:}
             \PY{n}{path} \PY{o}{=} \PY{l+s+s1}{\PYZsq{}}\PY{l+s+s1}{./data/}\PY{l+s+s1}{\PYZsq{}}
\end{Verbatim}

    \begin{Verbatim}[commandchars=\\\{\}]
{\color{incolor}In [{\color{incolor}77}]:} \PY{c+c1}{\PYZsh{} Load the Pandas libraries with alias \PYZsq{}pd\PYZsq{} }
         \PY{k+kn}{import} \PY{n+nn}{pandas} \PY{k}{as} \PY{n+nn}{pd} 
         \PY{c+c1}{\PYZsh{} Read data from file \PYZsq{}salary\PYZus{}vs\PYZus{}age\PYZus{}1.csv\PYZsq{} }
         \PY{c+c1}{\PYZsh{} (in the same directory that your python process is based)}
         \PY{c+c1}{\PYZsh{} Control delimiters, with read\PYZus{}table }
         \PY{n}{df1} \PY{o}{=} \PY{n}{pd}\PY{o}{.}\PY{n}{read\PYZus{}table}\PY{p}{(}\PY{n}{path} \PY{o}{+} \PY{l+s+s2}{\PYZdq{}}\PY{l+s+s2}{salary\PYZus{}vs\PYZus{}age\PYZus{}1.csv}\PY{l+s+s2}{\PYZdq{}}\PY{p}{,} \PY{n}{sep}\PY{o}{=}\PY{l+s+s2}{\PYZdq{}}\PY{l+s+s2}{;}\PY{l+s+s2}{\PYZdq{}}\PY{p}{)} 
         \PY{c+c1}{\PYZsh{} Preview the first 5 lines of the loaded data }
         \PY{n+nb}{print}\PY{p}{(}\PY{n}{df1}\PY{o}{.}\PY{n}{head}\PY{p}{(}\PY{p}{)}\PY{p}{)}
         
         \PY{n}{columns\PYZus{}titles} \PY{o}{=} \PY{p}{[}\PY{l+s+s2}{\PYZdq{}}\PY{l+s+s2}{Salary}\PY{l+s+s2}{\PYZdq{}}\PY{p}{,}\PY{l+s+s2}{\PYZdq{}}\PY{l+s+s2}{Age}\PY{l+s+s2}{\PYZdq{}}\PY{p}{]}
         \PY{n}{df2}\PY{o}{=}\PY{n}{df1}\PY{o}{.}\PY{n}{reindex}\PY{p}{(}\PY{n}{columns}\PY{o}{=}\PY{n}{columns\PYZus{}titles}\PY{p}{)}
         \PY{n}{df2}
         
         \PY{n}{df2}\PY{p}{[}\PY{l+s+s1}{\PYZsq{}}\PY{l+s+s1}{Salary}\PY{l+s+s1}{\PYZsq{}}\PY{p}{]} \PY{o}{=} \PY{n}{df2}\PY{p}{[}\PY{l+s+s1}{\PYZsq{}}\PY{l+s+s1}{Salary}\PY{l+s+s1}{\PYZsq{}}\PY{p}{]}\PY{o}{/}\PY{l+m+mi}{1000} 
         \PY{n}{df2}\PY{p}{[}\PY{l+s+s1}{\PYZsq{}}\PY{l+s+s1}{Age2}\PY{l+s+s1}{\PYZsq{}}\PY{p}{]}\PY{o}{=}\PY{n}{df2}\PY{p}{[}\PY{l+s+s1}{\PYZsq{}}\PY{l+s+s1}{Age}\PY{l+s+s1}{\PYZsq{}}\PY{p}{]}\PY{o}{*}\PY{o}{*}\PY{l+m+mi}{2}
         \PY{n}{df2}\PY{p}{[}\PY{l+s+s1}{\PYZsq{}}\PY{l+s+s1}{Age3}\PY{l+s+s1}{\PYZsq{}}\PY{p}{]}\PY{o}{=}\PY{n}{df2}\PY{p}{[}\PY{l+s+s1}{\PYZsq{}}\PY{l+s+s1}{Age}\PY{l+s+s1}{\PYZsq{}}\PY{p}{]}\PY{o}{*}\PY{o}{*}\PY{l+m+mi}{3}
         \PY{n}{df2}\PY{p}{[}\PY{l+s+s1}{\PYZsq{}}\PY{l+s+s1}{Age4}\PY{l+s+s1}{\PYZsq{}}\PY{p}{]}\PY{o}{=}\PY{n}{df2}\PY{p}{[}\PY{l+s+s1}{\PYZsq{}}\PY{l+s+s1}{Age}\PY{l+s+s1}{\PYZsq{}}\PY{p}{]}\PY{o}{*}\PY{o}{*}\PY{l+m+mi}{4}
         \PY{n}{df2}\PY{p}{[}\PY{l+s+s1}{\PYZsq{}}\PY{l+s+s1}{Age5}\PY{l+s+s1}{\PYZsq{}}\PY{p}{]}\PY{o}{=}\PY{n}{df2}\PY{p}{[}\PY{l+s+s1}{\PYZsq{}}\PY{l+s+s1}{Age}\PY{l+s+s1}{\PYZsq{}}\PY{p}{]}\PY{o}{*}\PY{o}{*}\PY{l+m+mi}{5}
         \PY{n}{df2}
\end{Verbatim}

    \begin{Verbatim}[commandchars=\\\{\}]
   Age  Salary
0   25  135000
1   27  105000
2   30  105000
3   35  220000
4   40  300000

    \end{Verbatim}
\begin{Verbatim}[commandchars=\\\{\}]
{\color{outcolor}Out[{\color{outcolor}77}]:}    Salary  Age  Age2    Age3      Age4        Age5
         0   135.0   25   625   15625    390625     9765625
         1   105.0   27   729   19683    531441    14348907
         2   105.0   30   900   27000    810000    24300000
         3   220.0   35  1225   42875   1500625    52521875
         4   300.0   40  1600   64000   2560000   102400000
         5   270.0   45  2025   91125   4100625   184528125
         6   265.0   50  2500  125000   6250000   312500000
         7   260.0   55  3025  166375   9150625   503284375
         8   240.0   60  3600  216000  12960000   777600000
         9   265.0   65  4225  274625  17850625  1160290625
\end{Verbatim}
    \begin{Verbatim}[commandchars=\\\{\}]
{\color{incolor}In [{\color{incolor}78}]:} \PY{k+kn}{from} \PY{n+nn}{sklearn}\PY{n+nn}{.}\PY{n+nn}{preprocessing} \PY{k+kn}{import} \PY{n}{MinMaxScaler}
         
         \PY{n}{mms} \PY{o}{=} \PY{n}{MinMaxScaler}\PY{p}{(}\PY{p}{)}
         \PY{n}{df3} \PY{o}{=} \PY{n}{pd}\PY{o}{.}\PY{n}{DataFrame}\PY{p}{(}\PY{n}{mms}\PY{o}{.}\PY{n}{fit\PYZus{}transform}\PY{p}{(}\PY{n}{df2}\PY{p}{)}\PY{p}{)}
         \PY{n}{df3}
\end{Verbatim}
\begin{Verbatim}[commandchars=\\\{\}]
{\color{outcolor}Out[{\color{outcolor}78}]:}           0      1         2         3         4         5
         0  0.153846  0.000  0.000000  0.000000  0.000000  0.000000
         1  0.000000  0.050  0.028889  0.015668  0.008065  0.003984
         2  0.000000  0.125  0.076389  0.043919  0.024019  0.012633
         3  0.589744  0.250  0.166667  0.105212  0.063574  0.037162
         4  1.000000  0.375  0.270833  0.186776  0.124248  0.080515
         5  0.846154  0.500  0.388889  0.291506  0.212486  0.151898
         6  0.820513  0.625  0.520833  0.422297  0.335588  0.263127
         7  0.794872  0.750  0.666667  0.582046  0.501718  0.428951
         8  0.692308  0.875  0.826389  0.773649  0.719895  0.667377
         9  0.820513  1.000  1.000000  1.000000  1.000000  1.000000
\end{Verbatim}
    \subsubsection{Z-Score}\label{z-score}

The \textbf{z-score} method (often called \textbf{standardization})
transforms the data into a distribution with a mean of 0 and a standard
deviation of 1. Each standardized value is computed by subtracting the
mean of the corresponding feature and then dividing by the standard
deviation.

\[
x_{new} = \frac{x_{old} - \mu}{\sigma}
\]

Unlike min-max scaling, the z-score does not rescale the feature to a
fixed range. The z-score typically ranges from -3.00 to 3.00 (more than
99\% of the data) if the input is normally distributed.

It is important to bear in mind that z-scores are not necessarily
normally distributed. They just scale the data and follow the same
distribution as the original input. This transformed distribution has a
mean of 0 and a standard deviation of 1 and is going to be the standard
normal distribution only if the input feature follows a normal
distribution.

Standardization can easily be achieved by using the built-in NumPy
methods mean and std:

    \begin{Verbatim}[commandchars=\\\{\}]
{\color{incolor}In [{\color{incolor}79}]:} \PY{k+kn}{import} \PY{n+nn}{numpy} \PY{k}{as} \PY{n+nn}{np}
         
         \PY{n}{X} \PY{o}{=} \PY{n}{np}\PY{o}{.}\PY{n}{array}\PY{p}{(}\PY{p}{[}\PY{l+m+mi}{6}\PY{p}{,} \PY{l+m+mi}{7}\PY{p}{,} \PY{l+m+mi}{7}\PY{p}{,} \PY{l+m+mi}{12}\PY{p}{,} \PY{l+m+mi}{13}\PY{p}{,} \PY{l+m+mi}{13}\PY{p}{,} \PY{l+m+mi}{15}\PY{p}{,} \PY{l+m+mi}{16}\PY{p}{,} \PY{l+m+mi}{19}\PY{p}{,} \PY{l+m+mi}{22}\PY{p}{]}\PY{p}{)}
         
         \PY{n}{X\PYZus{}std} \PY{o}{=} \PY{n}{np}\PY{o}{.}\PY{n}{copy}\PY{p}{(}\PY{n}{X}\PY{p}{)}
         \PY{n}{X\PYZus{}std} \PY{o}{=} \PY{p}{(}\PY{n}{X} \PY{o}{\PYZhy{}} \PY{n}{X}\PY{o}{.}\PY{n}{mean}\PY{p}{(}\PY{p}{)}\PY{p}{)} \PY{o}{/} \PY{n}{X}\PY{o}{.}\PY{n}{std}\PY{p}{(}\PY{p}{)}
         
         \PY{n+nb}{print}\PY{p}{(}\PY{n}{X\PYZus{}std}\PY{p}{)}
\end{Verbatim}

    \begin{Verbatim}[commandchars=\\\{\}]
[-1.39443338 -1.19522861 -1.19522861 -0.19920477  0.          0.
  0.39840954  0.5976143   1.19522861  1.79284291]

    \end{Verbatim}

    Or simply using the specific function of the stats module of scipy

    \begin{Verbatim}[commandchars=\\\{\}]
{\color{incolor}In [{\color{incolor}80}]:} \PY{k+kn}{import} \PY{n+nn}{scipy}\PY{n+nn}{.}\PY{n+nn}{stats} \PY{k}{as} \PY{n+nn}{stats}
         
         \PY{n}{stats}\PY{o}{.}\PY{n}{zscore}\PY{p}{(}\PY{n}{X}\PY{p}{)}
\end{Verbatim}
\begin{Verbatim}[commandchars=\\\{\}]
{\color{outcolor}Out[{\color{outcolor}80}]:} array([-1.39443338, -1.19522861, -1.19522861, -0.19920477,  0.        ,
                 0.        ,  0.39840954,  0.5976143 ,  1.19522861,  1.79284291])
\end{Verbatim}
    Standardization is very useful with gradient descent learning. In this
case the optimizer has to go through fewer steps to find a good or
optimal solution (the global cost minimum).

    Similar to the MinMaxScaler class, scikit-learn also implements a class
for standardization:

    \begin{Verbatim}[commandchars=\\\{\}]
{\color{incolor}In [{\color{incolor}81}]:} \PY{k+kn}{from} \PY{n+nn}{sklearn}\PY{n+nn}{.}\PY{n+nn}{preprocessing} \PY{k+kn}{import} \PY{n}{StandardScaler}
         
         \PY{n}{stdsc} \PY{o}{=} \PY{n}{StandardScaler}\PY{p}{(}\PY{p}{)}
         \PY{n}{df4} \PY{o}{=} \PY{n}{pd}\PY{o}{.}\PY{n}{DataFrame}\PY{p}{(}\PY{n}{stdsc}\PY{o}{.}\PY{n}{fit\PYZus{}transform}\PY{p}{(}\PY{n}{df2}\PY{p}{)}\PY{p}{)}
         \PY{n}{df4}
\end{Verbatim}
\begin{Verbatim}[commandchars=\\\{\}]
{\color{outcolor}Out[{\color{outcolor}81}]:}           0         1         2         3         4         5
         0 -1.170242 -1.359724 -1.189131 -1.041783 -0.920815 -0.824435
         1 -1.601005 -1.210304 -1.102065 -0.994071 -0.895974 -0.812022
         2 -1.601005 -0.986174 -0.958907 -0.908042 -0.846835 -0.785069
         3  0.050256 -0.612623 -0.686823 -0.721391 -0.725003 -0.708630
         4  1.198959 -0.239072 -0.372880 -0.473014 -0.538122 -0.573535
         5  0.768195  0.134478 -0.017078 -0.154092 -0.266345 -0.351091
         6  0.696401  0.508029  0.380582  0.244194  0.112820 -0.004480
         7  0.624608  0.881579  0.820102  0.730661  0.624511  0.512260
         8  0.337432  1.255130  1.301481  1.314127  1.296512  1.255243
         9  0.696401  1.628681  1.824719  2.003411  2.159252  2.291760
\end{Verbatim}
    \subsection{Feature Analysis}\label{feature-analysis}

    \subsubsection{Overfitting and Strategies for
Mitigation}\label{overfitting-and-strategies-for-mitigation}

When evaluating the performance of a machine learning model, a
significant discrepancy between its performance on the training dataset
and the test dataset is a strong indication of \textbf{overfitting}. As
discussed in Chapter X, overfitting occurs when a model captures the
noise or idiosyncrasies of the training data rather than the underlying
patterns. In other words, the model becomes overly tailored to the
specific observations in the training dataset, thereby failing to
generalize effectively to new, unseen data. This phenomenon is often
described as the model exhibiting \textbf{high variance}.

The root cause of overfitting can frequently be traced to the model's
complexity. A model that is too intricate for the given training data is
likely to overfit, as it has a higher capacity to memorize the data
rather than learn generalized patterns. Addressing overfitting is
essential to ensure that a model performs well not only on the training
dataset but also on new data in real-world scenarios.

\paragraph{Common Strategies to Reduce
Overfitting}\label{common-strategies-to-reduce-overfitting}

Several techniques can be employed to mitigate overfitting and improve a
model's generalization to unseen data:

\begin{enumerate}
\def\labelenumi{\arabic{enumi}.}
\tightlist
\item
  \textbf{Collecting More Training Data}:

  \begin{itemize}
  \tightlist
  \item
    Increasing the size of the training dataset allows the model to
    better discern underlying patterns and reduce the impact of noise.
    However, this approach is often impractical due to constraints in
    time, cost, or data availability.
  \end{itemize}
\end{enumerate}

\begin{enumerate}
\def\labelenumi{\arabic{enumi}.}
\setcounter{enumi}{1}
\tightlist
\item
  \textbf{Introducing Regularization}:

  \begin{itemize}
  \tightlist
  \item
    Regularization involves incorporating a penalty term into the
    model's objective function to discourage complexity. By doing so,
    the model is incentivized to use simpler solutions, which are less
    likely to overfit the training data.
  \end{itemize}
\end{enumerate}

\begin{enumerate}
\def\labelenumi{\arabic{enumi}.}
\setcounter{enumi}{2}
\tightlist
\item
  \textbf{Choosing a Simpler Model}:

  \begin{itemize}
  \tightlist
  \item
    Opting for a model with fewer parameters or a less complex
    architecture can reduce the risk of overfitting. Simplifying the
    model limits its ability to capture noise in the training data,
    thereby improving generalization.
  \end{itemize}
\end{enumerate}

\begin{enumerate}
\def\labelenumi{\arabic{enumi}.}
\setcounter{enumi}{3}
\tightlist
\item
  \textbf{Reducing Dimensionality}:

  \begin{itemize}
  \tightlist
  \item
    High-dimensional data can exacerbate overfitting due to the curse of
    dimensionality. Techniques such as \textbf{feature selection} or
    \textbf{feature extraction} can reduce the number of features,
    simplifying the model and requiring fewer parameters to be fitted.
  \end{itemize}
\end{enumerate}

    \subsubsection{Feature Selection and Feature
Extraction}\label{feature-selection-and-feature-extraction}

Feature selection and feature extraction are two distinct approaches to
reducing the dimensionality of a dataset. Both techniques aim to
simplify a machine learning model, improve its performance, and reduce
the risk of overfitting by addressing the challenges associated with
high-dimensional data. While these methods share the common goal of
enhancing a model's generalization capabilities, their methodologies
differ fundamentally.

\paragraph{Feature Selection}\label{feature-selection}

\textbf{Feature selection} involves identifying and retaining a subset
of the original features (variables) that are most relevant to the
target variable. Rather than transforming the data, feature selection
works within the existing feature space, discarding irrelevant or
redundant features. This approach helps reduce noise in the data while
maintaining the interpretability of the original features.

\textbf{Types of Feature Selection Methods}

\begin{enumerate}
\def\labelenumi{\arabic{enumi}.}
\tightlist
\item
  \textbf{Filter Methods}:

  \begin{itemize}
  \tightlist
  \item
    Use statistical measures to assess the relationship between each
    feature and the target variable.
  \item
    Examples:

    \begin{itemize}
    \tightlist
    \item
      Pearson correlation for continuous variables.
    \item
      Chi-squared tests for categorical variables.
    \item
      Mutual information for non-linear relationships.
    \end{itemize}
  \item
    Filter methods are computationally efficient and independent of the
    machine learning model.
  \end{itemize}
\end{enumerate}

\begin{enumerate}
\def\labelenumi{\arabic{enumi}.}
\setcounter{enumi}{1}
\tightlist
\item
  \textbf{Wrapper Methods}:

  \begin{itemize}
  \tightlist
  \item
    Use a predictive model to evaluate the utility of feature subsets.
  \item
    Examples:

    \begin{itemize}
    \tightlist
    \item
      Recursive Feature Elimination (RFE): Iteratively removes the least
      important features based on model performance.
    \item
      Forward Selection: Starts with no features and adds them one by
      one.
    \item
      Backward Elimination: Starts with all features and removes them
      one by one.
    \end{itemize}
  \item
    Wrapper methods are computationally intensive but often yield better
    results since they are tailored to the specific model.
  \end{itemize}
\end{enumerate}

\begin{enumerate}
\def\labelenumi{\arabic{enumi}.}
\setcounter{enumi}{2}
\tightlist
\item
  \textbf{Embedded Methods}:

  \begin{itemize}
  \tightlist
  \item
    Perform feature selection as part of the model training process.
  \item
    Examples:

    \begin{itemize}
    \tightlist
    \item
      Regularization techniques like Lasso (L1) and Ridge (L2)
      regression automatically shrink the coefficients of irrelevant
      features to zero or near zero.
    \item
      Feature importance scores in decision trees and ensemble methods
      (e.g., Random Forest, XGBoost).
    \end{itemize}
  \item
    Embedded methods balance computational efficiency with
    model-specific relevance.
  \end{itemize}
\end{enumerate}

\textbf{Benefits of Feature Selection}

\begin{itemize}
\tightlist
\item
  \textbf{Model Simplification}: Reduces the number of features, making
  the model easier to understand and interpret.
\item
  \textbf{Reduced Overfitting}: Eliminates irrelevant features that may
  introduce noise.
\item
  \textbf{Improved Efficiency}: Reduces computational cost during
  training and inference.
\end{itemize}

    \paragraph{Feature Extraction}\label{feature-extraction}

\textbf{Feature extraction} involves creating new features by
transforming or combining the existing features into a lower-dimensional
space. Unlike feature selection, which works within the original feature
space, feature extraction generates entirely new features. These new
features are often linear or non-linear combinations of the original
features, designed to capture the most informative aspects of the data.

\textbf{Common Feature Extraction Techniques}

\begin{enumerate}
\def\labelenumi{\arabic{enumi}.}
\tightlist
\item
  \textbf{Principal Component Analysis (PCA)}:

  \begin{itemize}
  \tightlist
  \item
    A linear technique that transforms the data into orthogonal
    components, ranked by the amount of variance they explain.
  \item
    The first few principal components capture the majority of the
    variance in the data, enabling dimensionality reduction.
  \end{itemize}
\end{enumerate}

\begin{enumerate}
\def\labelenumi{\arabic{enumi}.}
\setcounter{enumi}{1}
\tightlist
\item
  \textbf{Linear Discriminant Analysis (LDA)}:

  \begin{itemize}
  \tightlist
  \item
    A supervised technique that projects data into a lower-dimensional
    space to maximize class separability.
  \item
    Useful for classification tasks.
  \end{itemize}
\end{enumerate}

\begin{enumerate}
\def\labelenumi{\arabic{enumi}.}
\setcounter{enumi}{2}
\tightlist
\item
  \textbf{t-Distributed Stochastic Neighbor Embedding (t-SNE)}:

  \begin{itemize}
  \tightlist
  \item
    A non-linear technique for visualizing high-dimensional data in 2D
    or 3D.
  \item
    Preserves local relationships while reducing dimensionality.
  \end{itemize}
\end{enumerate}

\begin{enumerate}
\def\labelenumi{\arabic{enumi}.}
\setcounter{enumi}{3}
\tightlist
\item
  \textbf{Autoencoders}:

  \begin{itemize}
  \tightlist
  \item
    Neural networks designed to learn compressed representations of
    data.
  \item
    The encoder reduces the dimensionality, while the decoder
    reconstructs the original data from the compressed representation.
  \end{itemize}
\end{enumerate}

\textbf{Benefits of Feature Extraction}

\begin{itemize}
\tightlist
\item
  \textbf{Enhanced Generalization}: Captures underlying patterns or
  structures in the data.
\item
  \textbf{Reduced Dimensionality}: Transforms data into a compact
  representation while retaining important information.
\item
  \textbf{Applicability to High-Dimensional Data}: Effective when the
  number of original features is extremely large (e.g., image or text
  data).
\end{itemize}

    \paragraph{Feature Selection vs.~Feature
Extraction}\label{feature-selection-vs.-feature-extraction}

\begin{longtable}[]{@{}
  >{\raggedright\arraybackslash}p{(\columnwidth - 4\tabcolsep) * \real{0.2688}}
  >{\raggedright\arraybackslash}p{(\columnwidth - 4\tabcolsep) * \real{0.3656}}
  >{\raggedright\arraybackslash}p{(\columnwidth - 4\tabcolsep) * \real{0.3656}}@{}}
\toprule\noalign{}
\begin{minipage}[b]{\linewidth}\raggedright
Aspect
\end{minipage} & \begin{minipage}[b]{\linewidth}\raggedright
Feature Selection
\end{minipage} & \begin{minipage}[b]{\linewidth}\raggedright
Feature Extraction
\end{minipage} \\
\midrule\noalign{}
\endhead
\bottomrule\noalign{}
\endlastfoot
\textbf{Objective} & Retain a subset of original features. & Create new
features by transforming data. \\
\textbf{Output} & Subset of original features. & New features in a
transformed space. \\
\textbf{Interpretability} & Retains original feature meanings. & May
lose interpretability due to transformation. \\
\textbf{Examples} & Filter, wrapper, and embedded methods. & PCA, LDA,
t-SNE, autoencoders. \\
\textbf{Complexity} & Relatively simple. & May involve complex
transformations. \\
\end{longtable}

\paragraph{When to Use Feature Selection and Feature
Extraction}\label{when-to-use-feature-selection-and-feature-extraction}

\begin{itemize}
\tightlist
\item
  \textbf{Feature Selection}:

  \begin{itemize}
  \tightlist
  \item
    Use when interpretability of features is critical (e.g., medical
    diagnosis).
  \item
    Ideal when the dataset contains many irrelevant or redundant
    features.
  \end{itemize}
\item
  \textbf{Feature Extraction}:

  \begin{itemize}
  \tightlist
  \item
    Use when the dataset is high-dimensional and interpretability is
    less important.
  \item
    Suitable for tasks like image processing, natural language
    processing, and large-scale datasets.
  \end{itemize}
\end{itemize}

Both feature selection and feature extraction are powerful tools for
dimensionality reduction. The choice between them depends on the
specific problem, the nature of the dataset, and the need for
interpretability. In practice, these techniques are often used in
combination to preprocess data and enhance the performance of machine
learning models.

    \subsubsection{Sequential Backward Selection
(SBS)}\label{sequential-backward-selection-sbs}

Sequential Backward Selection (SBS) is a \textbf{\emph{wrapper}} method
for feature selection. It starts with the full set of features and
sequentially removes the least significant feature (one at a time) based
on a specified evaluation metric, such as accuracy, mean squared error
(MSE), or another performance measure of a model. The goal is to reduce
the feature set while maintaining or improving the model's performance.

\begin{quote}
Note : A \textbf{\emph{wrapper}} method in feature selection is an
approach that evaluates subsets of features by actually training and
testing a model on them. The selection process is guided by the model's
performance on a specified evaluation metric, such as accuracy,
precision, or mean squared error (MSE). Wrapper methods are
computationally intensive but often yield better feature subsets
compared to filter methods because they directly optimize for model
performance.
\end{quote}

\textbf{Steps in SBS}:

\begin{itemize}
\tightlist
\item
  \begin{enumerate}
  \def\labelenumi{\arabic{enumi}.}
  \tightlist
  \item
    \textbf{Start with all features}: Train a model using all the
    available features.
  \end{enumerate}
\item
  \begin{enumerate}
  \def\labelenumi{\arabic{enumi}.}
  \setcounter{enumi}{1}
  \tightlist
  \item
    \textbf{Evaluate feature importance}: Use a performance metric
    (e.g., cross-validated accuracy or MSE) to rank features.
  \end{enumerate}
\item
  \begin{enumerate}
  \def\labelenumi{\arabic{enumi}.}
  \setcounter{enumi}{2}
  \tightlist
  \item
    \textbf{Remove the least significant feature}: Eliminate the feature
    whose removal results in the smallest degradation (or the most
    improvement) in the performance metric.
  \end{enumerate}
\item
  \begin{enumerate}
  \def\labelenumi{\arabic{enumi}.}
  \setcounter{enumi}{3}
  \tightlist
  \item
    \textbf{Repeat}: Continue the process until a predefined number of
    features is reached or performance degrades significantly.
  \end{enumerate}
\end{itemize}

SBS is computationally expensive for high-dimensional datasets but is
effective in situations with a limited number of features or where
interpretability is crucial.

    \textbf{Example: Sequential Backward Selection with Financial Data}

Let's apply SBS to a financial dataset containing macroeconomic and
market indicators to predict stock returns.

\textbf{Dataset}:

Suppose we have the following predictors:

\begin{itemize}
\tightlist
\item
  GDP Growth Rate
\item
  Inflation Rate
\item
  Unemployment Rate
\item
  Interest Rate
\item
  S\&P 500 Returns
\item
  Bond Yield Spread
\item
  Oil Prices
\end{itemize}

The target variable is the 1-month forward return of a specific stock.

    \begin{Verbatim}[commandchars=\\\{\}]
{\color{incolor}In [{\color{incolor}4}]:} \PY{k+kn}{from} \PY{n+nn}{sklearn}\PY{n+nn}{.}\PY{n+nn}{feature\PYZus{}selection} \PY{k+kn}{import} \PY{n}{SequentialFeatureSelector}
        \PY{k+kn}{from} \PY{n+nn}{sklearn}\PY{n+nn}{.}\PY{n+nn}{linear\PYZus{}model} \PY{k+kn}{import} \PY{n}{LinearRegression}
        \PY{k+kn}{from} \PY{n+nn}{sklearn}\PY{n+nn}{.}\PY{n+nn}{model\PYZus{}selection} \PY{k+kn}{import} \PY{n}{train\PYZus{}test\PYZus{}split}
        \PY{k+kn}{from} \PY{n+nn}{sklearn}\PY{n+nn}{.}\PY{n+nn}{metrics} \PY{k+kn}{import} \PY{n}{mean\PYZus{}squared\PYZus{}error}
        \PY{k+kn}{import} \PY{n+nn}{pandas} \PY{k}{as} \PY{n+nn}{pd}
\end{Verbatim}

    \begin{Verbatim}[commandchars=\\\{\}]
{\color{incolor}In [{\color{incolor}5}]:} \PY{c+c1}{\PYZsh{} Example financial dataset}
        \PY{n}{data} \PY{o}{=} \PY{p}{\PYZob{}}
            \PY{l+s+s2}{\PYZdq{}}\PY{l+s+s2}{GDP Growth}\PY{l+s+s2}{\PYZdq{}}\PY{p}{:} \PY{p}{[}\PY{l+m+mf}{2.1}\PY{p}{,} \PY{l+m+mf}{1.9}\PY{p}{,} \PY{l+m+mf}{2.3}\PY{p}{,} \PY{l+m+mf}{2.0}\PY{p}{,} \PY{l+m+mf}{2.2}\PY{p}{]}\PY{p}{,}
            \PY{l+s+s2}{\PYZdq{}}\PY{l+s+s2}{Inflation}\PY{l+s+s2}{\PYZdq{}}\PY{p}{:} \PY{p}{[}\PY{l+m+mf}{1.8}\PY{p}{,} \PY{l+m+mf}{2.0}\PY{p}{,} \PY{l+m+mf}{1.9}\PY{p}{,} \PY{l+m+mf}{1.7}\PY{p}{,} \PY{l+m+mf}{2.1}\PY{p}{]}\PY{p}{,}
            \PY{l+s+s2}{\PYZdq{}}\PY{l+s+s2}{Unemployment}\PY{l+s+s2}{\PYZdq{}}\PY{p}{:} \PY{p}{[}\PY{l+m+mf}{4.0}\PY{p}{,} \PY{l+m+mf}{4.1}\PY{p}{,} \PY{l+m+mf}{3.9}\PY{p}{,} \PY{l+m+mf}{4.0}\PY{p}{,} \PY{l+m+mf}{4.1}\PY{p}{]}\PY{p}{,}
            \PY{l+s+s2}{\PYZdq{}}\PY{l+s+s2}{Interest Rate}\PY{l+s+s2}{\PYZdq{}}\PY{p}{:} \PY{p}{[}\PY{l+m+mf}{2.5}\PY{p}{,} \PY{l+m+mf}{2.6}\PY{p}{,} \PY{l+m+mf}{2.7}\PY{p}{,} \PY{l+m+mf}{2.5}\PY{p}{,} \PY{l+m+mf}{2.6}\PY{p}{]}\PY{p}{,}
            \PY{l+s+s2}{\PYZdq{}}\PY{l+s+s2}{S\PYZam{}P 500 Returns}\PY{l+s+s2}{\PYZdq{}}\PY{p}{:} \PY{p}{[}\PY{l+m+mf}{0.03}\PY{p}{,} \PY{l+m+mf}{0.02}\PY{p}{,} \PY{o}{\PYZhy{}}\PY{l+m+mf}{0.01}\PY{p}{,} \PY{l+m+mf}{0.04}\PY{p}{,} \PY{l+m+mf}{0.03}\PY{p}{]}\PY{p}{,}
            \PY{l+s+s2}{\PYZdq{}}\PY{l+s+s2}{Bond Yield Spread}\PY{l+s+s2}{\PYZdq{}}\PY{p}{:} \PY{p}{[}\PY{l+m+mf}{1.2}\PY{p}{,} \PY{l+m+mf}{1.1}\PY{p}{,} \PY{l+m+mf}{1.3}\PY{p}{,} \PY{l+m+mf}{1.0}\PY{p}{,} \PY{l+m+mf}{1.2}\PY{p}{]}\PY{p}{,}
            \PY{l+s+s2}{\PYZdq{}}\PY{l+s+s2}{Oil Prices}\PY{l+s+s2}{\PYZdq{}}\PY{p}{:} \PY{p}{[}\PY{l+m+mi}{60}\PY{p}{,} \PY{l+m+mi}{62}\PY{p}{,} \PY{l+m+mi}{59}\PY{p}{,} \PY{l+m+mi}{61}\PY{p}{,} \PY{l+m+mi}{63}\PY{p}{]}\PY{p}{,}
            \PY{l+s+s2}{\PYZdq{}}\PY{l+s+s2}{Stock Return}\PY{l+s+s2}{\PYZdq{}}\PY{p}{:} \PY{p}{[}\PY{l+m+mf}{0.05}\PY{p}{,} \PY{l+m+mf}{0.03}\PY{p}{,} \PY{l+m+mf}{0.02}\PY{p}{,} \PY{l+m+mf}{0.04}\PY{p}{,} \PY{l+m+mf}{0.06}\PY{p}{]}
        \PY{p}{\PYZcb{}}
\end{Verbatim}

    \begin{Verbatim}[commandchars=\\\{\}]
{\color{incolor}In [{\color{incolor}6}]:} \PY{n}{df} \PY{o}{=} \PY{n}{pd}\PY{o}{.}\PY{n}{DataFrame}\PY{p}{(}\PY{n}{data}\PY{p}{)}
        
        \PY{c+c1}{\PYZsh{} Features and target}
        \PY{n}{X} \PY{o}{=} \PY{n}{df}\PY{o}{.}\PY{n}{drop}\PY{p}{(}\PY{n}{columns}\PY{o}{=}\PY{p}{[}\PY{l+s+s2}{\PYZdq{}}\PY{l+s+s2}{Stock Return}\PY{l+s+s2}{\PYZdq{}}\PY{p}{]}\PY{p}{)}
        \PY{n}{y} \PY{o}{=} \PY{n}{df}\PY{p}{[}\PY{l+s+s2}{\PYZdq{}}\PY{l+s+s2}{Stock Return}\PY{l+s+s2}{\PYZdq{}}\PY{p}{]}
\end{Verbatim}

    \begin{Verbatim}[commandchars=\\\{\}]
{\color{incolor}In [{\color{incolor}7}]:} \PY{c+c1}{\PYZsh{} Train\PYZhy{}test split}
        \PY{n}{X\PYZus{}train}\PY{p}{,} \PY{n}{X\PYZus{}test}\PY{p}{,} \PY{n}{y\PYZus{}train}\PY{p}{,} \PY{n}{y\PYZus{}test} \PY{o}{=} \PY{n}{train\PYZus{}test\PYZus{}split}\PY{p}{(}\PY{n}{X}\PY{p}{,} \PY{n}{y}\PY{p}{,} \PY{n}{test\PYZus{}size}\PY{o}{=}\PY{l+m+mf}{0.2}\PY{p}{,} \PY{n}{random\PYZus{}state}\PY{o}{=}\PY{l+m+mi}{42}\PY{p}{)}
\end{Verbatim}

    \begin{Verbatim}[commandchars=\\\{\}]
{\color{incolor}In [{\color{incolor}8}]:} \PY{c+c1}{\PYZsh{} Linear Regression model}
        \PY{n}{model} \PY{o}{=} \PY{n}{LinearRegression}\PY{p}{(}\PY{p}{)}
\end{Verbatim}

    \begin{Verbatim}[commandchars=\\\{\}]
{\color{incolor}In [{\color{incolor}9}]:} \PY{l+s+sd}{\PYZsq{}\PYZsq{}\PYZsq{}}
        \PY{l+s+sd}{Sequential Backward Selection}
        
        \PY{l+s+sd}{The SequentialFeatureSelector in sklearn is a feature selection tool that iteratively selects or }
        \PY{l+s+sd}{eliminates features based on model performance. It works as a wrapper method, using a specified }
        \PY{l+s+sd}{predictive model to evaluate subsets of features and optimize for a given scoring metric (e.g., }
        \PY{l+s+sd}{accuracy, mean squared error).}
        
        \PY{l+s+sd}{Key Features:}
        
        \PY{l+s+sd}{    \PYZhy{} Direction:}
        \PY{l+s+sd}{        \PYZhy{} forward: Starts with no features and adds one at a time.}
        \PY{l+s+sd}{        \PYZhy{} backward: Starts with all features and removes one at a time.}
        \PY{l+s+sd}{    \PYZhy{} Scoring Metric: Evaluates feature subsets using metrics like accuracy or MSE (or custom scoring functions).}
        \PY{l+s+sd}{    \PYZhy{} Cross\PYZhy{}Validation: Supports cross\PYZhy{}validation for robust performance evaluation during feature selection.}
        \PY{l+s+sd}{    \PYZhy{} Customizable Subset Size: Allows specifying the desired number of features to select using the n\PYZus{}features\PYZus{}to\PYZus{}select parameter.}
        
        \PY{l+s+sd}{Usage:}
        
        \PY{l+s+sd}{    \PYZhy{} Define a predictive model (e.g., LinearRegression, RandomForestClassifier).}
        \PY{l+s+sd}{    \PYZhy{} Initialize SequentialFeatureSelector with the model, direction, scoring metric, and desired number of features.}
        \PY{l+s+sd}{    \PYZhy{} Fit the selector to the dataset to identify the optimal feature subset.}
        
        \PY{l+s+sd}{\PYZsq{}\PYZsq{}\PYZsq{}} 
        \PY{n}{sbs} \PY{o}{=} \PY{n}{SequentialFeatureSelector}\PY{p}{(}
            \PY{n}{model}\PY{p}{,} \PY{n}{direction}\PY{o}{=}\PY{l+s+s1}{\PYZsq{}}\PY{l+s+s1}{backward}\PY{l+s+s1}{\PYZsq{}}\PY{p}{,} \PY{n}{scoring}\PY{o}{=}\PY{l+s+s1}{\PYZsq{}}\PY{l+s+s1}{neg\PYZus{}mean\PYZus{}squared\PYZus{}error}\PY{l+s+s1}{\PYZsq{}}\PY{p}{,} \PY{n}{cv}\PY{o}{=}\PY{l+m+mi}{3}\PY{p}{,} \PY{n}{n\PYZus{}features\PYZus{}to\PYZus{}select}\PY{o}{=}\PY{l+m+mi}{3}
        \PY{p}{)}
        \PY{n}{sbs}\PY{o}{.}\PY{n}{fit}\PY{p}{(}\PY{n}{X\PYZus{}train}\PY{p}{,} \PY{n}{y\PYZus{}train}\PY{p}{)}
\end{Verbatim}
\begin{Verbatim}[commandchars=\\\{\}]
{\color{outcolor}Out[{\color{outcolor}9}]:} SequentialFeatureSelector(cv=3, direction='backward',
                                  estimator=LinearRegression(), n\_features\_to\_select=3,
                                  scoring='neg\_mean\_squared\_error')
\end{Verbatim}
    \begin{Verbatim}[commandchars=\\\{\}]
{\color{incolor}In [{\color{incolor}10}]:} \PY{c+c1}{\PYZsh{} Selected features}
         \PY{n}{selected\PYZus{}features} \PY{o}{=} \PY{n}{X}\PY{o}{.}\PY{n}{columns}\PY{p}{[}\PY{n}{sbs}\PY{o}{.}\PY{n}{get\PYZus{}support}\PY{p}{(}\PY{p}{)}\PY{p}{]}
         \PY{n+nb}{print}\PY{p}{(}\PY{l+s+s2}{\PYZdq{}}\PY{l+s+s2}{Selected Features:}\PY{l+s+s2}{\PYZdq{}}\PY{p}{,} \PY{n}{selected\PYZus{}features}\PY{p}{)}
\end{Verbatim}

    \begin{Verbatim}[commandchars=\\\{\}]
Selected Features: Index(['GDP Growth', 'Inflation', 'Interest Rate'], dtype='object')

    \end{Verbatim}

    \begin{Verbatim}[commandchars=\\\{\}]
{\color{incolor}In [{\color{incolor}11}]:} \PY{c+c1}{\PYZsh{} Evaluate performance with reduced features}
         \PY{n}{X\PYZus{}train\PYZus{}reduced} \PY{o}{=} \PY{n}{X\PYZus{}train}\PY{p}{[}\PY{n}{selected\PYZus{}features}\PY{p}{]}
         \PY{n}{X\PYZus{}test\PYZus{}reduced} \PY{o}{=} \PY{n}{X\PYZus{}test}\PY{p}{[}\PY{n}{selected\PYZus{}features}\PY{p}{]}
         \PY{n}{model}\PY{o}{.}\PY{n}{fit}\PY{p}{(}\PY{n}{X\PYZus{}train\PYZus{}reduced}\PY{p}{,} \PY{n}{y\PYZus{}train}\PY{p}{)}
         \PY{n}{predictions} \PY{o}{=} \PY{n}{model}\PY{o}{.}\PY{n}{predict}\PY{p}{(}\PY{n}{X\PYZus{}test\PYZus{}reduced}\PY{p}{)}
         \PY{n}{mse} \PY{o}{=} \PY{n}{mean\PYZus{}squared\PYZus{}error}\PY{p}{(}\PY{n}{y\PYZus{}test}\PY{p}{,} \PY{n}{predictions}\PY{p}{)}
         \PY{n+nb}{print}\PY{p}{(}\PY{l+s+s2}{\PYZdq{}}\PY{l+s+s2}{Mean Squared Error with Selected Features:}\PY{l+s+s2}{\PYZdq{}}\PY{p}{,} \PY{n}{mse}\PY{p}{)}
\end{Verbatim}

    \begin{Verbatim}[commandchars=\\\{\}]
Mean Squared Error with Selected Features: 0.00039999999999999514

    \end{Verbatim}

    \subsection{Dimensionality Reduction}\label{dimensionality-reduction}

    \subsubsection{The Curse of
Dimensionality}\label{the-curse-of-dimensionality}

In machine learning, the curse of dimensionality refers to the
exponential increase in computational complexity and the sparsity of
data as the number of dimensions (features) in a dataset grows. First
described in the context of numerical analysis, this phenomenon has
significant implications for machine learning models, particularly those
involving distance-based metrics, density estimation, and optimization.

\textbf{Key Challenges}

\begin{enumerate}
\def\labelenumi{\arabic{enumi}.}
\tightlist
\item
  \textbf{Data Sparsity}

  \begin{itemize}
  \tightlist
  \item
    In high-dimensional spaces, data points become sparse. Intuitively,
    as dimensions increase, the volume of the space grows exponentially,
    but the number of data points often remains constant or grows
    linearly. This sparsity makes it difficult to model meaningful
    relationships between data points.
  \end{itemize}
\end{enumerate}

\begin{enumerate}
\def\labelenumi{\arabic{enumi}.}
\setcounter{enumi}{1}
\tightlist
\item
  \textbf{Distance Metrics Become Less Informative}

  \begin{itemize}
  \tightlist
  \item
    Many machine learning algorithms rely on distance metrics (e.g.,
    Euclidean distance) to evaluate similarity between points. In
    high-dimensional spaces, the differences in distances between the
    nearest and farthest points tend to diminish, making it harder to
    differentiate between data points.
  \end{itemize}
\end{enumerate}

\begin{enumerate}
\def\labelenumi{\arabic{enumi}.}
\setcounter{enumi}{2}
\tightlist
\item
  \textbf{Overfitting}

  \begin{itemize}
  \tightlist
  \item
    High-dimensional datasets often contain redundant or irrelevant
    features that do not contribute meaningful information. Including
    these features can lead models to overfit the training data, as they
    learn patterns that do not generalize well to new data.
  \end{itemize}
\end{enumerate}

\begin{enumerate}
\def\labelenumi{\arabic{enumi}.}
\setcounter{enumi}{3}
\tightlist
\item
  \textbf{Increased Computational Cost}

  \begin{itemize}
  \tightlist
  \item
    High-dimensional data requires significantly more computational
    resources for storage, processing, and model training. This can slow
    down training and make real-time inference challenging.
  \end{itemize}
\end{enumerate}

\begin{enumerate}
\def\labelenumi{\arabic{enumi}.}
\setcounter{enumi}{4}
\tightlist
\item
  \textbf{Visualization Challenges}

  \begin{itemize}
  \tightlist
  \item
    Human intuition is limited to three dimensions. When working with
    high-dimensional data, it becomes challenging to visualize and
    interpret the relationships and distributions of data points.
  \end{itemize}
\end{enumerate}

\textbf{Examples of Impact} - \textbf{K-Nearest Neighbors (KNN)}: In
high-dimensional spaces, all points can appear equidistant, reducing the
effectiveness of nearest-neighbor algorithms. - \textbf{Support Vector
Machines (SVM)}: In high dimensions, finding an optimal hyperplane
becomes computationally expensive and prone to overfitting without
proper regularization. - \textbf{Clustering Algorithms}: Clustering
methods like K-means rely on distance metrics, which lose effectiveness
in high dimensions.

\textbf{Approaches to Mitigate the Curse} To address the curse of
dimensionality, several strategies are employed: 1.
\textbf{Dimensionality Reduction}\\
- Techniques like Principal Component Analysis (PCA), t-SNE, or
autoencoders reduce the number of features while preserving the
essential information.

\begin{enumerate}
\def\labelenumi{\arabic{enumi}.}
\setcounter{enumi}{1}
\tightlist
\item
  \textbf{Feature Selection}

  \begin{itemize}
  \tightlist
  \item
    Selecting the most relevant features based on their contribution to
    the model's performance can help minimize redundancy and noise.
  \end{itemize}
\item
  \textbf{Regularization}

  \begin{itemize}
  \tightlist
  \item
    Applying regularization techniques (e.g., L1/L2 penalties) reduces
    overfitting by discouraging complex models that depend heavily on
    high-dimensional feature spaces.
  \end{itemize}
\item
  \textbf{Domain Knowledge}

  \begin{itemize}
  \tightlist
  \item
    Incorporating domain knowledge to engineer or select meaningful
    features can reduce dimensionality and improve interpretability.
  \end{itemize}
\end{enumerate}

    \subsubsection{A Simple Example}\label{a-simple-example}

It turns out that many things behave very differently in
high-dimensional space. For example, if you pick a random point in a
unit square (a 1 × 1 square), it will have only about a 0.4\% chance of
being located less than 0.001 from a border (in other words, it is very
unlikely that a random point will be ``extreme'' along any dimension).
But in a 10,000-dimensional unit hypercube (a 1 × 1 × ⋯ × 1 cube, with
ten thousand 1s), this probability is greater than 99.999999\%. Most
points in a high-dimensional hypercube are very close to the border.
This means the distance between points increases on average. The mean
distance of points from the center also grows as the dimensionality
increases. Let's see a simple example\ldots{}

    \begin{Verbatim}[commandchars=\\\{\}]
{\color{incolor}In [{\color{incolor}12}]:} \PY{c+c1}{\PYZsh{} The Curse of Dimensionality}
         \PY{c+c1}{\PYZsh{} \PYZhy{}\PYZhy{}\PYZhy{}\PYZhy{}\PYZhy{}\PYZhy{}\PYZhy{}\PYZhy{}\PYZhy{}\PYZhy{}\PYZhy{}\PYZhy{}\PYZhy{}\PYZhy{}\PYZhy{}\PYZhy{}\PYZhy{}\PYZhy{}\PYZhy{}\PYZhy{}\PYZhy{}\PYZhy{}\PYZhy{}\PYZhy{}\PYZhy{}\PYZhy{}\PYZhy{}\PYZhy{}}
         \PY{c+c1}{\PYZsh{} As the number of features (dimensions) increases in a dataset, the volume of the feature space grows exponentially.}
         \PY{c+c1}{\PYZsh{} This can lead to challenges such as:}
         \PY{c+c1}{\PYZsh{} \PYZhy{} **Sparsity**: Data points are spread thinly across the space, making it harder to identify meaningful patterns.}
         \PY{c+c1}{\PYZsh{} \PYZhy{} **Computational inefficiency**: Algorithms may become slow or unfeasible due to the large search space.}
         \PY{c+c1}{\PYZsh{} \PYZhy{} **Overfitting**: Models may capture noise rather than meaningful patterns, reducing their generalizability.}
         \PY{c+c1}{\PYZsh{} }
         \PY{c+c1}{\PYZsh{} Let\PYZsq{}s start by visualizing the curse of dimensionality:}
         
         \PY{k+kn}{import} \PY{n+nn}{numpy} \PY{k}{as} \PY{n+nn}{np}
         \PY{k+kn}{import} \PY{n+nn}{matplotlib}\PY{n+nn}{.}\PY{n+nn}{pyplot} \PY{k}{as} \PY{n+nn}{plt} 
         \PY{c+c1}{\PYZsh{} Generates synthetic datasets for classification problems}
         \PY{c+c1}{\PYZsh{} https://scikit\PYZhy{}learn.org/stable/modules/generated/sklearn.datasets.make\PYZus{}classification.html}
         \PY{k+kn}{from} \PY{n+nn}{sklearn}\PY{n+nn}{.}\PY{n+nn}{datasets} \PY{k+kn}{import} \PY{n}{make\PYZus{}classification}  
         \PY{c+c1}{\PYZsh{} Generate datasets with increasing dimensions}
         \PY{c+c1}{\PYZsh{} Each dataset has a specified number of dimensions, and we calculate the average distance of data points from the mean.}
         
         \PY{c+c1}{\PYZsh{} make\PYZus{}classification:}
         \PY{c+c1}{\PYZsh{} This function generates a synthetic dataset useful for classification tasks. It allows control over:}
         \PY{c+c1}{\PYZsh{} \PYZhy{} `n\PYZus{}samples`: The number of samples (rows) in the dataset.}
         \PY{c+c1}{\PYZsh{} \PYZhy{} `n\PYZus{}features`: Total number of features (columns) in the dataset. This includes both informative and non\PYZhy{}informative features.}
         \PY{c+c1}{\PYZsh{} \PYZhy{} `n\PYZus{}informative`: The number of informative features, which contribute directly to the target labels. These features have predictive value.}
         \PY{c+c1}{\PYZsh{} \PYZhy{} **Difference**: `n\PYZus{}features` represents the total features in the dataset, including noise and redundant features. `n\PYZus{}informative` is a subset of these features that are directly useful for prediction.}
         \PY{c+c1}{\PYZsh{} \PYZhy{} `n\PYZus{}redundant`: Features that are linear combinations of the informative features and add no additional predictive value.}
         \PY{c+c1}{\PYZsh{} \PYZhy{} `n\PYZus{}classes`: The number of distinct classes in the dataset.}
         \PY{c+c1}{\PYZsh{} \PYZhy{} `random\PYZus{}state`: Ensures reproducibility by fixing the random seed.}
         \PY{n}{dimensions} \PY{o}{=} \PY{p}{[}\PY{l+m+mi}{10}\PY{p}{,}\PY{l+m+mi}{20}\PY{p}{,}\PY{l+m+mi}{30}\PY{p}{,}\PY{l+m+mi}{40}\PY{p}{,}\PY{l+m+mi}{50}\PY{p}{,}\PY{l+m+mi}{100}\PY{p}{,}\PY{l+m+mi}{1000}\PY{p}{]}
         \PY{n}{data\PYZus{}sparsity} \PY{o}{=} \PY{p}{[}\PY{p}{]}
         
         \PY{k}{for} \PY{n}{d} \PY{o+ow}{in} \PY{n}{dimensions}\PY{p}{:}
             \PY{c+c1}{\PYZsh{} Create synthetic data with \PYZsq{}d\PYZsq{} dimensions}
             \PY{n}{X}\PY{p}{,} \PY{n}{\PYZus{}} \PY{o}{=} \PY{n}{make\PYZus{}classification}\PY{p}{(}\PY{n}{n\PYZus{}samples}\PY{o}{=}\PY{l+m+mi}{1000}\PY{p}{,} \PY{n}{n\PYZus{}features}\PY{o}{=}\PY{n}{d}\PY{p}{,} \PY{n}{n\PYZus{}informative}\PY{o}{=}\PY{l+m+mi}{2}\PY{p}{,} \PY{n}{n\PYZus{}redundant}\PY{o}{=}\PY{l+m+mi}{0}\PY{p}{,} \PY{n}{random\PYZus{}state}\PY{o}{=}\PY{l+m+mi}{42}\PY{p}{)}
             \PY{c+c1}{\PYZsh{} Compute the average distance of points from the mean in this feature space}
             \PY{n}{sparsity} \PY{o}{=} \PY{n}{np}\PY{o}{.}\PY{n}{mean}\PY{p}{(}\PY{n}{np}\PY{o}{.}\PY{n}{linalg}\PY{o}{.}\PY{n}{norm}\PY{p}{(}\PY{n}{X} \PY{o}{\PYZhy{}} \PY{n}{X}\PY{o}{.}\PY{n}{mean}\PY{p}{(}\PY{n}{axis}\PY{o}{=}\PY{l+m+mi}{0}\PY{p}{)}\PY{p}{,} \PY{n}{axis}\PY{o}{=}\PY{l+m+mi}{1}\PY{p}{)}\PY{p}{)}
             \PY{n}{data\PYZus{}sparsity}\PY{o}{.}\PY{n}{append}\PY{p}{(}\PY{n}{sparsity}\PY{p}{)}
         
         \PY{c+c1}{\PYZsh{} Plot the effect of increasing dimensions on data sparsity}
         \PY{n}{plt}\PY{o}{.}\PY{n}{plot}\PY{p}{(}\PY{n}{dimensions}\PY{p}{,} \PY{n}{data\PYZus{}sparsity}\PY{p}{,} \PY{n}{marker}\PY{o}{=}\PY{l+s+s1}{\PYZsq{}}\PY{l+s+s1}{o}\PY{l+s+s1}{\PYZsq{}}\PY{p}{)}
         \PY{n}{plt}\PY{o}{.}\PY{n}{xlabel}\PY{p}{(}\PY{l+s+s1}{\PYZsq{}}\PY{l+s+s1}{Number of Dimensions}\PY{l+s+s1}{\PYZsq{}}\PY{p}{)}
         \PY{n}{plt}\PY{o}{.}\PY{n}{ylabel}\PY{p}{(}\PY{l+s+s1}{\PYZsq{}}\PY{l+s+s1}{Average Distance to Mean}\PY{l+s+s1}{\PYZsq{}}\PY{p}{)}
         \PY{n}{plt}\PY{o}{.}\PY{n}{title}\PY{p}{(}\PY{l+s+s1}{\PYZsq{}}\PY{l+s+s1}{Curse of Dimensionality}\PY{l+s+s1}{\PYZsq{}}\PY{p}{)}
         \PY{n}{plt}\PY{o}{.}\PY{n}{grid}\PY{p}{(}\PY{k+kc}{True}\PY{p}{)}
         \PY{n}{plt}\PY{o}{.}\PY{n}{show}\PY{p}{(}\PY{p}{)}
\end{Verbatim}

    \begin{center}
    \adjustimage{max size={0.9\linewidth}{0.9\paperheight}}{output_108_0.png}
    \end{center}
    { \hspace*{\fill} \\}
    
    This fact implies that highdimensional datasets are at risk of being
very sparse: most training instances are likely to be far away from each
other. Of course, this also means that a new instance will likely be far
away from any training instance, making predictions much less reliable
than in lower dimensions, since they will be based on much larger
extrapolations. In short, the more dimensions the training set has, the
greater the risk of overfitting it. In theory, one solution to the curse
of dimensionality could be to increase the size of the training set to
reach a sufficient density of training instances. Unfortunately, in
practice, the number of training instances required to reach a given
density grows exponentially with the number of dimensions.

    \subsubsection{Projections}\label{projections}

Projection is one of the most common techniques for dimensionality
reduction. The idea is to map high-dimensional data into a
lower-dimensional subspace, such as a plane or a line, in a way that
preserves key properties of the data. This idea is based on the fact
that in most real-world problems, training instances are not spread out
uniformly across all dimensions. Many features are almost constant,
while others are highly correlated. As a result, all training instances
actually lie within (or close to) a much lower-dimensional subspace of
the high-dimensional space.

    \paragraph{Principal Component Analysis
(PCA)}\label{principal-component-analysis-pca}

Principal Component Analysis (PCA) is a foundational technique in the
field of data analysis, statistics, and machine learning, employed to
reduce the dimensionality of a dataset while preserving as much of its
variability as possible. By identifying new orthogonal axes, or
principal components, PCA transforms the data into a new coordinate
system where the greatest variance lies along the first axis, the second
greatest variance lies along the second axis, and so forth. This
transformation serves to simplify the dataset, remove redundant
features, and highlight latent structures in the data, making PCA a
valuable tool in both exploratory data analysis and preprocessing for
machine learning algorithms.

The primary motivation for PCA is rooted in the challenges posed by
high-dimensional data. In many practical scenarios, datasets contain
numerous correlated features, leading to redundancy and difficulties in
interpretation. Additionally, as the dimensionality of the dataset
increases, computational efficiency and model performance often
deteriorate, a phenomenon known as the ``curse of dimensionality.'' PCA
addresses these challenges by projecting the data onto a
lower-dimensional subspace, defined by the principal components, without
a significant loss of information. The method relies on the mathematical
framework of linear algebra and statistics, making it both theoretically
elegant and computationally efficient.

To fully understand PCA, we must derive its mathematical foundation. Let
us consider a dataset \(\mathbf{X} \in \mathbb{R}^{n \times d}\), where
\(n\) is the number of observations and \(d\) is the number of features.
Each row of \(\mathbf{X}\) represents a data point, and each column
represents a feature. The goal of PCA is to find a set of orthogonal
axes, known as principal components, onto which the data can be
projected, such that the variance of the projected data along these axes
is maximized.

    \paragraph{PCA using SVD}\label{pca-using-svd}

\textbf{Simple Dataset Generation}

\begin{itemize}
\tightlist
\item
  Trigonometric functions (cos and sin) produce periodic patterns for
  X{[}:, 0{]} and X{[}:, 1{]}.
\item
  Random angles ensure a spread of points along the trigonometric curve.
\item
  X{[}:, 2{]} combines the first two dimensions with specific weights
  (w1 and w2) to simulate a dependent variable with added randomness.
\end{itemize}

    \begin{Verbatim}[commandchars=\\\{\}]
{\color{incolor}In [{\color{incolor}13}]:} \PY{c+c1}{\PYZsh{}}
         \PY{c+c1}{\PYZsh{} Explanation:}
         \PY{c+c1}{\PYZsh{} \PYZhy{} The dataset is generated in three dimensions.}
         \PY{c+c1}{\PYZsh{} \PYZhy{} The first two dimensions (X[:, 0] and X[:, 1]) are based on trigonometric functions of random angles.}
         \PY{c+c1}{\PYZsh{} \PYZhy{} The third dimension (X[:, 2]) is created as a weighted linear combination of the first two dimensions, with added noise.}
         \PY{c+c1}{\PYZsh{} \PYZhy{} Noise is introduced to make the dataset more realistic by simulating measurement errors or natural variations.}
         \PY{c+c1}{\PYZsh{}}
         \PY{k+kn}{import} \PY{n+nn}{numpy} \PY{k}{as} \PY{n+nn}{np}
         \PY{k+kn}{import} \PY{n+nn}{matplotlib}\PY{n+nn}{.}\PY{n+nn}{pyplot} \PY{k}{as} \PY{n+nn}{plt}
         
         \PY{c+c1}{\PYZsh{} Set the random seed for reproducibility}
         \PY{n}{np}\PY{o}{.}\PY{n}{random}\PY{o}{.}\PY{n}{seed}\PY{p}{(}\PY{l+m+mi}{4}\PY{p}{)}
         \PY{c+c1}{\PYZsh{} Number of data points to generate}
         \PY{n}{m} \PY{o}{=} \PY{l+m+mi}{1000}
         \PY{c+c1}{\PYZsh{} Coefficients for the linear combination in the Z dimension}
         \PY{n}{w1}\PY{p}{,} \PY{n}{w2} \PY{o}{=} \PY{l+m+mf}{0.1}\PY{p}{,} \PY{l+m+mf}{0.3}
         \PY{c+c1}{\PYZsh{} Noise level to add random variation to the data}
         \PY{n}{noise} \PY{o}{=} \PY{l+m+mf}{0.1}
         \PY{c+c1}{\PYZsh{} Generate random angles uniformly distributed between \PYZhy{}0.5 and (3π/2 \PYZhy{} 0.5)}
         \PY{n}{angles} \PY{o}{=} \PY{n}{np}\PY{o}{.}\PY{n}{random}\PY{o}{.}\PY{n}{rand}\PY{p}{(}\PY{n}{m}\PY{p}{)} \PY{o}{*} \PY{l+m+mi}{3} \PY{o}{*} \PY{n}{np}\PY{o}{.}\PY{n}{pi} \PY{o}{/} \PY{l+m+mi}{2} \PY{o}{\PYZhy{}} \PY{l+m+mf}{0.5}
         \PY{c+c1}{\PYZsh{} Initialize an empty 3D array to store the dataset}
         \PY{n}{X} \PY{o}{=} \PY{n}{np}\PY{o}{.}\PY{n}{empty}\PY{p}{(}\PY{p}{(}\PY{n}{m}\PY{p}{,} \PY{l+m+mi}{3}\PY{p}{)}\PY{p}{)}
         \PY{c+c1}{\PYZsh{} Compute the X[:, 0] (first dimension) as a combination of sine and cosine functions of the angles}
         \PY{n}{X}\PY{p}{[}\PY{p}{:}\PY{p}{,} \PY{l+m+mi}{0}\PY{p}{]} \PY{o}{=} \PY{n}{np}\PY{o}{.}\PY{n}{cos}\PY{p}{(}\PY{n}{angles}\PY{p}{)} \PY{o}{+} \PY{n}{np}\PY{o}{.}\PY{n}{sin}\PY{p}{(}\PY{n}{angles}\PY{p}{)} \PY{o}{/} \PY{l+m+mi}{2} \PY{o}{+} \PY{n}{noise} \PY{o}{*} \PY{n}{np}\PY{o}{.}\PY{n}{random}\PY{o}{.}\PY{n}{randn}\PY{p}{(}\PY{n}{m}\PY{p}{)} \PY{o}{/} \PY{l+m+mi}{2}
         \PY{c+c1}{\PYZsh{} Compute the X[:, 1] (second dimension) using a scaled sine function and adding noise}
         \PY{n}{X}\PY{p}{[}\PY{p}{:}\PY{p}{,} \PY{l+m+mi}{1}\PY{p}{]} \PY{o}{=} \PY{n}{np}\PY{o}{.}\PY{n}{sin}\PY{p}{(}\PY{n}{angles}\PY{p}{)} \PY{o}{*} \PY{l+m+mf}{0.7} \PY{o}{+} \PY{n}{noise} \PY{o}{*} \PY{n}{np}\PY{o}{.}\PY{n}{random}\PY{o}{.}\PY{n}{randn}\PY{p}{(}\PY{n}{m}\PY{p}{)} \PY{o}{/} \PY{l+m+mi}{2}
         \PY{c+c1}{\PYZsh{} Compute the X[:, 2] (third dimension) as a linear combination of the first two dimensions plus noise}
         \PY{n}{X}\PY{p}{[}\PY{p}{:}\PY{p}{,} \PY{l+m+mi}{2}\PY{p}{]} \PY{o}{=} \PY{n}{X}\PY{p}{[}\PY{p}{:}\PY{p}{,} \PY{l+m+mi}{0}\PY{p}{]} \PY{o}{*} \PY{n}{w1} \PY{o}{+} \PY{n}{X}\PY{p}{[}\PY{p}{:}\PY{p}{,} \PY{l+m+mi}{1}\PY{p}{]} \PY{o}{*} \PY{n}{w2} \PY{o}{+} \PY{n}{noise} \PY{o}{*} \PY{n}{np}\PY{o}{.}\PY{n}{random}\PY{o}{.}\PY{n}{randn}\PY{p}{(}\PY{n}{m}\PY{p}{)}
\end{Verbatim}

    \begin{Verbatim}[commandchars=\\\{\}]
{\color{incolor}In [{\color{incolor}14}]:} \PY{c+c1}{\PYZsh{} Create a 3D scatter plot of the dataset}
         \PY{n}{fig} \PY{o}{=} \PY{n}{plt}\PY{o}{.}\PY{n}{figure}\PY{p}{(}\PY{n}{figsize}\PY{o}{=}\PY{p}{(}\PY{l+m+mi}{10}\PY{p}{,} \PY{l+m+mi}{7}\PY{p}{)}\PY{p}{)}
         \PY{n}{ax} \PY{o}{=} \PY{n}{fig}\PY{o}{.}\PY{n}{add\PYZus{}subplot}\PY{p}{(}\PY{l+m+mi}{111}\PY{p}{,} \PY{n}{projection}\PY{o}{=}\PY{l+s+s1}{\PYZsq{}}\PY{l+s+s1}{3d}\PY{l+s+s1}{\PYZsq{}}\PY{p}{)}
         \PY{c+c1}{\PYZsh{} Scatter plot with X[:, 0], X[:, 1], and X[:, 2]}
         \PY{n}{ax}\PY{o}{.}\PY{n}{scatter}\PY{p}{(}\PY{n}{X}\PY{p}{[}\PY{p}{:}\PY{p}{,} \PY{l+m+mi}{0}\PY{p}{]}\PY{p}{,} \PY{n}{X}\PY{p}{[}\PY{p}{:}\PY{p}{,} \PY{l+m+mi}{1}\PY{p}{]}\PY{p}{,} \PY{n}{X}\PY{p}{[}\PY{p}{:}\PY{p}{,} \PY{l+m+mi}{2}\PY{p}{]}\PY{p}{,} \PY{n}{c}\PY{o}{=}\PY{l+s+s1}{\PYZsq{}}\PY{l+s+s1}{b}\PY{l+s+s1}{\PYZsq{}}\PY{p}{,} \PY{n}{marker}\PY{o}{=}\PY{l+s+s1}{\PYZsq{}}\PY{l+s+s1}{.}\PY{l+s+s1}{\PYZsq{}}\PY{p}{,} \PY{n}{label}\PY{o}{=}\PY{l+s+s1}{\PYZsq{}}\PY{l+s+s1}{Data points}\PY{l+s+s1}{\PYZsq{}}\PY{p}{)}
         \PY{c+c1}{\PYZsh{} Add labels to the axes}
         \PY{n}{ax}\PY{o}{.}\PY{n}{set\PYZus{}xlabel}\PY{p}{(}\PY{l+s+s1}{\PYZsq{}}\PY{l+s+s1}{X0 (Dimension 1)}\PY{l+s+s1}{\PYZsq{}}\PY{p}{)}
         \PY{n}{ax}\PY{o}{.}\PY{n}{set\PYZus{}ylabel}\PY{p}{(}\PY{l+s+s1}{\PYZsq{}}\PY{l+s+s1}{X1 (Dimension 2)}\PY{l+s+s1}{\PYZsq{}}\PY{p}{)}
         \PY{n}{ax}\PY{o}{.}\PY{n}{set\PYZus{}zlabel}\PY{p}{(}\PY{l+s+s1}{\PYZsq{}}\PY{l+s+s1}{X2 (Dimension 3)}\PY{l+s+s1}{\PYZsq{}}\PY{p}{)}
         \PY{c+c1}{\PYZsh{} Add a title and legend}
         \PY{n}{ax}\PY{o}{.}\PY{n}{set\PYZus{}title}\PY{p}{(}\PY{l+s+s1}{\PYZsq{}}\PY{l+s+s1}{3D Dataset Visualization}\PY{l+s+s1}{\PYZsq{}}\PY{p}{)}
         \PY{n}{ax}\PY{o}{.}\PY{n}{legend}\PY{p}{(}\PY{p}{)}
         \PY{c+c1}{\PYZsh{} Show the plot}
         \PY{n}{plt}\PY{o}{.}\PY{n}{show}\PY{p}{(}\PY{p}{)}
\end{Verbatim}

    \begin{center}
    \adjustimage{max size={0.9\linewidth}{0.9\paperheight}}{output_114_0.png}
    \end{center}
    { \hspace*{\fill} \\}
    
    So how can you find the principal components of a training set? Luckily,
there is a standard matrix factorization technique called Singular Value
Decomposition (SVD) that can decompose the training set matrix X into
the matrix multiplication of three matrices

\[
U \Sigma V^T 
\] where \(V\) contains all the principal components that we are looking
for. The following Python code uses NumPy's \texttt{svd()} function to
obtain all the principal components of the training set, then extracts
the first two PCs:

    \begin{Verbatim}[commandchars=\\\{\}]
{\color{incolor}In [{\color{incolor}15}]:} \PY{c+c1}{\PYZsh{} Explanation:}
         \PY{c+c1}{\PYZsh{} 1. The dataset is centered by subtracting its mean along each feature, making it zero\PYZhy{}centered.}
         \PY{c+c1}{\PYZsh{} 2. SVD is performed to decompose the centered data into three components:}
         \PY{c+c1}{\PYZsh{}    \PYZhy{} U contains the directions of the samples in the transformed space.}
         \PY{c+c1}{\PYZsh{}    \PYZhy{} s holds the singular values, indicating the importance of each principal component.}
         \PY{c+c1}{\PYZsh{}    \PYZhy{} Vt contains the principal directions (columns of Vt.T).}
         \PY{c+c1}{\PYZsh{} 3. The first principal component (c1) is the direction along which the variance of the data is maximized.}
         \PY{c+c1}{\PYZsh{} 4. The second principal component (c2) is orthogonal to the first and explains the second highest variance.}
         
         \PY{c+c1}{\PYZsh{} The principal components c1 and c2 can now be used for further analysis, such as projecting the data onto these axes.}
         
         \PY{c+c1}{\PYZsh{} Center the dataset by subtracting the mean of each feature (column)}
         \PY{c+c1}{\PYZsh{} This ensures that the dataset has a mean of zero along each dimension}
         \PY{n}{X\PYZus{}centered} \PY{o}{=} \PY{n}{X} \PY{o}{\PYZhy{}} \PY{n}{X}\PY{o}{.}\PY{n}{mean}\PY{p}{(}\PY{n}{axis}\PY{o}{=}\PY{l+m+mi}{0}\PY{p}{)}
         
         \PY{c+c1}{\PYZsh{} Perform Singular Value Decomposition (SVD)}
         \PY{c+c1}{\PYZsh{} np.linalg.svd decomposes a matrix as: X\PYZus{}centered = U * diag(s) * Vt}
         \PY{c+c1}{\PYZsh{} \PYZhy{} U: Matrix with left singular vectors (m x m, orthonormal)}
         \PY{c+c1}{\PYZsh{} \PYZhy{} s: Singular values (1D array of size min(m, n))}
         \PY{c+c1}{\PYZsh{} \PYZhy{} Vt: Transpose of the right singular vectors (n x n, orthonormal)}
         \PY{n}{U}\PY{p}{,} \PY{n}{s}\PY{p}{,} \PY{n}{Vt} \PY{o}{=} \PY{n}{np}\PY{o}{.}\PY{n}{linalg}\PY{o}{.}\PY{n}{svd}\PY{p}{(}\PY{n}{X\PYZus{}centered}\PY{p}{)}
         
         \PY{c+c1}{\PYZsh{} Extract the first principal component}
         \PY{c+c1}{\PYZsh{} The first column of V (or Vt.T) corresponds to the direction of maximum variance}
         \PY{n}{c1} \PY{o}{=} \PY{n}{Vt}\PY{o}{.}\PY{n}{T}\PY{p}{[}\PY{p}{:}\PY{p}{,} \PY{l+m+mi}{0}\PY{p}{]}
         
         \PY{c+c1}{\PYZsh{} Extract the second principal component}
         \PY{c+c1}{\PYZsh{} The second column of V (or Vt.T) corresponds to the direction of the second highest variance}
         \PY{n}{c2} \PY{o}{=} \PY{n}{Vt}\PY{o}{.}\PY{n}{T}\PY{p}{[}\PY{p}{:}\PY{p}{,} \PY{l+m+mi}{1}\PY{p}{]}
         
         \PY{c+c1}{\PYZsh{} Display the results}
         \PY{n+nb}{print}\PY{p}{(}\PY{l+s+s2}{\PYZdq{}}\PY{l+s+s2}{First Principal Component (c1):}\PY{l+s+se}{\PYZbs{}n}\PY{l+s+s2}{\PYZdq{}}\PY{p}{,} \PY{n}{c1}\PY{p}{)}
         \PY{n+nb}{print}\PY{p}{(}\PY{l+s+s2}{\PYZdq{}}\PY{l+s+s2}{Second Principal Component (c2):}\PY{l+s+se}{\PYZbs{}n}\PY{l+s+s2}{\PYZdq{}}\PY{p}{,} \PY{n}{c2}\PY{p}{)}
\end{Verbatim}

    \begin{Verbatim}[commandchars=\\\{\}]
First Principal Component (c1):
 [0.94490587 0.27289805 0.18077487]
Second Principal Component (c2):
 [-0.31404016  0.91156632  0.26537827]

    \end{Verbatim}

    \begin{Verbatim}[commandchars=\\\{\}]
{\color{incolor}In [{\color{incolor}16}]:} \PY{k+kn}{import} \PY{n+nn}{matplotlib}\PY{n+nn}{.}\PY{n+nn}{pyplot} \PY{k}{as} \PY{n+nn}{plt}
         
         \PY{c+c1}{\PYZsh{} Calculate the mean of the dataset for centering}
         \PY{n}{mean} \PY{o}{=} \PY{n}{X}\PY{o}{.}\PY{n}{mean}\PY{p}{(}\PY{n}{axis}\PY{o}{=}\PY{l+m+mi}{0}\PY{p}{)}
         
         \PY{c+c1}{\PYZsh{} Scale factors for the principal directions to make them visible}
         \PY{n}{scale\PYZus{}factor} \PY{o}{=} \PY{l+m+mi}{2}
         
         \PY{c+c1}{\PYZsh{} Calculate endpoints of the principal directions}
         \PY{c+c1}{\PYZsh{} c1 and c2 are scaled and anchored at the mean}
         \PY{n}{pc1\PYZus{}end} \PY{o}{=} \PY{n}{mean} \PY{o}{+} \PY{n}{scale\PYZus{}factor} \PY{o}{*} \PY{n}{c1}
         \PY{n}{pc2\PYZus{}end} \PY{o}{=} \PY{n}{mean} \PY{o}{+} \PY{n}{scale\PYZus{}factor} \PY{o}{*} \PY{n}{c2}
         
         \PY{c+c1}{\PYZsh{} Create a 3D scatter plot for the dataset}
         \PY{n}{fig} \PY{o}{=} \PY{n}{plt}\PY{o}{.}\PY{n}{figure}\PY{p}{(}\PY{n}{figsize}\PY{o}{=}\PY{p}{(}\PY{l+m+mi}{10}\PY{p}{,} \PY{l+m+mi}{7}\PY{p}{)}\PY{p}{)}
         \PY{n}{ax} \PY{o}{=} \PY{n}{fig}\PY{o}{.}\PY{n}{add\PYZus{}subplot}\PY{p}{(}\PY{l+m+mi}{111}\PY{p}{,} \PY{n}{projection}\PY{o}{=}\PY{l+s+s1}{\PYZsq{}}\PY{l+s+s1}{3d}\PY{l+s+s1}{\PYZsq{}}\PY{p}{)}
         
         \PY{c+c1}{\PYZsh{} Plot the dataset points}
         \PY{n}{ax}\PY{o}{.}\PY{n}{scatter}\PY{p}{(}\PY{n}{X}\PY{p}{[}\PY{p}{:}\PY{p}{,} \PY{l+m+mi}{0}\PY{p}{]}\PY{p}{,} \PY{n}{X}\PY{p}{[}\PY{p}{:}\PY{p}{,} \PY{l+m+mi}{1}\PY{p}{]}\PY{p}{,} \PY{n}{X}\PY{p}{[}\PY{p}{:}\PY{p}{,} \PY{l+m+mi}{2}\PY{p}{]}\PY{p}{,} \PY{n}{c}\PY{o}{=}\PY{l+s+s1}{\PYZsq{}}\PY{l+s+s1}{b}\PY{l+s+s1}{\PYZsq{}}\PY{p}{,} \PY{n}{marker}\PY{o}{=}\PY{l+s+s1}{\PYZsq{}}\PY{l+s+s1}{.}\PY{l+s+s1}{\PYZsq{}}\PY{p}{,} \PY{n}{label}\PY{o}{=}\PY{l+s+s1}{\PYZsq{}}\PY{l+s+s1}{Data Points}\PY{l+s+s1}{\PYZsq{}}\PY{p}{)}
         
         \PY{c+c1}{\PYZsh{} Plot the first principal direction (c1) as a red arrow}
         \PY{n}{ax}\PY{o}{.}\PY{n}{quiver}\PY{p}{(}
             \PY{n}{mean}\PY{p}{[}\PY{l+m+mi}{0}\PY{p}{]}\PY{p}{,} \PY{n}{mean}\PY{p}{[}\PY{l+m+mi}{1}\PY{p}{]}\PY{p}{,} \PY{n}{mean}\PY{p}{[}\PY{l+m+mi}{2}\PY{p}{]}\PY{p}{,}  \PY{c+c1}{\PYZsh{} Starting point (mean)}
             \PY{n}{c1}\PY{p}{[}\PY{l+m+mi}{0}\PY{p}{]} \PY{o}{*} \PY{n}{scale\PYZus{}factor}\PY{p}{,} \PY{n}{c1}\PY{p}{[}\PY{l+m+mi}{1}\PY{p}{]} \PY{o}{*} \PY{n}{scale\PYZus{}factor}\PY{p}{,} \PY{n}{c1}\PY{p}{[}\PY{l+m+mi}{2}\PY{p}{]} \PY{o}{*} \PY{n}{scale\PYZus{}factor}\PY{p}{,}  \PY{c+c1}{\PYZsh{} Direction (scaled)}
             \PY{n}{color}\PY{o}{=}\PY{l+s+s1}{\PYZsq{}}\PY{l+s+s1}{r}\PY{l+s+s1}{\PYZsq{}}\PY{p}{,} \PY{n}{label}\PY{o}{=}\PY{l+s+s1}{\PYZsq{}}\PY{l+s+s1}{Principal Direction 1 (c1)}\PY{l+s+s1}{\PYZsq{}}
         \PY{p}{)}
         
         \PY{c+c1}{\PYZsh{} Plot the second principal direction (c2) as a green arrow}
         \PY{n}{ax}\PY{o}{.}\PY{n}{quiver}\PY{p}{(}
             \PY{n}{mean}\PY{p}{[}\PY{l+m+mi}{0}\PY{p}{]}\PY{p}{,} \PY{n}{mean}\PY{p}{[}\PY{l+m+mi}{1}\PY{p}{]}\PY{p}{,} \PY{n}{mean}\PY{p}{[}\PY{l+m+mi}{2}\PY{p}{]}\PY{p}{,}  \PY{c+c1}{\PYZsh{} Starting point (mean)}
             \PY{n}{c2}\PY{p}{[}\PY{l+m+mi}{0}\PY{p}{]} \PY{o}{*} \PY{n}{scale\PYZus{}factor}\PY{p}{,} \PY{n}{c2}\PY{p}{[}\PY{l+m+mi}{1}\PY{p}{]} \PY{o}{*} \PY{n}{scale\PYZus{}factor}\PY{p}{,} \PY{n}{c2}\PY{p}{[}\PY{l+m+mi}{2}\PY{p}{]} \PY{o}{*} \PY{n}{scale\PYZus{}factor}\PY{p}{,}  \PY{c+c1}{\PYZsh{} Direction (scaled)}
             \PY{n}{color}\PY{o}{=}\PY{l+s+s1}{\PYZsq{}}\PY{l+s+s1}{g}\PY{l+s+s1}{\PYZsq{}}\PY{p}{,} \PY{n}{label}\PY{o}{=}\PY{l+s+s1}{\PYZsq{}}\PY{l+s+s1}{Principal Direction 2 (c2)}\PY{l+s+s1}{\PYZsq{}}
         \PY{p}{)}
         
         \PY{c+c1}{\PYZsh{} Add axis labels}
         \PY{n}{ax}\PY{o}{.}\PY{n}{set\PYZus{}xlabel}\PY{p}{(}\PY{l+s+s1}{\PYZsq{}}\PY{l+s+s1}{X0 (Dimension 1)}\PY{l+s+s1}{\PYZsq{}}\PY{p}{)}
         \PY{n}{ax}\PY{o}{.}\PY{n}{set\PYZus{}ylabel}\PY{p}{(}\PY{l+s+s1}{\PYZsq{}}\PY{l+s+s1}{X1 (Dimension 2)}\PY{l+s+s1}{\PYZsq{}}\PY{p}{)}
         \PY{n}{ax}\PY{o}{.}\PY{n}{set\PYZus{}zlabel}\PY{p}{(}\PY{l+s+s1}{\PYZsq{}}\PY{l+s+s1}{X2 (Dimension 3)}\PY{l+s+s1}{\PYZsq{}}\PY{p}{)}
         
         \PY{c+c1}{\PYZsh{} Add a title and legend}
         \PY{n}{ax}\PY{o}{.}\PY{n}{set\PYZus{}title}\PY{p}{(}\PY{l+s+s1}{\PYZsq{}}\PY{l+s+s1}{3D Dataset with Principal Directions}\PY{l+s+s1}{\PYZsq{}}\PY{p}{)}
         \PY{n}{ax}\PY{o}{.}\PY{n}{legend}\PY{p}{(}\PY{p}{)}
         
         \PY{c+c1}{\PYZsh{} Show the plot}
         \PY{n}{plt}\PY{o}{.}\PY{n}{show}\PY{p}{(}\PY{p}{)}
\end{Verbatim}

    \begin{center}
    \adjustimage{max size={0.9\linewidth}{0.9\paperheight}}{output_117_0.png}
    \end{center}
    { \hspace*{\fill} \\}
    
    The following code constructs a diagonal matrix \(S\) using the singular
values \(s\) obtained from Singular Value Decomposition (SVD).

    \begin{Verbatim}[commandchars=\\\{\}]
{\color{incolor}In [{\color{incolor}17}]:} \PY{c+c1}{\PYZsh{} Explanation:}
         \PY{c+c1}{\PYZsh{} 1. S is part of the SVD decomposition: X\PYZus{}centered = U * S * V\PYZca{}T.}
         \PY{c+c1}{\PYZsh{} 2. It ensures S has the correct shape (m x n) for compatibility with U (m x m) and V\PYZca{}T (n x n).}
         \PY{c+c1}{\PYZsh{} 3. This step is essential if you plan to reconstruct the original dataset or analyze the singular values.}
         \PY{c+c1}{\PYZsh{}}
         \PY{c+c1}{\PYZsh{} Get the dimensions of the original dataset}
         \PY{c+c1}{\PYZsh{} m: number of data points (rows)}
         \PY{c+c1}{\PYZsh{} n: number of features (columns)}
         \PY{n}{m}\PY{p}{,} \PY{n}{n} \PY{o}{=} \PY{n}{X}\PY{o}{.}\PY{n}{shape}
         
         \PY{c+c1}{\PYZsh{} Initialize an empty matrix S of the same shape as the centered dataset}
         \PY{c+c1}{\PYZsh{} S will store the singular values in a diagonal matrix format}
         \PY{n}{S} \PY{o}{=} \PY{n}{np}\PY{o}{.}\PY{n}{zeros}\PY{p}{(}\PY{n}{X\PYZus{}centered}\PY{o}{.}\PY{n}{shape}\PY{p}{)}
         
         \PY{c+c1}{\PYZsh{} Populate the top\PYZhy{}left n x n submatrix of S with the singular values}
         \PY{c+c1}{\PYZsh{} \PYZhy{} np.diag(s) creates a square diagonal matrix from the singular values s}
         \PY{c+c1}{\PYZsh{} \PYZhy{} S[:n, :n] ensures only the first n rows and columns are filled, matching the shape of V\PYZca{}T}
         \PY{n}{S}\PY{p}{[}\PY{p}{:}\PY{n}{n}\PY{p}{,} \PY{p}{:}\PY{n}{n}\PY{p}{]} \PY{o}{=} \PY{n}{np}\PY{o}{.}\PY{n}{diag}\PY{p}{(}\PY{n}{s}\PY{p}{)}
\end{Verbatim}

    We can checks whether the matrix \texttt{X\_centered}is approximately
equal to the reconstructed matrix from its Singular Value Decomposition
(SVD).

\begin{itemize}
\tightlist
\item
  \textbf{\texttt{np.allclose()}}:

  \begin{itemize}
  \tightlist
  \item
    A NumPy function that checks if two arrays are element-wise equal
    within a specified tolerance.
  \item
    It returns \texttt{True} if all elements of the two arrays are
    approximately equal, and \texttt{False} otherwise.
  \item
    Default tolerances are:

    \begin{itemize}
    \tightlist
    \item
      \texttt{atol}: Absolute tolerance (default is \$1
      \times 10\^{}\{-8\} \$).
    \item
      \texttt{rtol}: Relative tolerance (default is \$1
      \times 10\^{}\{-5\} \$).
    \end{itemize}
  \end{itemize}
\item
  \textbf{\texttt{U.dot(S).dot(Vt)}}:

  \begin{itemize}
  \tightlist
  \item
    The reconstructed dataset using the SVD components:

    \begin{itemize}
    \tightlist
    \item
      \(U\): Left singular vectors.
    \item
      \(S\): Diagonal matrix of singular values.
    \item
      \(V^T\): Transpose of the right singular vectors.
    \end{itemize}
  \item
    The matrix product \(U \cdot S \cdot V^T\) should ideally
    reconstruct \(X_{\text{centered}}\).
  \end{itemize}
\end{itemize}

    \begin{Verbatim}[commandchars=\\\{\}]
{\color{incolor}In [{\color{incolor}18}]:} \PY{n}{np}\PY{o}{.}\PY{n}{allclose}\PY{p}{(}\PY{n}{X\PYZus{}centered}\PY{p}{,} \PY{n}{U}\PY{o}{.}\PY{n}{dot}\PY{p}{(}\PY{n}{S}\PY{p}{)}\PY{o}{.}\PY{n}{dot}\PY{p}{(}\PY{n}{Vt}\PY{p}{)}\PY{p}{)}
\end{Verbatim}
\begin{Verbatim}[commandchars=\\\{\}]
{\color{outcolor}Out[{\color{outcolor}18}]:} True
\end{Verbatim}
    To project the training set onto the hyperplane, you can simply compute
the matrix multiplication of the training set matrix \(X\) by the matrix
\(W_d\), defined as the matrix containing the first \(d\) principal
components (i.e., the matrix composed of the first \(d\) columns of
\(V\)). The next snapshot of code reduces the dimensionality of the
dataset from n-dimensions (original features) to 2D (first two principal
components). The reduced dataset retains the maximum variance possible
in the 2D subspace.

    \begin{Verbatim}[commandchars=\\\{\}]
{\color{incolor}In [{\color{incolor}19}]:} \PY{n}{W2} \PY{o}{=} \PY{n}{Vt}\PY{o}{.}\PY{n}{T}\PY{p}{[}\PY{p}{:}\PY{p}{,} \PY{p}{:}\PY{l+m+mi}{2}\PY{p}{]}
         \PY{n}{X2D} \PY{o}{=} \PY{n}{X\PYZus{}centered}\PY{o}{.}\PY{n}{dot}\PY{p}{(}\PY{n}{W2}\PY{p}{)}
         \PY{n}{X2D\PYZus{}using\PYZus{}svd} \PY{o}{=} \PY{n}{X2D}
         \PY{n+nb}{print}\PY{p}{(}\PY{n}{X2D\PYZus{}using\PYZus{}svd}\PY{p}{)}
\end{Verbatim}

    \begin{Verbatim}[commandchars=\\\{\}]
[[-1.22406148 -0.53113997]
 [ 0.15884932  0.44624623]
 [-1.23082641 -0.45801283]
 {\ldots}
 [-1.28801245 -0.08442393]
 [ 1.11641326 -0.09614666]
 [ 0.7136855   0.33106703]]

    \end{Verbatim}

    \paragraph{PCA using Scikit-Learn}\label{pca-using-scikit-learn}

With Scikit-Learn, PCA is really trivial. It even takes care of mean
centering for you:

    \begin{Verbatim}[commandchars=\\\{\}]
{\color{incolor}In [{\color{incolor}20}]:} \PY{k+kn}{from} \PY{n+nn}{sklearn}\PY{n+nn}{.}\PY{n+nn}{decomposition} \PY{k+kn}{import} \PY{n}{PCA}
         
         \PY{n}{pca} \PY{o}{=} \PY{n}{PCA}\PY{p}{(}\PY{n}{n\PYZus{}components}\PY{o}{=}\PY{l+m+mi}{2}\PY{p}{)}
         \PY{n}{X2D} \PY{o}{=} \PY{n}{pca}\PY{o}{.}\PY{n}{fit\PYZus{}transform}\PY{p}{(}\PY{n}{X}\PY{p}{)}
\end{Verbatim}

    \begin{Verbatim}[commandchars=\\\{\}]
{\color{incolor}In [{\color{incolor}21}]:} \PY{n}{X2D}\PY{p}{[}\PY{p}{:}\PY{l+m+mi}{5}\PY{p}{]}
\end{Verbatim}
\begin{Verbatim}[commandchars=\\\{\}]
{\color{outcolor}Out[{\color{outcolor}21}]:} array([[ 1.22406148,  0.53113997],
                [-0.15884932, -0.44624623],
                [ 1.23082641,  0.45801283],
                [ 0.82757991, -0.24037692],
                [ 0.81598401, -0.23558458]])
\end{Verbatim}
    \begin{Verbatim}[commandchars=\\\{\}]
{\color{incolor}In [{\color{incolor}22}]:} \PY{n}{X2D\PYZus{}using\PYZus{}svd}\PY{p}{[}\PY{p}{:}\PY{l+m+mi}{5}\PY{p}{]}
\end{Verbatim}
\begin{Verbatim}[commandchars=\\\{\}]
{\color{outcolor}Out[{\color{outcolor}22}]:} array([[-1.22406148, -0.53113997],
                [ 0.15884932,  0.44624623],
                [-1.23082641, -0.45801283],
                [-0.82757991,  0.24037692],
                [-0.81598401,  0.23558458]])
\end{Verbatim}
    Notice that running PCA multiple times on slightly different datasets
may result in different results. In general the only difference is that
some axes may be flipped. In this example, PCA using Scikit-Learn gives
the same projection as the one given by the SVD approach, except both
axes are flipped:

    \begin{Verbatim}[commandchars=\\\{\}]
{\color{incolor}In [{\color{incolor}23}]:} \PY{n}{np}\PY{o}{.}\PY{n}{allclose}\PY{p}{(}\PY{n}{X2D}\PY{p}{,} \PY{o}{\PYZhy{}}\PY{n}{X2D\PYZus{}using\PYZus{}svd}\PY{p}{)}
\end{Verbatim}
\begin{Verbatim}[commandchars=\\\{\}]
{\color{outcolor}Out[{\color{outcolor}23}]:} True
\end{Verbatim}
    \paragraph{Explained Variance Ratio}\label{explained-variance-ratio}

Another very useful piece of information is the explained variance ratio
of each principal component, available via the
explained\_variance\_ratio\_ variable. It indicates the proportion of
the dataset's variance that lies along the axis of each principal
component. For example, let's look at the explained variance ratios of
the first two components of the 3D dataset

    \begin{Verbatim}[commandchars=\\\{\}]
{\color{incolor}In [{\color{incolor}24}]:} \PY{n}{pca}\PY{o}{.}\PY{n}{explained\PYZus{}variance\PYZus{}ratio\PYZus{}}
\end{Verbatim}
\begin{Verbatim}[commandchars=\\\{\}]
{\color{outcolor}Out[{\color{outcolor}24}]:} array([0.85341663, 0.1369031 ])
\end{Verbatim}
    The first dimension explains 84.2\% of the variance, while the second
explains 14.6\%. By projecting down to 2D, we lost about 1\% of the
variance:

    \begin{Verbatim}[commandchars=\\\{\}]
{\color{incolor}In [{\color{incolor}25}]:} \PY{l+m+mi}{1} \PY{o}{\PYZhy{}} \PY{n}{pca}\PY{o}{.}\PY{n}{explained\PYZus{}variance\PYZus{}ratio\PYZus{}}\PY{o}{.}\PY{n}{sum}\PY{p}{(}\PY{p}{)}
\end{Verbatim}
\begin{Verbatim}[commandchars=\\\{\}]
{\color{outcolor}Out[{\color{outcolor}25}]:} 0.00968027240262126
\end{Verbatim}
    \paragraph{A Simple Example of PCA on Interest Rate
Data}\label{a-simple-example-of-pca-on-interest-rate-data}

PCA as a concept is useful for measuring risk arising from a set of
correlated market variables. For example, interest rates that are quoted
in the market have correlation to each other. Thus, an interest rate for
tenor 1Y is not independent of interest rate for tenor say 3Y. There is
always some degree of correlation between all the tenor points with each
other. To achieve this objective, the PCA model computes a set of
variables that are called as Principal components (PCs). PCA is a model
which involves transforming a set of observations (i.e.~interest rate
time series in our case) into a set of uncorrelated variables called as
the PCs. This transformation behaves in way such that the first PC
explains the largest possible variance, and this accounts for majority
of the variability in the data. Each succeeding component in turn
explains the highest possible variance while at the same time following
the condition of orthogonality to each of the preceding PCs. Resulting
PCs computed by the model are uncorrelated to each other, thereby
allowing them to be used independently with respect to each other.

    \begin{Verbatim}[commandchars=\\\{\}]
{\color{incolor}In [{\color{incolor}28}]:} \PY{k+kn}{from} \PY{n+nn}{sklearn}\PY{n+nn}{.}\PY{n+nn}{decomposition} \PY{k+kn}{import} \PY{n}{PCA}
         \PY{k+kn}{from} \PY{n+nn}{sklearn}\PY{n+nn}{.}\PY{n+nn}{preprocessing} \PY{k+kn}{import} \PY{n}{scale}
         \PY{k+kn}{import} \PY{n+nn}{pandas} \PY{k}{as} \PY{n+nn}{pd}
\end{Verbatim}

    \begin{Verbatim}[commandchars=\\\{\}]
{\color{incolor}In [{\color{incolor}30}]:} \PY{n}{path} \PY{o}{=} \PY{l+s+s1}{\PYZsq{}}\PY{l+s+s1}{./data/}\PY{l+s+s1}{\PYZsq{}}
         \PY{n}{data} \PY{o}{=} \PY{n}{pd}\PY{o}{.}\PY{n}{read\PYZus{}csv}\PY{p}{(}\PY{n}{path} \PY{o}{+} \PY{l+s+s1}{\PYZsq{}}\PY{l+s+s1}{MarketData.csv}\PY{l+s+s1}{\PYZsq{}}\PY{p}{,} \PY{n}{sep}\PY{o}{=}\PY{l+s+s1}{\PYZsq{}}\PY{l+s+s1}{,}\PY{l+s+s1}{\PYZsq{}}\PY{p}{)}
         \PY{n}{x} \PY{o}{=} \PY{n}{pd}\PY{o}{.}\PY{n}{DataFrame}\PY{p}{(}\PY{n}{data}\PY{p}{)}
         \PY{n}{x}\PY{o}{.}\PY{n}{head}\PY{p}{(}\PY{p}{)}
\end{Verbatim}
\begin{Verbatim}[commandchars=\\\{\}]
{\color{outcolor}Out[{\color{outcolor}30}]:}    Date    3m    6m    1y    2y    3y    4y    5y    7y   10y
         0     1  7.71  7.90  8.16  8.56  8.71  8.75  8.94  9.04  8.76
         1     2  7.75  7.93  8.17  8.58  8.72  8.76  8.96  9.05  8.76
         2     3  7.68  7.90  8.18  8.54  8.69  8.75  8.92  9.02  8.73
         3     4  7.69  7.93  8.22  8.55  8.70  8.76  8.93  9.03  8.75
         4     5  7.69  7.96  8.22  8.55  8.70  8.76  8.93  9.03  8.75
\end{Verbatim}
    \begin{Verbatim}[commandchars=\\\{\}]
{\color{incolor}In [{\color{incolor}31}]:} \PY{n}{df} \PY{o}{=} \PY{n}{x}\PY{o}{.}\PY{n}{drop}\PY{p}{(}\PY{n}{axis}\PY{o}{=}\PY{l+m+mi}{1}\PY{p}{,}\PY{n}{columns}\PY{o}{=}\PY{p}{[}\PY{l+s+s1}{\PYZsq{}}\PY{l+s+s1}{Date}\PY{l+s+s1}{\PYZsq{}}\PY{p}{]}\PY{p}{)}
         \PY{n}{X} \PY{o}{=} \PY{n}{df}\PY{o}{.}\PY{n}{values}
\end{Verbatim}

    \begin{Verbatim}[commandchars=\\\{\}]
{\color{incolor}In [{\color{incolor}32}]:} \PY{c+c1}{\PYZsh{}Normalization of the data}
         \PY{n}{X} \PY{o}{=} \PY{n}{scale}\PY{p}{(}\PY{n}{X}\PY{p}{)}
\end{Verbatim}

    Factor loadings can be calculated as below:

    \begin{Verbatim}[commandchars=\\\{\}]
{\color{incolor}In [{\color{incolor}33}]:} \PY{n}{pca} \PY{o}{=} \PY{n}{PCA}\PY{p}{(}\PY{n}{n\PYZus{}components}\PY{o}{=}\PY{l+m+mi}{9}\PY{p}{)}
         \PY{n}{pca}\PY{o}{.}\PY{n}{fit}\PY{p}{(}\PY{n}{X}\PY{p}{)}
         \PY{n}{factor\PYZus{}loading} \PY{o}{=} \PY{n}{pca}\PY{o}{.}\PY{n}{components\PYZus{}}
         \PY{n}{df\PYZus{}factor\PYZus{}loading} \PY{o}{=} \PY{n}{pd}\PY{o}{.}\PY{n}{DataFrame}\PY{p}{(}\PY{n}{factor\PYZus{}loading}\PY{p}{)}
         \PY{n}{df\PYZus{}factor\PYZus{}loading}
\end{Verbatim}
\begin{Verbatim}[commandchars=\\\{\}]
{\color{outcolor}Out[{\color{outcolor}33}]:}           0         1         2         3         4         5         6  \textbackslash{}
         0 -0.326333 -0.332127 -0.335253 -0.336409 -0.336407 -0.336645 -0.336749   
         1  0.544483  0.481083  0.254532  0.026220 -0.094576 -0.232499 -0.195293   
         2 -0.420097 -0.089586  0.239718  0.367905  0.368985  0.193024  0.162056   
         3 -0.483507  0.155548  0.552238  0.194323 -0.150791 -0.285729 -0.310074   
         4 -0.077813  0.031178 -0.068074 -0.189461 -0.066528  0.199704  0.509370   
         5  0.265687 -0.373126 -0.286407  0.548879  0.267776  0.002223 -0.390429   
         6  0.156875 -0.313745  0.089269  0.445102 -0.474288 -0.417560  0.506926   
         7 -0.113533  0.227273 -0.269583 -0.058133  0.562689 -0.692176  0.238268   
         8  0.269520 -0.579435  0.541370 -0.417573  0.318277 -0.142704  0.020379   
         
                   7         8  
         0 -0.331415 -0.328477  
         1 -0.439282 -0.335432  
         2 -0.254609 -0.598480  
         3 -0.106423  0.432952  
         4 -0.714587  0.371542  
         5 -0.316872  0.286948  
         6  0.089134 -0.081522  
         7  0.028055  0.078216  
         8 -0.035621  0.028572  
\end{Verbatim}
    Factor loadings explain the relation between the impact of a factor on
interest rates at respective tenor points. In PCA we also analyse the
amount of dispersion explained by each of the PCs. Now we will see which
PC contributes how much amount of variance/dispersion:

    \begin{Verbatim}[commandchars=\\\{\}]
{\color{incolor}In [{\color{incolor}34}]:} \PY{c+c1}{\PYZsh{} variance percent of each PC}
         \PY{n}{variance\PYZus{}percent\PYZus{}df} \PY{o}{=} \PY{n}{pd}\PY{o}{.}\PY{n}{DataFrame}\PY{p}{(}\PY{n}{data}\PY{o}{=}\PY{n}{pca}\PY{o}{.}\PY{n}{explained\PYZus{}variance\PYZus{}}\PY{p}{)}
         \PY{n}{variance\PYZus{}ratio\PYZus{}df} \PY{o}{=} \PY{n}{pd}\PY{o}{.}\PY{n}{DataFrame}\PY{p}{(}\PY{n}{data}\PY{o}{=}\PY{n}{pca}\PY{o}{.}\PY{n}{explained\PYZus{}variance\PYZus{}ratio\PYZus{}}\PY{p}{)}
         \PY{n}{variance\PYZus{}ratio\PYZus{}df} \PY{o}{=} \PY{n}{variance\PYZus{}ratio\PYZus{}df} \PY{o}{*} \PY{l+m+mi}{100}
         \PY{n}{variance\PYZus{}ratio\PYZus{}df}
\end{Verbatim}
\begin{Verbatim}[commandchars=\\\{\}]
{\color{outcolor}Out[{\color{outcolor}34}]:}            0
         0  96.406078
         1   1.931193
         2   1.183464
         3   0.242575
         4   0.125015
         5   0.059899
         6   0.032387
         7   0.013283
         8   0.006106
\end{Verbatim}
    From the table alongside, we observe that PC1 explains almost 96\% of
the total variation, and PC2 explains close to 1.95\% of total
variation. Therefore, rather than using all PCs in the subsequent
calculation, we can only use PC1 and PC2 in further calculation as these
two components explain close to 98\% of the total variance.

\begin{itemize}
\tightlist
\item
  PC1 corresponds to the roughly the parallel shift in the yield curve.
\item
  PC2 corresponds to roughly a steepening in the yield curve.
\end{itemize}

This is in-line with the theory of fixed income risk measurement which
states that majority of the movement in the price of a bond is explained
by the parallel shift in the yield curve and the residual movements in
the price is explained by steepening and curvature of the interest rate
curve.

    \subsubsection{Linear Discriminant Analysis
(LDA)}\label{linear-discriminant-analysis-lda}

\textbf{Linear Discriminant Analysis (LDA)} is a supervised machine
learning technique used for:

\begin{enumerate}
\def\labelenumi{\arabic{enumi}.}
\item
  \textbf{Dimensionality Reduction}: Transforming data into a
  lower-dimensional space while preserving class separability.
\item
  \textbf{Classification}: Modeling the differences between classes.
\end{enumerate}

It projects data onto a new axis (or axes) that maximize class
separability based on the relationship between features and class
labels.

\textbf{Key Objectives}

\begin{enumerate}
\def\labelenumi{\arabic{enumi}.}
\tightlist
\item
  \textbf{Maximize Between-Class Variance}:

  \begin{itemize}
  \tightlist
  \item
    Ensure the means of different classes are as far apart as possible
    in the projected space.
  \end{itemize}
\item
  \textbf{Minimize Within-Class Variance}:

  \begin{itemize}
  \tightlist
  \item
    Ensure the spread of data points within each class is as small as
    possible.
  \end{itemize}
\item
  \textbf{Find the Optimal Projection}:

  \begin{itemize}
  \tightlist
  \item
    LDA identifies a linear combination of features (a projection) that
    best separates multiple classes.
  \end{itemize}
\end{enumerate}

\textbf{Steps in LDA}

\begin{enumerate}
\def\labelenumi{\arabic{enumi}.}
\tightlist
\item
  \textbf{Compute the Mean of Each Class}:

  \begin{itemize}
  \tightlist
  \item
    For each class, compute the mean vector of the features.
  \end{itemize}
\end{enumerate}

\begin{enumerate}
\def\labelenumi{\arabic{enumi}.}
\setcounter{enumi}{1}
\item
  \textbf{Compute the Within-Class Scatter Matrix (\$ S\_W \$)}:

  \begin{itemize}
  \tightlist
  \item
    Measures the spread (variance) of data points within each class.
  \end{itemize}

  \[
  S_W = \sum_{i=1}^k \sum_{x \in C_i} (x - \mu_i)(x - \mu_i)^T
  \]

  Where \$ k \$ is the number of classes, \$ x \$ is a data point, and
  \$ \mu\_i \$ is the mean vector of class \$ i \$.
\end{enumerate}

\begin{enumerate}
\def\labelenumi{\arabic{enumi}.}
\setcounter{enumi}{2}
\item
  \textbf{Compute the Between-Class Scatter Matrix (\$ S\_B \$)}:

  \begin{itemize}
  \tightlist
  \item
    Measures the spread of class means relative to the overall mean.
  \end{itemize}

  \[
  S_B = \sum_{i=1}^k N_i (\mu_i - \mu)(\mu_i - \mu)^T
  \]

  Where \$ N\_i \$ is the number of samples in class \$ i \$, and \$
  \mu \$ is the overall mean vector.
\end{enumerate}

\begin{enumerate}
\def\labelenumi{\arabic{enumi}.}
\setcounter{enumi}{3}
\item
  \textbf{Solve the Generalized Eigenvalue Problem}:

  \begin{itemize}
  \tightlist
  \item
    Find the eigenvalues and eigenvectors of the matrix:
  \end{itemize}

  \[
    S_W^{-1} S_B
    \]

  \begin{itemize}
  \tightlist
  \item
    The eigenvectors define the directions of the new axes (discriminant
    components), and the eigenvalues indicate their importance.
  \end{itemize}
\end{enumerate}

\begin{enumerate}
\def\labelenumi{\arabic{enumi}.}
\setcounter{enumi}{4}
\tightlist
\item
  \textbf{Select the Top Discriminant Components}:

  \begin{itemize}
  \tightlist
  \item
    Choose the eigenvectors corresponding to the largest eigenvalues.
    For \$ k \$ classes, at most \$ k-1 \$ discriminant components can
    be used.
  \end{itemize}
\end{enumerate}

\begin{enumerate}
\def\labelenumi{\arabic{enumi}.}
\setcounter{enumi}{5}
\tightlist
\item
  \textbf{Project the Data}:

  \begin{itemize}
  \tightlist
  \item
    Transform the data into the new space defined by the top
    discriminant components.
  \end{itemize}
\end{enumerate}

\textbf{Differences Between LDA and PCA}

\begin{longtable}[]{@{}
  >{\raggedright\arraybackslash}p{(\columnwidth - 4\tabcolsep) * \real{0.2717}}
  >{\raggedright\arraybackslash}p{(\columnwidth - 4\tabcolsep) * \real{0.3696}}
  >{\raggedright\arraybackslash}p{(\columnwidth - 4\tabcolsep) * \real{0.3587}}@{}}
\toprule\noalign{}
\begin{minipage}[b]{\linewidth}\raggedright
Aspect
\end{minipage} & \begin{minipage}[b]{\linewidth}\raggedright
LDA
\end{minipage} & \begin{minipage}[b]{\linewidth}\raggedright
PCA
\end{minipage} \\
\midrule\noalign{}
\endhead
\bottomrule\noalign{}
\endlastfoot
\textbf{Type} & Supervised & Unsupervised \\
\textbf{Goal} & Maximize class separability & Maximize variance \\
\textbf{Components} & Based on class relationships & Based on overall
data variance \\
\textbf{Application} & Classification, Dimensionality Reduction &
Dimensionality Reduction \\
\end{longtable}

\textbf{Applications}

\begin{enumerate}
\def\labelenumi{\arabic{enumi}.}
\tightlist
\item
  \textbf{Dimensionality Reduction}:

  \begin{itemize}
  \tightlist
  \item
    Preprocessing step for high-dimensional data before applying a
    classifier.
  \end{itemize}
\item
  \textbf{Classification}:

  \begin{itemize}
  \tightlist
  \item
    Used as a classifier itself or to improve the performance of other
    classification models.
  \end{itemize}
\item
  \textbf{Feature Selection}:

  \begin{itemize}
  \tightlist
  \item
    Identifying the most discriminative features for separating classes.
  \end{itemize}
\end{enumerate}

    \begin{Verbatim}[commandchars=\\\{\}]
{\color{incolor}In [{\color{incolor}35}]:} \PY{c+c1}{\PYZsh{}\PYZsh{}\PYZsh{} **Example Workflow (in Python)**}
         
         \PY{k+kn}{from} \PY{n+nn}{sklearn}\PY{n+nn}{.}\PY{n+nn}{discriminant\PYZus{}analysis} \PY{k+kn}{import} \PY{n}{LinearDiscriminantAnalysis}
         \PY{k+kn}{from} \PY{n+nn}{sklearn}\PY{n+nn}{.}\PY{n+nn}{datasets} \PY{k+kn}{import} \PY{n}{load\PYZus{}iris}
         \PY{k+kn}{import} \PY{n+nn}{matplotlib}\PY{n+nn}{.}\PY{n+nn}{pyplot} \PY{k}{as} \PY{n+nn}{plt}
         \PY{k+kn}{import} \PY{n+nn}{numpy} \PY{k}{as} \PY{n+nn}{np}
         
         \PY{c+c1}{\PYZsh{} Load the Iris dataset}
         \PY{n}{data} \PY{o}{=} \PY{n}{load\PYZus{}iris}\PY{p}{(}\PY{p}{)}
         \PY{n}{X} \PY{o}{=} \PY{n}{data}\PY{o}{.}\PY{n}{data}  \PY{c+c1}{\PYZsh{} Features}
         \PY{n}{y} \PY{o}{=} \PY{n}{data}\PY{o}{.}\PY{n}{target}  \PY{c+c1}{\PYZsh{} Class labels}
         
         \PY{c+c1}{\PYZsh{} Perform LDA}
         \PY{n}{lda} \PY{o}{=} \PY{n}{LinearDiscriminantAnalysis}\PY{p}{(}\PY{n}{n\PYZus{}components}\PY{o}{=}\PY{l+m+mi}{2}\PY{p}{)}
         \PY{n}{X\PYZus{}lda} \PY{o}{=} \PY{n}{lda}\PY{o}{.}\PY{n}{fit\PYZus{}transform}\PY{p}{(}\PY{n}{X}\PY{p}{,} \PY{n}{y}\PY{p}{)}
         
         \PY{c+c1}{\PYZsh{} Plot the projected data}
         \PY{n}{plt}\PY{o}{.}\PY{n}{figure}\PY{p}{(}\PY{n}{figsize}\PY{o}{=}\PY{p}{(}\PY{l+m+mi}{8}\PY{p}{,} \PY{l+m+mi}{6}\PY{p}{)}\PY{p}{)}
         \PY{k}{for} \PY{n}{i}\PY{p}{,} \PY{n}{label} \PY{o+ow}{in} \PY{n+nb}{enumerate}\PY{p}{(}\PY{n}{np}\PY{o}{.}\PY{n}{unique}\PY{p}{(}\PY{n}{y}\PY{p}{)}\PY{p}{)}\PY{p}{:}
             \PY{n}{plt}\PY{o}{.}\PY{n}{scatter}\PY{p}{(}
                 \PY{n}{X\PYZus{}lda}\PY{p}{[}\PY{n}{y} \PY{o}{==} \PY{n}{label}\PY{p}{,} \PY{l+m+mi}{0}\PY{p}{]}\PY{p}{,}
                 \PY{n}{X\PYZus{}lda}\PY{p}{[}\PY{n}{y} \PY{o}{==} \PY{n}{label}\PY{p}{,} \PY{l+m+mi}{1}\PY{p}{]}\PY{p}{,}
                 \PY{n}{label}\PY{o}{=}\PY{l+s+sa}{f}\PY{l+s+s1}{\PYZsq{}}\PY{l+s+s1}{Class }\PY{l+s+si}{\PYZob{}}\PY{n}{label}\PY{l+s+si}{\PYZcb{}}\PY{l+s+s1}{\PYZsq{}}
             \PY{p}{)}
         \PY{n}{plt}\PY{o}{.}\PY{n}{xlabel}\PY{p}{(}\PY{l+s+s1}{\PYZsq{}}\PY{l+s+s1}{LDA Component 1}\PY{l+s+s1}{\PYZsq{}}\PY{p}{)}
         \PY{n}{plt}\PY{o}{.}\PY{n}{ylabel}\PY{p}{(}\PY{l+s+s1}{\PYZsq{}}\PY{l+s+s1}{LDA Component 2}\PY{l+s+s1}{\PYZsq{}}\PY{p}{)}
         \PY{n}{plt}\PY{o}{.}\PY{n}{title}\PY{p}{(}\PY{l+s+s1}{\PYZsq{}}\PY{l+s+s1}{Linear Discriminant Analysis (LDA)}\PY{l+s+s1}{\PYZsq{}}\PY{p}{)}
         \PY{n}{plt}\PY{o}{.}\PY{n}{legend}\PY{p}{(}\PY{p}{)}
         \PY{n}{plt}\PY{o}{.}\PY{n}{grid}\PY{p}{(}\PY{k+kc}{True}\PY{p}{)}
         \PY{n}{plt}\PY{o}{.}\PY{n}{show}\PY{p}{(}\PY{p}{)}
\end{Verbatim}

    \begin{center}
    \adjustimage{max size={0.9\linewidth}{0.9\paperheight}}{output_145_0.png}
    \end{center}
    { \hspace*{\fill} \\}
    
    \textbf{Key Insights}

\begin{itemize}
\tightlist
\item
  LDA optimizes for class separability, making it particularly effective
  for classification problems.
\item
  It is computationally efficient and works well for linearly separable
  data.
\item
  LDA assumes:

  \begin{enumerate}
  \def\labelenumi{\arabic{enumi}.}
  \tightlist
  \item
    Classes have a Gaussian distribution.
  \item
    All classes share the same covariance matrix.
  \item
    Features are linearly related to the target.
  \end{enumerate}
\end{itemize}

    \subsection{Appendix}\label{appendix}

    \subsubsection{Mathematical Derivation of
PCA}\label{mathematical-derivation-of-pca}

\textbf{1. Centering the Data}

The first step in PCA is to center the data by subtracting the mean of
each feature from the corresponding feature values. Let
\(\mathbf{\mu} = \frac{1}{n} \sum_{i=1}^n \mathbf{x}_i\) denote the mean
vector of the dataset, where \(\mathbf{x}_i\) is the \(i\)-th row of
\(\mathbf{X}\). The centered dataset is given by:

\[
\mathbf{X}_c = \mathbf{X} - \mathbf{1}\mathbf{\mu}^\top,
\]

where \(\mathbf{1} \in \mathbb{R}^{n \times 1}\) is a column vector of
ones.

    \textbf{2. Covariance Matrix}

The variability in the data is quantified by the covariance matrix,
which captures the pairwise covariances between features. The covariance
matrix of the centered dataset is defined as:

\[
\mathbf{\Sigma} = \frac{1}{n} \mathbf{X}_c^\top \mathbf{X}_c,
\]

where \(\mathbf{\Sigma} \in \mathbb{R}^{d \times d}\). Each entry
\(\Sigma_{ij}\) represents the covariance between the \(i\)-th and
\(j\)-th features.

    \textbf{3. Principal Components as Eigenvectors}

The principal components are the directions in the feature space that
maximize the variance of the data when projected onto them.
Mathematically, this problem can be formulated as finding a vector
\(\mathbf{w} \in \mathbb{R}^d\) that maximizes the variance of the data
projected onto \(\mathbf{w}\). The variance of the projection is given
by:

\[
\text{Var}(\mathbf{w}) = \mathbf{w}^\top \mathbf{\Sigma} \mathbf{w}.
\]

To ensure a unique solution, we impose the constraint that
\(\mathbf{w}\) is a unit vector, i.e., \(\|\mathbf{w}\|_2 = 1\). This
leads to the optimization problem:

\[
\max_{\|\mathbf{w}\|_2=1} \mathbf{w}^\top \mathbf{\Sigma} \mathbf{w}.
\]

This is a constrained optimization problem that can be solved using the
method of Lagrange multipliers. The Lagrangian is given by:

\[
\mathcal{L}(\mathbf{w}, \lambda) = \mathbf{w}^\top \mathbf{\Sigma} \mathbf{w} - \lambda (\mathbf{w}^\top \mathbf{w} - 1),
\]

where \(\lambda\) is the Lagrange multiplier. Taking the gradient with
respect to \(\mathbf{w}\) and setting it to zero yields:

\[
\frac{\partial \mathcal{L}}{\partial \mathbf{w}} = 2\mathbf{\Sigma} \mathbf{w} - 2\lambda \mathbf{w} = 0.
\]

Simplifying, we obtain:

\[
\mathbf{\Sigma} \mathbf{w} = \lambda \mathbf{w}.
\]

This is the eigenvalue equation, where \(\lambda\) is an eigenvalue of
\(\mathbf{\Sigma}\), and \(\mathbf{w}\) is the corresponding
eigenvector.

    \textbf{FIGURA DA GERON}

    \textbf{\emph{Source: Géron A. ``Hand on Machine Learning with
Scikit-Learn Keras and Tensorflow''}}

    \textbf{4. Selecting Principal Components}

The eigenvectors of the covariance matrix \(\mathbf{\Sigma}\) represent
the principal directions of the data, while the eigenvalues indicate the
amount of variance captured along each principal direction. By sorting
the eigenvalues in descending order and selecting the top \(k\)
eigenvectors, we define the \(k\)-dimensional subspace that captures the
most variance.

Let \(\mathbf{W} \in \mathbb{R}^{d \times k}\) denote the matrix of the
top \(k\) eigenvectors. The reduced-dimensional representation of the
data is given by:

\[
\mathbf{X}_{\text{reduced}} = \mathbf{X}_c \mathbf{W}.
\]

    \textbf{Interpretation of the Results}

The principal components are orthogonal directions in the feature space
that explain the maximum variance of the data. The first principal
component explains the greatest variance, the second principal component
explains the second greatest variance, and so on. By projecting the data
onto these components, PCA reduces the dimensionality of the dataset
while retaining the most significant patterns and structures.

\textbf{Applications of PCA} PCA is widely used in various fields,
including image compression, exploratory data analysis, and
preprocessing for machine learning. Its ability to reduce noise, remove
multicollinearity, and simplify datasets makes it a valuable tool in
modern data science workflows.

    \subsection{Appendix}\label{appendix}

    A more realistic dataset for credit risk example

    \begin{Verbatim}[commandchars=\\\{\}]
{\color{incolor}In [{\color{incolor}63}]:} \PY{c+c1}{\PYZsh{} Define rating categories and their default probabilities}
         \PY{n}{rating\PYZus{}categories} \PY{o}{=} \PY{p}{\PYZob{}}
             \PY{l+s+s2}{\PYZdq{}}\PY{l+s+s2}{AAA}\PY{l+s+s2}{\PYZdq{}}\PY{p}{:} \PY{p}{\PYZob{}}\PY{l+s+s2}{\PYZdq{}}\PY{l+s+s2}{default\PYZus{}prob}\PY{l+s+s2}{\PYZdq{}}\PY{p}{:} \PY{l+m+mf}{0.01}\PY{p}{,} \PY{l+s+s2}{\PYZdq{}}\PY{l+s+s2}{income\PYZus{}range}\PY{l+s+s2}{\PYZdq{}}\PY{p}{:} \PY{p}{(}\PY{l+m+mi}{100000}\PY{p}{,} \PY{l+m+mi}{200000}\PY{p}{)}\PY{p}{,} \PY{l+s+s2}{\PYZdq{}}\PY{l+s+s2}{loan\PYZus{}range}\PY{l+s+s2}{\PYZdq{}}\PY{p}{:} \PY{p}{(}\PY{l+m+mi}{50000}\PY{p}{,} \PY{l+m+mi}{300000}\PY{p}{)}\PY{p}{\PYZcb{}}\PY{p}{,}
             \PY{l+s+s2}{\PYZdq{}}\PY{l+s+s2}{AA}\PY{l+s+s2}{\PYZdq{}}\PY{p}{:} \PY{p}{\PYZob{}}\PY{l+s+s2}{\PYZdq{}}\PY{l+s+s2}{default\PYZus{}prob}\PY{l+s+s2}{\PYZdq{}}\PY{p}{:} \PY{l+m+mf}{0.02}\PY{p}{,} \PY{l+s+s2}{\PYZdq{}}\PY{l+s+s2}{income\PYZus{}range}\PY{l+s+s2}{\PYZdq{}}\PY{p}{:} \PY{p}{(}\PY{l+m+mi}{90000}\PY{p}{,} \PY{l+m+mi}{180000}\PY{p}{)}\PY{p}{,} \PY{l+s+s2}{\PYZdq{}}\PY{l+s+s2}{loan\PYZus{}range}\PY{l+s+s2}{\PYZdq{}}\PY{p}{:} \PY{p}{(}\PY{l+m+mi}{40000}\PY{p}{,} \PY{l+m+mi}{250000}\PY{p}{)}\PY{p}{\PYZcb{}}\PY{p}{,}
             \PY{l+s+s2}{\PYZdq{}}\PY{l+s+s2}{A}\PY{l+s+s2}{\PYZdq{}}\PY{p}{:} \PY{p}{\PYZob{}}\PY{l+s+s2}{\PYZdq{}}\PY{l+s+s2}{default\PYZus{}prob}\PY{l+s+s2}{\PYZdq{}}\PY{p}{:} \PY{l+m+mf}{0.03}\PY{p}{,} \PY{l+s+s2}{\PYZdq{}}\PY{l+s+s2}{income\PYZus{}range}\PY{l+s+s2}{\PYZdq{}}\PY{p}{:} \PY{p}{(}\PY{l+m+mi}{80000}\PY{p}{,} \PY{l+m+mi}{160000}\PY{p}{)}\PY{p}{,} \PY{l+s+s2}{\PYZdq{}}\PY{l+s+s2}{loan\PYZus{}range}\PY{l+s+s2}{\PYZdq{}}\PY{p}{:} \PY{p}{(}\PY{l+m+mi}{35000}\PY{p}{,} \PY{l+m+mi}{200000}\PY{p}{)}\PY{p}{\PYZcb{}}\PY{p}{,}
             \PY{l+s+s2}{\PYZdq{}}\PY{l+s+s2}{BBB}\PY{l+s+s2}{\PYZdq{}}\PY{p}{:} \PY{p}{\PYZob{}}\PY{l+s+s2}{\PYZdq{}}\PY{l+s+s2}{default\PYZus{}prob}\PY{l+s+s2}{\PYZdq{}}\PY{p}{:} \PY{l+m+mf}{0.05}\PY{p}{,} \PY{l+s+s2}{\PYZdq{}}\PY{l+s+s2}{income\PYZus{}range}\PY{l+s+s2}{\PYZdq{}}\PY{p}{:} \PY{p}{(}\PY{l+m+mi}{60000}\PY{p}{,} \PY{l+m+mi}{140000}\PY{p}{)}\PY{p}{,} \PY{l+s+s2}{\PYZdq{}}\PY{l+s+s2}{loan\PYZus{}range}\PY{l+s+s2}{\PYZdq{}}\PY{p}{:} \PY{p}{(}\PY{l+m+mi}{30000}\PY{p}{,} \PY{l+m+mi}{150000}\PY{p}{)}\PY{p}{\PYZcb{}}\PY{p}{,}
             \PY{l+s+s2}{\PYZdq{}}\PY{l+s+s2}{BB}\PY{l+s+s2}{\PYZdq{}}\PY{p}{:} \PY{p}{\PYZob{}}\PY{l+s+s2}{\PYZdq{}}\PY{l+s+s2}{default\PYZus{}prob}\PY{l+s+s2}{\PYZdq{}}\PY{p}{:} \PY{l+m+mf}{0.10}\PY{p}{,} \PY{l+s+s2}{\PYZdq{}}\PY{l+s+s2}{income\PYZus{}range}\PY{l+s+s2}{\PYZdq{}}\PY{p}{:} \PY{p}{(}\PY{l+m+mi}{40000}\PY{p}{,} \PY{l+m+mi}{120000}\PY{p}{)}\PY{p}{,} \PY{l+s+s2}{\PYZdq{}}\PY{l+s+s2}{loan\PYZus{}range}\PY{l+s+s2}{\PYZdq{}}\PY{p}{:} \PY{p}{(}\PY{l+m+mi}{20000}\PY{p}{,} \PY{l+m+mi}{100000}\PY{p}{)}\PY{p}{\PYZcb{}}\PY{p}{,}
             \PY{l+s+s2}{\PYZdq{}}\PY{l+s+s2}{B}\PY{l+s+s2}{\PYZdq{}}\PY{p}{:} \PY{p}{\PYZob{}}\PY{l+s+s2}{\PYZdq{}}\PY{l+s+s2}{default\PYZus{}prob}\PY{l+s+s2}{\PYZdq{}}\PY{p}{:} \PY{l+m+mf}{0.15}\PY{p}{,} \PY{l+s+s2}{\PYZdq{}}\PY{l+s+s2}{income\PYZus{}range}\PY{l+s+s2}{\PYZdq{}}\PY{p}{:} \PY{p}{(}\PY{l+m+mi}{30000}\PY{p}{,} \PY{l+m+mi}{100000}\PY{p}{)}\PY{p}{,} \PY{l+s+s2}{\PYZdq{}}\PY{l+s+s2}{loan\PYZus{}range}\PY{l+s+s2}{\PYZdq{}}\PY{p}{:} \PY{p}{(}\PY{l+m+mi}{15000}\PY{p}{,} \PY{l+m+mi}{80000}\PY{p}{)}\PY{p}{\PYZcb{}}\PY{p}{,}
             \PY{l+s+s2}{\PYZdq{}}\PY{l+s+s2}{CCC}\PY{l+s+s2}{\PYZdq{}}\PY{p}{:} \PY{p}{\PYZob{}}\PY{l+s+s2}{\PYZdq{}}\PY{l+s+s2}{default\PYZus{}prob}\PY{l+s+s2}{\PYZdq{}}\PY{p}{:} \PY{l+m+mf}{0.25}\PY{p}{,} \PY{l+s+s2}{\PYZdq{}}\PY{l+s+s2}{income\PYZus{}range}\PY{l+s+s2}{\PYZdq{}}\PY{p}{:} \PY{p}{(}\PY{l+m+mi}{25000}\PY{p}{,} \PY{l+m+mi}{80000}\PY{p}{)}\PY{p}{,} \PY{l+s+s2}{\PYZdq{}}\PY{l+s+s2}{loan\PYZus{}range}\PY{l+s+s2}{\PYZdq{}}\PY{p}{:} \PY{p}{(}\PY{l+m+mi}{10000}\PY{p}{,} \PY{l+m+mi}{60000}\PY{p}{)}\PY{p}{\PYZcb{}}\PY{p}{,}
             \PY{l+s+s2}{\PYZdq{}}\PY{l+s+s2}{CC}\PY{l+s+s2}{\PYZdq{}}\PY{p}{:} \PY{p}{\PYZob{}}\PY{l+s+s2}{\PYZdq{}}\PY{l+s+s2}{default\PYZus{}prob}\PY{l+s+s2}{\PYZdq{}}\PY{p}{:} \PY{l+m+mf}{0.35}\PY{p}{,} \PY{l+s+s2}{\PYZdq{}}\PY{l+s+s2}{income\PYZus{}range}\PY{l+s+s2}{\PYZdq{}}\PY{p}{:} \PY{p}{(}\PY{l+m+mi}{20000}\PY{p}{,} \PY{l+m+mi}{70000}\PY{p}{)}\PY{p}{,} \PY{l+s+s2}{\PYZdq{}}\PY{l+s+s2}{loan\PYZus{}range}\PY{l+s+s2}{\PYZdq{}}\PY{p}{:} \PY{p}{(}\PY{l+m+mi}{8000}\PY{p}{,} \PY{l+m+mi}{40000}\PY{p}{)}\PY{p}{\PYZcb{}}\PY{p}{,}
             \PY{l+s+s2}{\PYZdq{}}\PY{l+s+s2}{C}\PY{l+s+s2}{\PYZdq{}}\PY{p}{:} \PY{p}{\PYZob{}}\PY{l+s+s2}{\PYZdq{}}\PY{l+s+s2}{default\PYZus{}prob}\PY{l+s+s2}{\PYZdq{}}\PY{p}{:} \PY{l+m+mf}{0.50}\PY{p}{,} \PY{l+s+s2}{\PYZdq{}}\PY{l+s+s2}{income\PYZus{}range}\PY{l+s+s2}{\PYZdq{}}\PY{p}{:} \PY{p}{(}\PY{l+m+mi}{15000}\PY{p}{,} \PY{l+m+mi}{50000}\PY{p}{)}\PY{p}{,} \PY{l+s+s2}{\PYZdq{}}\PY{l+s+s2}{loan\PYZus{}range}\PY{l+s+s2}{\PYZdq{}}\PY{p}{:} \PY{p}{(}\PY{l+m+mi}{5000}\PY{p}{,} \PY{l+m+mi}{20000}\PY{p}{)}\PY{p}{\PYZcb{}}\PY{p}{,}
             \PY{l+s+s2}{\PYZdq{}}\PY{l+s+s2}{D}\PY{l+s+s2}{\PYZdq{}}\PY{p}{:} \PY{p}{\PYZob{}}\PY{l+s+s2}{\PYZdq{}}\PY{l+s+s2}{default\PYZus{}prob}\PY{l+s+s2}{\PYZdq{}}\PY{p}{:} \PY{l+m+mf}{0.80}\PY{p}{,} \PY{l+s+s2}{\PYZdq{}}\PY{l+s+s2}{income\PYZus{}range}\PY{l+s+s2}{\PYZdq{}}\PY{p}{:} \PY{p}{(}\PY{l+m+mi}{10000}\PY{p}{,} \PY{l+m+mi}{30000}\PY{p}{)}\PY{p}{,} \PY{l+s+s2}{\PYZdq{}}\PY{l+s+s2}{loan\PYZus{}range}\PY{l+s+s2}{\PYZdq{}}\PY{p}{:} \PY{p}{(}\PY{l+m+mi}{2000}\PY{p}{,} \PY{l+m+mi}{10000}\PY{p}{)}\PY{p}{\PYZcb{}}\PY{p}{,}
         \PY{p}{\PYZcb{}}
         
         \PY{c+c1}{\PYZsh{} Generate the dataset with improved consistency}
         \PY{n}{num\PYZus{}samples} \PY{o}{=} \PY{l+m+mi}{10}
         \PY{n}{ratings} \PY{o}{=} \PY{n+nb}{list}\PY{p}{(}\PY{n}{rating\PYZus{}categories}\PY{o}{.}\PY{n}{keys}\PY{p}{(}\PY{p}{)}\PY{p}{)}
         
         \PY{n}{df\PYZus{}consistent\PYZus{}credit\PYZus{}risk} \PY{o}{=} \PY{n}{pd}\PY{o}{.}\PY{n}{DataFrame}\PY{p}{(}\PY{p}{)}
         
         \PY{c+c1}{\PYZsh{} Generate data row by row ensuring consistency}
         \PY{k}{for} \PY{n}{\PYZus{}} \PY{o+ow}{in} \PY{n+nb}{range}\PY{p}{(}\PY{n}{num\PYZus{}samples}\PY{p}{)}\PY{p}{:}
             \PY{n}{rating} \PY{o}{=} \PY{n}{np}\PY{o}{.}\PY{n}{random}\PY{o}{.}\PY{n}{choice}\PY{p}{(}\PY{n}{ratings}\PY{p}{)}  \PY{c+c1}{\PYZsh{} Select a credit rating}
             \PY{n}{rating\PYZus{}info} \PY{o}{=} \PY{n}{rating\PYZus{}categories}\PY{p}{[}\PY{n}{rating}\PY{p}{]}
         
             \PY{n}{income} \PY{o}{=} \PY{n}{np}\PY{o}{.}\PY{n}{random}\PY{o}{.}\PY{n}{randint}\PY{p}{(}\PY{o}{*}\PY{n}{rating\PYZus{}info}\PY{p}{[}\PY{l+s+s2}{\PYZdq{}}\PY{l+s+s2}{income\PYZus{}range}\PY{l+s+s2}{\PYZdq{}}\PY{p}{]}\PY{p}{)}  \PY{c+c1}{\PYZsh{} Income based on rating}
             \PY{n}{loan\PYZus{}amount} \PY{o}{=} \PY{n}{np}\PY{o}{.}\PY{n}{random}\PY{o}{.}\PY{n}{randint}\PY{p}{(}\PY{o}{*}\PY{n}{rating\PYZus{}info}\PY{p}{[}\PY{l+s+s2}{\PYZdq{}}\PY{l+s+s2}{loan\PYZus{}range}\PY{l+s+s2}{\PYZdq{}}\PY{p}{]}\PY{p}{)}  \PY{c+c1}{\PYZsh{} Loan based on rating}
             \PY{n}{age} \PY{o}{=} \PY{n}{np}\PY{o}{.}\PY{n}{random}\PY{o}{.}\PY{n}{randint}\PY{p}{(}\PY{l+m+mi}{18}\PY{p}{,} \PY{l+m+mi}{75}\PY{p}{)}  \PY{c+c1}{\PYZsh{} Random age}
             \PY{n}{employment\PYZus{}status} \PY{o}{=} \PY{n}{np}\PY{o}{.}\PY{n}{random}\PY{o}{.}\PY{n}{choice}\PY{p}{(}\PY{p}{[}\PY{l+s+s2}{\PYZdq{}}\PY{l+s+s2}{Employed}\PY{l+s+s2}{\PYZdq{}}\PY{p}{,} \PY{l+s+s2}{\PYZdq{}}\PY{l+s+s2}{Unemployed}\PY{l+s+s2}{\PYZdq{}}\PY{p}{,} \PY{l+s+s2}{\PYZdq{}}\PY{l+s+s2}{Self\PYZhy{}Employed}\PY{l+s+s2}{\PYZdq{}}\PY{p}{]}\PY{p}{)}
         
             \PY{c+c1}{\PYZsh{} Default history based on rating\PYZsq{}s probability}
             \PY{n}{default\PYZus{}history} \PY{o}{=} \PY{n}{np}\PY{o}{.}\PY{n}{random}\PY{o}{.}\PY{n}{choice}\PY{p}{(}\PY{p}{[}\PY{l+s+s2}{\PYZdq{}}\PY{l+s+s2}{Yes}\PY{l+s+s2}{\PYZdq{}}\PY{p}{,} \PY{l+s+s2}{\PYZdq{}}\PY{l+s+s2}{No}\PY{l+s+s2}{\PYZdq{}}\PY{p}{]}\PY{p}{,} \PY{n}{p}\PY{o}{=}\PY{p}{[}\PY{n}{rating\PYZus{}info}\PY{p}{[}\PY{l+s+s2}{\PYZdq{}}\PY{l+s+s2}{default\PYZus{}prob}\PY{l+s+s2}{\PYZdq{}}\PY{p}{]}\PY{p}{,} \PY{l+m+mi}{1} \PY{o}{\PYZhy{}} \PY{n}{rating\PYZus{}info}\PY{p}{[}\PY{l+s+s2}{\PYZdq{}}\PY{l+s+s2}{default\PYZus{}prob}\PY{l+s+s2}{\PYZdq{}}\PY{p}{]}\PY{p}{]}\PY{p}{)}
         
             \PY{c+c1}{\PYZsh{} Append row to DataFrame}
             \PY{n}{df\PYZus{}consistent\PYZus{}credit\PYZus{}risk} \PY{o}{=} \PY{n}{pd}\PY{o}{.}\PY{n}{concat}\PY{p}{(}\PY{p}{[}\PY{n}{df\PYZus{}consistent\PYZus{}credit\PYZus{}risk}\PY{p}{,} 
                 \PY{n}{pd}\PY{o}{.}\PY{n}{DataFrame}\PY{p}{(}\PY{p}{[}\PY{p}{[}\PY{n}{rating}\PY{p}{,} \PY{n}{income}\PY{p}{,} \PY{n}{age}\PY{p}{,} \PY{n}{employment\PYZus{}status}\PY{p}{,} \PY{n}{loan\PYZus{}amount}\PY{p}{,} \PY{n}{default\PYZus{}history}\PY{p}{]}\PY{p}{]}\PY{p}{,} 
                              \PY{n}{columns}\PY{o}{=}\PY{p}{[}\PY{l+s+s2}{\PYZdq{}}\PY{l+s+s2}{rating}\PY{l+s+s2}{\PYZdq{}}\PY{p}{,} \PY{l+s+s2}{\PYZdq{}}\PY{l+s+s2}{income}\PY{l+s+s2}{\PYZdq{}}\PY{p}{,} \PY{l+s+s2}{\PYZdq{}}\PY{l+s+s2}{age}\PY{l+s+s2}{\PYZdq{}}\PY{p}{,} \PY{l+s+s2}{\PYZdq{}}\PY{l+s+s2}{employment\PYZus{}status}\PY{l+s+s2}{\PYZdq{}}\PY{p}{,} \PY{l+s+s2}{\PYZdq{}}\PY{l+s+s2}{loan\PYZus{}amount}\PY{l+s+s2}{\PYZdq{}}\PY{p}{,} \PY{l+s+s2}{\PYZdq{}}\PY{l+s+s2}{default\PYZus{}history}\PY{l+s+s2}{\PYZdq{}}\PY{p}{]}\PY{p}{)}\PY{p}{]}\PY{p}{)}
         
         \PY{c+c1}{\PYZsh{} Reset index}
         \PY{n}{df\PYZus{}consistent\PYZus{}credit\PYZus{}risk}\PY{o}{.}\PY{n}{reset\PYZus{}index}\PY{p}{(}\PY{n}{drop}\PY{o}{=}\PY{k+kc}{True}\PY{p}{,} \PY{n}{inplace}\PY{o}{=}\PY{k+kc}{True}\PY{p}{)}
         
         \PY{n}{df\PYZus{}consistent\PYZus{}credit\PYZus{}risk}
\end{Verbatim}
\begin{Verbatim}[commandchars=\\\{\}]
{\color{outcolor}Out[{\color{outcolor}63}]:}   rating  income  age employment\_status  loan\_amount default\_history
         0     BB   89729   46        Unemployed        74326              No
         1    AAA  196986   68          Employed        80082              No
         2     CC   38722   38          Employed        38860              No
         3    CCC   43563   39        Unemployed        10379              No
         4      C   16028   28        Unemployed        18742             Yes
         5     BB   66969   44        Unemployed        86941              No
         6      A  115193   64        Unemployed       140246              No
         7      C   18750   22        Unemployed        11251              No
         8    CCC   69276   28          Employed        16086              No
         9      A  135037   62     Self-Employed       184408              No
\end{Verbatim}
    \subsection{References and Credits}\label{references-and-credits}

    \textbf{WEB}

\textbf{Abhyankar Ameya}, ``\emph{Exploring Risk Analytics using PCA
with Python}'',
\href{https://abhyankar-ameya.medium.com/exploring-risk-analytics-using-pca-with-python-3aca369cbfe4}{Medium},
data files for the interest rate example and further details about the
python code can be dowloaded from the github repository of the author
\href{https://github.com/Ameya1983/TheAlchemist}{here}


    % Add a bibliography block to the postdoc
    
    
    
\end{document}
